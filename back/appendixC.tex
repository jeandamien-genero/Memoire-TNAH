\renewcommand{\thesection}{C.1}
\chapter{C. Feuille de route et typologie des erreurs}

\section{Feuille de route}
\label{ann:feuille_route}

Cette liste reprend le texte des \issues{} ouvertes dans le \gitlab{} des \odm{} au commencement du stage.

\begin{enumerate}
        \item \textit{Trier et renommer les fichiers} :
    \begin{itemize}
        \item  Identifier les fichiers de monographies et les fichiers de paratexte ;
        \item  Donner un identifiant aux fichiers de monographies en fonction des identifiants déjà existant dans le fichier de référence ;
        \item  Créer un identifiant pour les fichiers de paratexte ;
        \item  Ajouter ces identifiants aux \texttt{@xml:id} de chaque fichier ;
        \item  Créer un \textit{mapping} de l’ensemble des fichiers sous la forme d’un CSV avec le nom du fichier et son identifiant ;
        \item  Créer un fichier \texttt{master.xml} contenant des renvois vers les autres fichiers grâce à des \texttt{<xi:includes>}.
    \end{itemize}

    \item \textit{Mettre à jour l’attribut \texttt{@url} dans la balise \texttt{<graphic>}} :
    \begin{itemize}
        \item  Les images des pages sont stockées localement sur Humanum, mais également en ligne sur Internet Archives.
        \item  Trouver l’url de chaque page de chaque volume sur \ia ;
        \item  Remplacer automatiquement le chemin local par l’url de l’image dans chaque fichier.
    \end{itemize}

    \item \textit{Tester la conformité du schéma} :
    \begin{itemize}
        \item  Écrire un script pour tester la validité des arbres XML de chaque fichier ;
        \item  Corriger les erreurs qui seraient signalées par ce script.
    \end{itemize}

    \item \textit{Intégrer les métadonnées des « enquêtés »} :
    \begin{itemize}
        \item  Créer un fichier référentiel de personnes (XML) à partir du fichier CSV de prosopographie ;
        \item  Ajouter aux paragraphes 2 des monographies ("§2. - Etat civil de la famille") des \texttt{@refs} au référentiel de personnes.
    \end{itemize}

    \item \textit{Intégrer modèle de citation dans les chapitres et créer un système de référence bibliographique} :
    \begin{itemize}
        \item  Pour chaque niveau de la structure d'une monographie  ;
        \item  Pour chaque chapitre (sans descendre dans les niveaux) ;
        \item  Par exemple en utilisant DTS.
    \end{itemize}

    \item \textit{Corrections des transcriptions} :
    \begin{itemize}
        \item  Paragraphe par paragraphe, implémenter une correction automatique des transcriptions.
    \end{itemize}

    \item \textit{Corriger le passage de source vers split} :
    \begin{itemize}
        \item  Corriger le script python de transformation des fichiers sources en une suite de fichiers XML TEI.
        \item  Éliminer les traces de teiCorpus.
    \end{itemize}

    \item \textit{Simplifier l'implémentation de la structure logique} :
    \begin{itemize}
        \item Aplatir la structure des monographies :
        \begin{itemize}
            \item  Est-il vraiment nécessaire d'utiliser les div enchâssées ? Une structure à plat avec des marqueurs signalant le début d'une nouvelle section ne suffirait-elle pas ?
            \item  Comment gérer l'articulation entre l'arbre principal (un arbre pour l'ensemble des monographies) et les sous-arbres (un sous-arbre par monographie) ? L'ensemble du corpus n'a pas de front/back mais chaque volume possède un front et un back et chaque monographie a potentiellement un front et un back.
            \item  Avec quelles restrictions peut-on créer une structure similaire à celle d'un fichier \LaTeX{} avec ses imports ?
        \end{itemize}
    \end{itemize}
\end{enumerate}

\clearpage
\renewcommand{\thesection}{C.2}
\section{Relevé des erreurs dans la structure
logique}\label{ann:releve_erreurs}

Jean-Damien Généro, 8 juillet 2020\footnote{Reproduction d'une synthèse mise en ligne sur une issue \textit{GitLab}.}.

\subsection{Déficit dans la
transcription}\label{ann:deficit-transcr}

\begin{itemize}
\item
  \emph{Déficit partiel :} \texttt{s1t2\_chapt\_11} et \texttt{s1t3\_chapt\_10} (manque
  début §7), \texttt{s1t2\_ch\-apt\_4} (manquent quatre pages de la note A), \texttt{\texttt{s2t3\_chapt\_14}} (manquent vingt lignes).
\item
  \emph{Déficit majeur :} \texttt{s1t4\_chapt\_8}, \texttt{11}, \texttt{15} et \texttt{s1t5\_chapt\_13} et
  14 (seules les notes ont été transcrites, le reste se trouve dans des
  \texttt{\textless{}figure\textgreater{}}), \texttt{s1t5\_chapt\_12}
  (transcription à partir du §14, le reste dans des
  \texttt{\textless{}figure\textgreater{}}).
\end{itemize}

\subsection[Titres non imprimés]{Titres manquants parce que non imprimés dans les exemplaires
d'\ia}\label{ann:titre-non-imprimes}

\begin{itemize}
\item
  \texttt{s1t1\_chapt\_12} : manque le titre de la s/section II.3 \textit{Mode
  d'existence\ldots{}} ;
\item
  \texttt{s2t1\_chapt\_5} : manque le titre de la s/section II.4 \textit{Histoire
  de\ldots{}} ;
\item
  \texttt{s2t1\_chapt\_20} : manque les titres des s/sections II.3 \textit{Mode
  d'existence\ldots{}} et II.4 \textit{Histoire de\ldots{}} (mais ce titre
  est repris dans l'intitulé du §12 qui suit) ;
\item
  \texttt{s2t4\_chapt\_9} : manque les titres de la s/section II.1 \textit{Définition
  du lieu\ldots{}} ;
\item
  \texttt{s3t1\_chapt\_3} : manque le titre des s/sections II.2 \textit{Moyens
  d'existence\ldots{}}, II.3 \textit{Mode d'existence\ldots{}} et II.4
  \textit{Histoire de\ldots{}} (mais ce titre est repris dans l'intitulé du
  §12 qui suit) ;
\item
  \texttt{s2t5\_chapt\_15} : manque le titre de la s/section II.4 \textit{Histoire
  de\ldots{}}.
\end{itemize}

\subsection{Structure allégée}\label{ann:structure-allegee}

Titres de s/s/section intégrés au début de paragraphe + certains titres
non utilisés (spec. §16).

\begin{itemize}
\item
  \texttt{s2t2\_chapt\_7} (précis) : pas de s/s/section §16 et de titre pour la
  partie \textit{Notes} (composée d'une seule \emph{nota}) ;
\item
  \texttt{s2t2\_chapt\_9} (précis) : pas de titre de section II\textit{ Observations
  préliminaires\ldots{}} et pas de s/s/section §16 ;
\item
  \texttt{s2t2\_chapt\_11} (précis) : pas de titre de section II \textit{Observations
  préliminaires\ldots{}}, pas de titre pour les s/s/sections (sauf les
  paragraphes de budget §14 et §15), pas de s/s/section §16, pas de
  section \textit{Notes}.
\item
  \texttt{s2t3\_chapt\_7} (précis) : pas de titre de section II \textit{Observations
  préliminaires\ldots{}}, pas de s/s/section §16, les titres des
  s/s/sections sont en début de paragraphe ;
\item
  \texttt{s2t5\_chapt\_9} (précis) : dans la section II, tous les titres des
  s/s/sections sont en début de paragraphe (à l'exception des
  paragraphes de budget §14 et §15), il n'y a pas de s/s/section §16 ;
\item
  \texttt{s2t5\_chapt\_16} (précis) : pas de s/s/section dans la s/section II.4
  \textit{Histoire de\ldots{}}, pas de s/s/section §16, la section \textit{Notes} est
  occupée par la traduction d'un document.
\end{itemize}

\subsection{Pas de s/s/section §16 et pas de section
\textit{Notes}}\label{ann:no-notes-no-16}

\begin{itemize}
\item
  (voir \emph{supra} \texttt{s2t2\_chapt\_11}) ;
\item
  \texttt{s2t1\_chapt\_6} et \texttt{s2t5\_chapt\_12}.
\end{itemize}

\subsection{Remarques et cas
particuliers}\label{ann:remarques-et-cas-particuliers}

\begin{itemize}
\item
  Dans \texttt{s1t5\_chapt\_9}, \texttt{10} et \texttt{11}, une section des notes est intitulée
  \textit{Précis de monographie} et organisée comme un précis (càd avec des
  sections et des titres). Néanmoins il ne s'agit pas de précis tels
  qu'on en trouve dans les séries 2 et 3, puisqu'ils ne constituent pas
  un ensemble logique indépendant de la monographie (il y a des
  paragraphes avant et des paragraphes après). Donc le précis est
  considéré comme une sub\_sub\_section et ses divisions comme des
  sub\_sub\_sub\_section.
\item
  \texttt{s2t5\_chapt\_17} et \texttt{18} découpés chacun en deux fichiers (table
  analytique et table des matières).
\item
  \texttt{s3t1\_chapt\_2} (\textit{Société générale\ldots{}}) et \texttt{s3t2\_chapt\_10}
  (\textit{Usine hydraulique\ldots{}}) sont des cas particuliers avec des
  structures totalement différentes des autres monographies. En plus de
  l'en-tête, le premier se divise en trois grandes sections :
  observations préliminaires (sorte d'intro), première partie, deuxième
  partie, troisième partie, conclusion et appendices ; le second est
  plus succinct avec l'en-tête, une grande section avec des s/sections
  numérotées et ensuite des appendices.
\item
  Dans \texttt{s3t2\_chapt\_9} se trouve une partie \textit{Note} sur l'état de la
  famille en 1905 entre le §13 et §14, que j'ai structurée comme une
  s/s/section.
\item
  La s/section II.5, consacrée aux budgets (§14, §15, §16), n'est pas
  titrée sauf dans les précis de monographies \texttt{s2t1\_chapt\_6},
  \texttt{s2t2\_chapt\_9} et 11, \texttt{s2t3\_chapt\_7} et 14, \texttt{s2t5\_chapt\_9} et 13,
  \texttt{s3t1\_chapt\_4} (le titre est \textit{Budget domestique annuel}).
\end{itemize}