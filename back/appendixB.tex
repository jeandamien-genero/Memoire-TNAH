\renewcommand{\thesection}{B.1}
\chapter{B. Structure logique}

\section{Structure logique des monographies}
\label{structure}

Document établi à partir d'un relevé effectué par Alix Chagué.

\begin{enumerate}[A.]
    \item \textbf{\textit{Titre.}}
    \item \textbf{\textit{Observations préliminaires définissant la condition des divers membres de la famille.}}
    \begin{enumerate}[I.]
        \item \textbf{\textit{Définition du lieu, de l'organisation industrielle et de la famille.}}
        \begin{enumerate}[]
            \item \textit{§ 1. État du sol, de l'industrie et de la population.}
            \item \textit{§ 2. État civil de la famille.}
            \item \textit{§ 3. Religion et habitudes morales.}
            \item \textit{§ 4. Hygiène et services de santé.}
            \item \textit{§ 5. Rang de la famille.}
        \end{enumerate}
        \item \textbf{\textit{Moyens d'existence de la famille.}}
        \begin{enumerate}[]
            \item \textit{§ 6. Propriétés.}
            \item \textit{§ 7. Subventions.}
            \item \textit{§ 8. Travaux et industries.}
        \end{enumerate}
        \item \textbf{\textit{Mode d'existence de la famille.}}
        \begin{enumerate}[]
            \item \textit{§ 9. Aliments et repas.}
            \item \textit{§ 10. Habitation, mobilier et vêtements.}
            \item \textit{§ 11. Récréations.}
        \end{enumerate}
        \item \textbf{\textit{Histoire de la famille.}}
        \begin{enumerate}[]
            \item \textit{§ 12. Phases principales de l'existence.}
            \item \textit{§ 13. M\oe{}urs et institutions assurant le bien-être physique et moral de la famille.}
        \end{enumerate}
        \item (\textbf{\textit{Budget domestique annuel}}\footnote{Cette section ne possède un titre que dans huit monographies}).
        \begin{enumerate}[]
            \item \textit{§ 14. Budget des recettes de l'année.}
            \item \textit{§ 15. Budget des dépenses de l'année.}
            \item \textit{Comptes annexés aux budgets} \footnotesize{(n° 1 à 84) puis} \textit{§ 16. Comptes annexés aux budgets.}
        \end{enumerate}
    \end{enumerate}
    \item \textbf{\textit{Notes}} \footnotesize{(n° 1 à 84) puis} \textbf{\textit{Éléments divers de la constitution sociale}}.
    \begin{enumerate}[]
            \item (A) \textbf{\textit{(titre du paragraphe)}} \footnotesize{(n° 1 à 84) puis} § 17. \textbf{\textit{(titre du paragraphe)}}.
            \item (B) \textbf{\textit{(titre du paragraphe)}} \footnotesize{(n° 1 à 84) puis} § 18. \textbf{\textit{(titre du paragraphe)}}.
            \item \textbf{\textit{etc.}}
        \end{enumerate}
\end{enumerate}

\clearpage

\renewcommand{\thesection}{B.2}
\section{Encodage de la structure logique}
\label{ann:encodage-structure}

Ce code a pour but de présenter l'enchaînement des \texttt{<div>}, \texttt{<head>} et \texttt{<p>} au sein du \texttt{<body>}. Tous les attributs, à l'exception de \texttt{<@type>} et \texttt{<@n>}, ont été retirés.

\begin{lstlisting}[language=XML,breaklines]
<div type="chapter">
    <div n="001" type="section">
        <pb facs="#" />
        <p>numéro</p>
        <p>titre</p>
        <p>sous-titre</p>
        <p>date</p>
        <p>auteur</p>
    </div>
    <div n="002" type="section">
        <head type="section" >Observations préliminaires</head>
        <div n="001" type="sub_section">
            <head type="sub_section" >I. Définition du lieu, de l'organisation industrielle et de la famille</head>
            <div n="001" type="sub_sub_section">
                <head type="sub_sub_section" >§ 1ᵉʳ. - État du sol, de l'industrie et de la population.</head>
                <p>texte</p>
            </div>
            <div n="002" type="sub_sub_section">
                <head type="sub_sub_section" >§ 2. - État civil de la famille.</head>
                <p>texte</p>
            </div>
            <div n="003" type="sub_sub_section">
                <head type="sub_sub_section" >§ 3. - Religion et habitudes morales.</head>
                <p>texte</p>
            </div>
            <div n="004" type="sub_sub_section">
                <head type="sub_sub_section" >§ 4. - Hygiène et service de santé.</head>
                <p>texte</p>
            </div>
            <div n="005" type="sub_sub_section">
                <head type="sub_sub_section" >§ 5. - Rang de la famille.</head>
                <p>texte</p>
            </div>
        </div>
        <div n="002" type="sub_section">
            <head type="sub_section" >II. Moyens d'existence de la famille</head>
            <div n="001" type="sub_sub_section">
                <head type="sub_sub_section" >§ 6. - Propriétés.</head>
                <p>texte</p>
            </div>
            <div n="002" type="sub_sub_section">
                <head type="sub_sub_section" >§ 7. - Subventions.</head>
                <p>texte</p>
            </div>
            <div n="003" type="sub_sub_section">
                <head type="sub_sub_section" >§ 8. - Travaux et industries.</head>
                <pb facs="#" />
                <p>texte</p>
            </div>
        </div>
        <div n="003" type="sub_section">
            <head type="sub_section" >III. Mode d'existence de la famille</head>
            <div n="001" type="sub_sub_section">
                <head type="sub_sub_section" >§ 9. - Aliments et repas.</head>
                <p>texte</p>
            </div>
            <div n="002" type="sub_sub_section">
                <head type="sub_sub_section" >§ 10. - Habitation, mobilier et vêtements.</head>
                <p>texte</p>
            </div>
            <div n="003" type="sub_sub_section">
                <pb facs="#" />
                <head type="sub_sub_section" >§ 11. - Récréations.</head>
                <p>texte</p>
            </div>
        </div>
        <div n="004" type="sub_section">
            <pb facs="#" />
            <head type="sub_section" >IV. Histoire de la famille</head>
            <div n="001" type="sub_sub_section">
                <head type="sub_sub_section" >§ 12. - Phases principales de l'existence.</head>
                <p>texte</p>
            </div>
            <div n="002" type="sub_sub_section">
                <head type="sub_sub_section" >§ 13. - Moeurs et institutions assurant le  bien-être physique et moral de la famille.</head>
                <p>texte</p>
            </div>
        </div>
        <div n="005" type="sub_section">
            <div n="001" type="sub_sub_section">
                <pb facs="#" />
                <head type="sub_sub_section" >§ 14. - Budget des recettes de l'année.</head>
                <figure type="illustration">
                    <graphic facs="#" />
                </figure>
            </div>
            <div n="002" type="sub_sub_section">
                <pb facs="#" />
                <head type="sub_sub_section" >§ 15. - Budget des dépenses de l'année.</head>
                <figure>
                    <graphic facs="#" />
                </figure>
            </div>
            <div n="003" type="sub_sub_section">
                <pb facs="#" />
                <head type="sub_sub_section">Comptes annexés aux budgets.</head>
                <figure>
                    <graphic facs="#" />
                </figure>
            </div>
        </div>
        <div n="003" type="section">
            <pb facs="#" />
            <head type="section" >NOTES</head>
            <div n="001" type="sub_section">
                <div n="001" type="sub_sub_section">
                    <head type="sub_sub_section" >(A) titre</head>
                    <p>texte</p>
                </div>
            </div>
        </div>
    </div>
</div>
\end{lstlisting}