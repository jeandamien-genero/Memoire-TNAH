\part*{Annexes}
\addcontentsline{toc}{part}{Annexes}
\appendix
\renewcommand{\thechapter}{A}
\chapter{A. \lodm}

\section{Liste des monographies et fichiers correspondant}
\label{mapping}

\textit{La date de publication du volume est indiquée entre parenthèses, tandis que les dates de parution des fascicules le sont entre crochets droits.}

\subsection{Série 1, vol. 1 (1857).}

\begin{center}
\begin{longtable}{ | c | p{9.5cm} | c | }
\hline
Id & Intitulé & Fichier \\ \hline
\citecode{401a} & Page de titre & \citecode{s1t1\_chapt\_1.xml} \\ \hline
\citecode{402a} & Avertissement. Considérations générales sur la Société internationale des études pratiques d'économie sociale. Son but et ses moyens d'action. & \citecode{s1t1\_chapt\_2.xml} \\ \hline
\citecode{403a} & Institution. Société internationale des études pratiques d'économie sociale. Fondation et premiers travaux. & \citecode{s1t1\_chapt\_3.xml} \\ \hline
\citecode{404a} & Définitions par ordre alphabétiques des termes à employer dans les monographies, pour désigner les ouvriers, leurs moyens d'existence, et les rapports qui les unissent soit entre eux, soit avec les autres classes. & \citecode{s1t1\_chapt\_4.xml} \\ \hline
\citecode{405a} & Explications des signes de renvoi et des abréviations. & \citecode{s1t1\_chapt\_5.xml} \\ \hline
\citecode{001a} & Charpentier de Paris (Seine - France), de la Corporation des compagnons du Devoir & \citecode{s1t1\_chapt\_6.xml} \\ \hline
\citecode{002a} & Manœuvre-Agriculteur de la Champagne pouilleuse (Marne - France) & \citecode{s1t1\_chapt\_7.xml} \\ \hline
\citecode{003a} & Paysans en communauté du Lavedan (Hautes-Pyrénées - France) & \citecode{s1t1\_chapt\_8.xml} \\ \hline
\citecode{004a} & Paysan du Labourd (Basses-Pyrénées - France) & \citecode{s1t1\_chapt\_9.xml} \\ \hline
\citecode{005a} & Métayer de la banlieue de Florence (Grand-Duché de Toscane) & \citecode{s1t1\_chapt\_10.xml} \\ \hline
\citecode{006a} & Nourrisseur de vaches de la banlieue de Londres (Middlesex - Angleterre) & \citecode{s1t1\_chapt\_11.xml} \\ \hline
\citecode{007a} & Tisseur en châles de la fabrique urbaine collective de Paris (Seine - France) & \citecode{s1t1\_chapt\_12.xml} \\ \hline
\citecode{008a} & Manœuvre-agriculteur du comté de Nottingham (Angleterre) & \citecode{s1t1\_chapt\_13.xml} \\ \hline
\citecode{009a} & Pêcheur côtier, maître de barque de Saint-Sébastien (Guipuscoa - Espagne) & \citecode{s1t1\_chapt\_14.xml} \\ \hline
\citecode{406a} & Tables alphabétique et analytique des matières contenues dans ce tome premier. & \citecode{s1t1\_chapt\_15.xml} \\ \hline
\citecode{407a} & Liste des monographies destinées aux prochaines publications de la société d'économie sociale. & \citecode{s1t1\_chapt\_16.xml} \\ \hline
\citecode{408a} & Tables des matières contenues dans ce tome premier & \citecode{s1t1\_chapt\_17.xml} \\ \hline
\end{longtable}
\end{center}

\subsection{Série 1, vol. 2 (1858).}

\begin{center}
\begin{longtable}{ | c | p{9.5cm} | c | }
\hline
Id & Intitulé & Fichier \\ \hline
\citecode{409a} & Page de titre & \citecode{s1t2\_chapt\_1.xml} \\ \hline
\citecode{410a} & Avertissement & \citecode{s1t2\_chapt\_2.xml} \\ \hline
\citecode{010a} & Ferblantier, couvreur et vitrier d'Aix-les-Bains (Savoie - États Sardes) & \citecode{s1t2\_chapt\_3.xml} \\ \hline
\citecode{011a} & Carrier des environs de Paris (Seine - France) & \citecode{s1t2\_chapt\_4.xml} \\ \hline
\citecode{012a} & Menuisier-charpentier (Nedjar) de Tanger (Province de Tanger - Maroc) & \citecode{s1t2\_chapt\_5.xml} \\ \hline
\citecode{013a} & Tailleur d'habits de Paris (Seine - France) & \citecode{s1t2\_chapt\_6.xml} \\ \hline
\citecode{014a} & Compositeur-typographe de Bruxelles (Brabant - Belgique) & \citecode{s1t2\_chapt\_7.xml} \\ \hline
\citecode{015a} & Décapeur d'outils en acier de la fabrique d'Hérimoncourt (Doubs - France) & \citecode{s1t2\_chapt\_8.xml} \\ \hline
\citecode{016a} & Monteur d'outils en acier de la fabrique d'Hérimoncourt (Doubs - France) & \citecode{s1t2\_chapt\_9.xml} \\ \hline
\citecode{017a} & Porteur d'eau de Paris (Seine - France) & \citecode{s1t2\_chapt\_10.xml} \\ \hline
\citecode{018a} & Paysans en communauté et en polygamie de Bousrah (Esky Cham), dans le pays de Haouran (Syrie - Empire Ottoman) & \citecode{s1t2\_chapt\_11.xml} \\ \hline
\citecode{019a} & Débardeur et piocheur de craie de la banlieue de Paris (Seine - France) & \citecode{s1t2\_chapt\_12.xml} \\ \hline
\citecode{411a} & Tables alphabétique et analytique des matières contenues dans ce tome second & \citecode{s1t2\_chapt\_13.xml} \\ \hline
\citecode{412a} & Errata & \citecode{s1t2\_chapt\_14.xml} \\ \hline
\citecode{413a} & Tables des matières contenues dans ce tome second & \citecode{s1t2\_chapt\_15.xml} \\ \hline
\end{longtable}
\end{center}

\subsection{Série 1, vol. 3 (1861).}

\begin{center}
\begin{longtable}{ | c | p{9.5cm} | c | }
\hline
Id & Intitulé & Fichier \\ \hline
\citecode{414a} & Page de titre & \citecode{s1t3\_chapt\_1.xml} \\ \hline
\citecode{415a} & Avertissement & \citecode{s1t3\_chapt\_2.xml} \\ \hline
\citecode{416a} & Rapport. Société d'économie sociale. Travaux de 1859-1860 & \citecode{s1t3\_chapt\_3.xml} \\ \hline
\citecode{417a} & Liste générale des membres de la Société internationale des études pratiques d'économie sociale & \citecode{s1t3\_chapt\_4.xml} \\ \hline
\citecode{020a} & Brodeuses des Vosges (Vosges - France) & \citecode{s1t3\_chapt\_5.xml} \\ \hline
\citecode{021a} & Paysan et savonnier de la Basse-Provence (Bouches-du-Rhône - France) & \citecode{s1t3\_chapt\_6.xml} \\ \hline
\citecode{022a} & Mineur des Placers du comté de Mariposa (Californie - États-Unis) & \citecode{s1t3\_chapt\_7.xml} \\ \hline
\citecode{023a} & Manœuvre-vigneron de l'Aunis (Charente-inférieure - France) & \citecode{s1t3\_chapt\_8.xml} \\ \hline
\citecode{024a} & Lingère de Lille (Nord - France) & \citecode{s1t3\_chapt\_9.xml} \\ \hline
\citecode{025a} & Parfumeur de Tunis (Régence de Tunis - Afrique) du bazar appelé : El Attharin-el-kebar (les grands parfumeurs) & \citecode{s1t3\_chapt\_10.xml} \\ \hline
\citecode{026a} & Instituteur primaire d'une commune rurale de la Normandie (Eure - France) & \citecode{s1t3\_chapt\_11.xml} \\ \hline
\citecode{027a} & Manœuvre à famille nombreuse de Paris (Seine - France) & \citecode{s1t3\_chapt\_12.xml} \\ \hline
\citecode{028a} & Fondeur de plomb des Alpes Apuanes (Toscane - Italie) & \citecode{s1t3\_chapt\_13.xml} \\ \hline
\citecode{418a} & Tables alphabétique et analytique des matières contenues dans ce tome troisième & \citecode{s1t3\_chapt\_14.xml} \\ \hline
\citecode{419a} & Errata de ce tome troisième & \citecode{s1t3\_chapt\_15.xml} \\ \hline
\citecode{420a} & Tables des matières contenues dans ce tome troisième & \citecode{s1t3\_chapt\_16.xml} \\ \hline
\end{longtable}
\end{center}

\subsection{Série 1, vol. 4 (1862).}

\begin{center}
\begin{longtable}{ | c | p{9.5cm} | c | }
\hline
Id & Intitulé & Fichier \\ \hline
\citecode{421a} & Page de titre & \citecode{s1t4\_chapt\_1.xml} \\ \hline
\citecode{422a} & Explications des signes de renvoi et des abréviations. & \citecode{s1t4\_chapt\_2.xml} \\ \hline
\citecode{423a} & Avertissement & \citecode{s1t4\_chapt\_3.xml} \\ \hline
\citecode{424a} & Rapport. Société d'économie sociale. Travaux de 1860-1861 & \citecode{s1t4\_chapt\_4.xml} \\ \hline
\citecode{425a} & Instruction. Méthode d'observation des monographies de famille propre à l'ouvrage intitulé Les ouvriers européens & \citecode{s1t4\_chapt\_5.xml} \\ \hline
\citecode{426a} & Histoire de la famille. Prix fondé par M. le baron de Damas. Par la société d'économie sociale & \citecode{s1t4\_chapt\_6.xml} \\ \hline
\citecode{029a} & Paysan d'un village à banlieue morcelée du Laonnais (Aisne - France) & \citecode{s1t4\_chapt\_7.xml} \\ \hline
\citecode{030a} & Paysans en communauté du Ning-Po-Fou (province de Tché-Kian - Chine) & \citecode{s1t4\_chapt\_8.xml} \\ \hline
\citecode{031a} & Mulâtre affranchi de l'Ile de la Réunion (Océan Indien) & \citecode{s1t4\_chapt\_9.xml} \\ \hline
\citecode{032a} & Manœuvre-vigneron de la Basse-Bourgogne (Yonne - France) & \citecode{s1t4\_chapt\_10.xml} \\ \hline
\citecode{033a} & Compositeur-typographe de Paris (Seine - France) & \citecode{s1t4\_chapt\_11.xml} \\ \hline
\citecode{034a} & Auvergnat brocanteur en boutique à Paris (Seine - France) & \citecode{s1t4\_chapt\_12.xml} \\ \hline
\citecode{035a} & Mineur de la Maremme de Toscane (Toscane - Italie) & \citecode{s1t4\_chapt\_13.xml} \\ \hline
\citecode{036a} & Tisserand des Vosges (Haut-Rhin - France) & \citecode{s1t4\_chapt\_14.xml} \\ \hline
\citecode{037a} & Pêcheur côtier, maître de barques, de Marken (Hollande septentrionale - Pays-Bas) & \citecode{s1t4\_chapt\_15.xml} \\ \hline
\citecode{427a} & Société internationale des études pratiques d'économie sociale. Officiers composants les conseils d'administration et de surveillance pour la session 1863-1864. & \citecode{s1t4\_chapt\_16.xml} \\ \hline
\citecode{428a} & Liste générale des membres de la Société internationale des études pratiques d'économie sociale au 1er août 1863 & \citecode{s1t4\_chapt\_17.xml} \\ \hline
\citecode{429a} & Tables alphabétique et analytique des matières contenues dans ce tome quatrième & \citecode{s1t4\_chapt\_18.xml} \\ \hline
\citecode{430a} & Errata de ce tome quatrième & \citecode{s1t4\_chapt\_19.xml} \\ \hline
\citecode{431a} & Tables des matières contenues dans ce tome quatrième & \citecode{s1t4\_chapt\_20.xml} \\ \hline
\end{longtable}
\end{center}

\subsection{Série 1, vol. 5 [1875, 1883, 1884] (1885).}

\begin{center}
\begin{longtable}{ | c | p{9.5cm} | c | }
\hline
Id & Intitulé & Fichier \\ \hline
\citecode{432a} & Page de titre & \citecode{s1t5\_chapt\_1.xml} \\ \hline
\citecode{433a} & Oeuvres de F. Le Play & \citecode{s1t5\_chapt\_2.xml} \\ \hline
\citecode{434a} & Sommaire. Monographies de familles publiées dans ce volume & \citecode{s1t5\_chapt\_3.xml} \\ \hline
\citecode{435a} & Avertissement & \citecode{s1t5\_chapt\_4.xml} \\ \hline
\citecode{436a} & Explications des signes de renvoi et des abréviations employés dans le cours de cet ouvrage & \citecode{s1t5\_chapt\_5.xml} \\ \hline
\citecode{038a} & Fermiers à communauté taisible du Nivernais (Saône-et-Loire - France) & \citecode{s1t5\_chapt\_6.xml} \\ \hline
\citecode{039a} & Paysan de Saint-Irénée (Bas-Canada - Amérique du Nord) & \citecode{s1t5\_chapt\_7.xml} \\ \hline
\citecode{040a} & L'Ouvrier éventailliste de Sainte-Geneviève (Oise - France) & \citecode{s1t5\_chapt\_8.xml} \\ \hline
\citecode{041a} & Ouvrier cordonnier de Malakoff (Seine - France) & \citecode{s1t5\_chapt\_9.xml} \\ \hline
\citecode{041b} & Précis d'une monographie ayant pour objet un chiffonnier instable et, par alternance, mégissier fumiste et brossier de Paris (France - Seine) & \textit{In supra.} \\ \hline
\citecode{042a} & Serrurier-forgeron de Paris (Seine - France) & \citecode{s1t5\_chapt\_10.xml} \\ \hline
\citecode{042b} & Précis d'une monographie ayant pour objet le monteur en bronze de Paris & \textit{In supra.} \\ \hline
\citecode{043a} & Brigadier de la Garde républicaine de Paris (Seine - France) & \citecode{s1t5\_chapt\_11.xml} \\ \hline
\citecode{044a} & Paysan-résinier de Lévignacq (Landes - France) & \citecode{s1t5\_chapt\_12.xml} \\ \hline
\citecode{045a} & Bûcheron usager de l'ancien Comté de Dabo (Lorraine allemande) & \citecode{s1t5\_chapt\_13.xml} \\ \hline
\citecode{046a} & Paysans en communauté et colporteurs émigrants de Tabou-Douchd-El-Baar (Grande Kabylie - Province d'Alger) & \citecode{s1t5\_chapt\_14.xml} \\ \hline
\citecode{437a} & Société d'économie sociale. Conseil d'administration pour l'année 1885 & \citecode{s1t5\_chapt\_15.xml} \\ \hline
\citecode{438a} &  Liste générale des membres de la Société d'économie sociale au 15 mars 1885 & \citecode{s1t5\_chapt\_16.xml} \\ \hline
\citecode{439a} & Tables alphabétique et analytique des matières contenues dans ce tome cinquième & \citecode{s1t5\_chapt\_17.xml} \\ \hline
\citecode{440a} & Tables des matières contenues dans ce tome cinquième & \citecode{s1t5\_chapt\_18.xml} \\ \hline
\end{longtable}
\end{center}

\subsection{Série 2, vol. 1 [1885-1887] (1887).}

\begin{center}
\begin{longtable}{ | c | p{9.5cm} | c | }
\hline
Id & Intitulé & Fichier \\ \hline
\citecode{441a} & Page de titre & \citecode{s2t1\_chapt\_1.xml} \\ \hline
\citecode{442a} & Sommaire des monographies de familles publiées dans ce volume & \citecode{s2t1\_chapt\_2.xml} \\ \hline
\citecode{443a} & Avertissement sur ce premier volume, deuxième série des Ouvriers des deux mondes & \citecode{s2t1\_chapt\_3.xml} \\ \hline
\citecode{047a} & Paysan-paludier du Bourg de Batz (Loire-Inférieure - France) & \citecode{s2t1\_chapt\_4.xml} \\ \hline
\citecode{048b} & Précis d'une monographie de l'armurier des manufactures impériales de Toula (Grande-Russie) & \citecode{s2t1\_chapt\_6.xml} \\ \hline
\citecode{048a} & Bordiers émancipés en communauté rurale de la Grande-Russie & \citecode{s2t1\_chapt\_5.xml} \\ \hline
\citecode{049a} & Charron des forges et fonderies de Montataire (Oise - France) & \citecode{s2t1\_chapt\_16.xml} \\ \hline
\citecode{050a} & Faienciers de Nevers (Nièvre - France) & \citecode{s2t1\_chapt\_17.xml} \\ \hline
\citecode{051a} & Cultivateur-maraicher de Deuil (Seine-et-Oise - France) & \citecode{s2t1\_chapt\_18.xml} \\ \hline
\citecode{052a} & Pêcheur-côtier, maître de barque, de Martigues (Bouches-du-Rhône - France) & \citecode{s2t1\_chapt\_19.xml} \\ \hline
\citecode{053a} & Métayer à famille-souche du pays d'Horte (Landes - France) & \citecode{s2t1\_chapt\_20.xml} \\ \hline
\citecode{054a} & Arabes pasteurs nomades de la tribu des Larbas (Région saharienne de l'Algérie) & \citecode{s2t1\_chapt\_21.xml} \\ \hline
\citecode{055a} & Gantier de Grenoble (Isère - France) & \citecode{s2t1\_chapt\_22.xml} \\ \hline
\citecode{444a} & Tables alphabétique et analytique des matières contenues dans le présent volume & \citecode{s2t1\_chapt\_24.xml} \\ \hline
\citecode{445a} & Table des matières dans ce tome premier (deuxième série) & \citecode{s2t1\_chapt\_25.xml} \\ \hline
\end{longtable}
\end{center}

\subsection{Série 2, vol. 2 [1887-1889] (1890).}

\begin{center}
\begin{longtable}{ | c | p{9.5cm} | c | }
\hline
Id & Intitulé & Fichier \\ \hline
\citecode{446a} & Page de titre & \citecode{s2t2\_chapt\_1.xml} \\ \hline
\citecode{447a} & Page de titre & \citecode{s2t2\_chapt\_2.xml} \\ \hline
\citecode{448a} & Sommaire des monographies de familles publiées dans ce volume & \citecode{s2t2\_chapt\_3.xml} \\ \hline
\citecode{449a} & Avertissement sur ce deuxième tome de la deuxième série des Ouvriers des deux mondes & \citecode{s2t2\_chapt\_4.xml} \\ \hline
\citecode{056a} & Tourneur-mécanicien des usines de la Société Cockerill, de Seraing (Belgique) & \citecode{s2t2\_chapt\_5.xml} \\ \hline
\citecode{057a} & Bordier (Fellah) berbère de la Grande-Kabylie (Province d'Alger) & \citecode{s2t2\_chapt\_6.xml} \\ \hline
\citecode{057b} & Précis d'une monographie du paysan colon du Sahel (Algérie) & \citecode{s2t2\_chapt\_7.xml} \\ \hline
\citecode{058a} & Pêcheur côtier d'Heyst (Flandre occidentale - Belgique) & \citecode{s2t2\_chapt\_8.xml} \\ \hline
\citecode{058b} & Précis d'une monographie du pêcheur côtier, maître de barque, d'Étretat (Seine-Inférieure - France) & \citecode{s2t2\_chapt\_9.xml} \\ \hline
\citecode{059a} & Paysan-métayer de la Basse Provence (Bouches-du-Rhône - France) & \citecode{s2t2\_chapt\_10.xml} \\ \hline
\citecode{059b} & Précis d'une monographie du paysan et maçon émigrant de la Marche (Creuse - France) & \citecode{s2t2\_chapt\_11.xml} \\ \hline
\citecode{060a} & Mineur silésien du bassin houiller de la Ruhr (Prusse rhénane - Allemagne) & \citecode{s2t2\_chapt\_12.xml} \\ \hline
\citecode{061a} & Mineur des soufrières de Lercara (Province de Palerme - Sicile & \citecode{s2t2\_chapt\_13.xml} \\ \hline
\citecode{062a} & Tailleur de Silex et vigneron de l'Orléanais (Loir-et-Cher - France) & \citecode{s2t2\_chapt\_14.xml} \\ \hline
\citecode{063a} & Vigneron précariste et métayer de Valmontone (Province de Rome - Italie) & \citecode{s2t2\_chapt\_15.xml} \\ \hline
\citecode{064a} & Paysans corses en communauté, porchers-bergers des montagnes de Bastelica & \citecode{s2t2\_chapt\_16.xml} \\ \hline
\citecode{450a} & Tables alphabétique et analytique des matières contenues dans le présent tome, avec index explicatif des mots employés dans un sens propre à l'économie sociale & \citecode{s2t2\_chapt\_17-1.xml} \\ \hline
\citecode{450b} & Table des matières dans ce tome deuxième (deuxième série) & \citecode{s2t2\_chapt\_17-2.xml} \\ \hline
\end{longtable}
\end{center}

\subsection{Série 2, vol. 3 [1890-1892] (1892).}

\begin{center}
\begin{longtable}{ | c | p{9.5cm} | c | }
\hline
Id & Intitulé & Fichier \\ \hline
\citecode{451a} & Page de titre & \citecode{s2t3\_chapt\_1.xml} \\ \hline
\citecode{452a} & Page de titre & \citecode{s2t3\_chapt\_2.xml} \\ \hline
\citecode{453a} & Sommaire des monographies de familles publiées dans ce volume & \citecode{s2t3\_chapt\_3.xml} \\ \hline
\citecode{454a} & Avertissement sur ce troisième tome de la deuxième série des Ouvriers des deux mondes & \citecode{s2t3\_chapt\_4.xml} \\ \hline
\citecode{065a} & Métayers en communauté du Confolentais (Charente - France) & \citecode{s2t3\_chapt\_5.xml} \\ \hline
\citecode{066a} & Vignerons de Ribeauvillé (Alsace) & \citecode{s2t3\_chapt\_6.xml} \\ \hline
\citecode{066b} & Précis d'une monographie du pêcheur-côtier du Finmark (Laponie - Norvège) & \citecode{s2t3\_chapt\_7.xml} \\ \hline
\citecode{066c} & Précis d'une monographie d'un tisserand d'Hilversum (Hollande septentrionale - Pays-Bas) & \citecode{s2t3\_chapt\_14.xml} \\ \hline
\citecode{067a} & Tisserand de la fabrique collective de Gand (Flandre orientale - Belgique & \citecode{s2t3\_chapt\_28.xml} \\ \hline
\citecode{068a} & Paysan agriculteur de Torremaggiore (Province de Foggia - Italie) & \citecode{s2t3\_chapt\_29.xml} \\ \hline
\citecode{069a} & Tanneur de Nottingham (Angleterre) & \citecode{s2t3\_chapt\_30.xml} \\ \hline
\citecode{070a} & Charpentier indépendant de Paris (Seine - France) & \citecode{s2t3\_chapt\_31.xml} \\ \hline
\citecode{071a} & Conducteur-typographe de l'agglomération bruxelloise (Brabant - Belgique) & \citecode{s2t3\_chapt\_32.xml} \\ \hline
\citecode{072a} & Coutelier de la fabrique collective de Gembloux (Province de Namur - Belgique) & \citecode{s2t3\_chapt\_33.xml} \\ \hline
\citecode{455a} & Tables alphabétique et analytique des matières contenues dans le présent tome, avec index explicatif des mots employés dans un sens propre à l'économie sociale & \citecode{s2t3\_chapt\_34.xml} \\ \hline
\citecode{456a} & Table des matières dans ce tome troisième (deuxième série) & \citecode{s2t3\_chapt\_35.xml} \\ \hline
\end{longtable}
\end{center}

\subsection{Série 2, vol. 4 [1892-1895] (1895).}

\begin{center}
\begin{longtable}{ | c | p{9.5cm} | c | }
\hline
Id & Intitulé & Fichier \\ \hline
\citecode{457a} & Page de titre & \citecode{s2t4\_chapt\_1.xml} \\ \hline
\citecode{458a} & Page de titre & \citecode{s2t4\_chapt\_2.xml} \\ \hline
\citecode{459a} & Sommaire des monographies de familles publiées dans ce volume & \citecode{s2t4\_chapt\_3.xml} \\ \hline
\citecode{460a} & Avertissement sur ce quatrième volume de la deuxième série & \citecode{s2t4\_chapt\_4.xml} \\ \hline
\citecode{073a} & Ajusteur-surveillant de l'usine de Guise (Aisne - France) & \citecode{s2t4\_chapt\_5.xml} \\ \hline
\citecode{074a} & Ébéniste parisien de haut luxe (Seine - France) & \citecode{s2t4\_chapt\_6.xml} \\ \hline
\citecode{075a} & Métayer de l'Ouest du Texas (États-Unis d'Amérique) & \citecode{s2t4\_chapt\_7.xml} \\ \hline
\citecode{076a} & Ouvrière mouleuse en cartonnage d'une fabrique collective de jouets parisiens (Seine - France) & \citecode{s2t4\_chapt\_8.xml} \\ \hline
\citecode{077a} & Savetier de Bâle (Suisse) & \citecode{s2t4\_chapt\_9.xml} \\ \hline
\citecode{078a} & Ouvrier-employé de la fabrique coopérative de papiers d'Angoulême (Charente - France) & \citecode{s2t4\_chapt\_10.xml} \\ \hline
\citecode{079a} & Tisseur de San Leucio (Province de Caserte - Italie) & \citecode{s2t4\_chapt\_11.xml} \\ \hline
\citecode{080a} & Fermiers montagnards du Haut-Forez (Loire - France) & \citecode{s2t4\_chapt\_12.xml} \\ \hline
\citecode{081a} & Allumeur de réverbères de Nancy (Meurthe-et-Moselle - France) & \citecode{s2t4\_chapt\_13.xml} \\ \hline
\citecode{461a} & Tables alphabétique et analytique des matières contenues dans le présent tome, avec index explicatif des mots employés dans un sens propre à l'économie sociale & \citecode{s2t4\_chapt\_14.xml} \\ \hline
\citecode{462a} & Table des matières contenues dans ce tome quatrième (deuxième série) & \citecode{s2t4\_chapt\_15.xml} \\ \hline
\end{longtable}
\end{center}

\subsection{Série 2, vol. 5 [1895-1899] (1899).}

\begin{center}
\begin{longtable}{ | c | p{9.5cm} | c | }
\hline
Id & Intitulé & Fichier \\ \hline
\citecode{463a} & Page de titre & \citecode{s2t5\_chapt\_1.xml} \\ \hline
\citecode{464a} & Société d'économie sociale [nota : liste des publications] & \citecode{s2t5\_chapt\_2.xml} \\ \hline
\citecode{465a} & Sommaire des monographies de familles publiées dans ce volume & \citecode{s2t5\_chapt\_3.xml} \\ \hline
\citecode{466a} & Avertissement sur ce cinquième tome de la deuxième série & \citecode{s2t5\_chapt\_4.xml} \\ \hline
\citecode{082a} & Ouvrier garnisseur de canons de fusils de la fabrique collective d'armes à feu de Liège (Liège - Belgique) & \citecode{s2t5\_chapt\_5.xml} \\ \hline
\citecode{083a} & Fileur en peigné et régleur de métier de la Manufacture du Val-des-Bois (Marne - France) & \citecode{s2t5\_chapt\_6.xml} \\ \hline
\citecode{084a} & Cordonnier d'Iseghem (Flandre Occidentale - Belgique) & \citecode{s2t5\_chapt\_7.xml} \\ \hline
\citecode{085a} & Paysan métayer (Contadino mezzajuolo) de Roccasancasciano (Romagne Toscane - Italie) & \citecode{s2t5\_chapt\_8.xml} \\ \hline
\citecode{085b} & Précis d'une monographie d'un ouvrier agriculteur de la campagne de Ravenne (Romagne - Italie) & \citecode{s2t5\_chapt\_9.xml} \\ \hline
\citecode{086a} & Mineur des mines de houille du Pas-de-Calais (France) & \citecode{s2t5\_chapt\_10.xml} \\ \hline
\citecode{087a} & Agriculteur du Pas-de-Calais (France) & \citecode{s2t5\_chapt\_11.xml} \\ \hline
\citecode{088a} & Serrurier-forgeron du quartier de Picpus, à Paris (France) & \citecode{s2t5\_chapt\_12.xml} \\ \hline
\citecode{088b} & Précis d'une monographie du serrurier poseur de persiennes en fer de Paris & \citecode{s2t5\_chapt\_13.xml} \\ \hline
\citecode{089a} & Piqueur sociétaire de la Mine aux Mineurs de Monthieux (Loire - France) & \citecode{s2t5\_chapt\_14.xml} \\ \hline
\citecode{090a} & Petit fonctionnaire de Pnom-Penh (Cambodge) & \citecode{s2t5\_chapt\_15.xml} \\ \hline
\citecode{090b} & Précis d'une monographie d'un manœuvre-coolie de Pnom-Penh (Cambodge) & \citecode{s2t5\_chapt\_16.xml} \\ \hline
\citecode{091a} & Métayer de Corrèze (Bas Limousin - France) & \citecode{s2t5\_chapt\_17.xml} \\ \hline
\citecode{467a} & Tables alphabétique et analytique des matières contenues dans le présent tome, avec index explicatif des mots employés dans un sens propre à l'économie sociale & \citecode{s2t5\_chapt\_18-1.xml} \\ \hline
\citecode{467b} & Table des matières dans ce tome cinquième & \citecode{s2t5\_chapt\_18-2.xml} \\ \hline
\end{longtable}
\end{center}

\subsection{Série 3, vol. 1 [1900-1904] (1904).}

\begin{center}
\begin{longtable}{ | c | p{9.5cm} | c | }
\hline
Id & Intitulé & Fichier \\ \hline
\citecode{468a} & [Fichier sans texte] & \citecode{s3t1\_chapt\_1.xml} \\ \hline
\citecode{472a} & La société générale des papeteries du Limousins & \citecode{s3t1\_chapt\_2.xml} \\ \hline
\citecode{092a} & Fermier normand de Jersey & \citecode{s3t1\_chapt\_3.xml} \\ \hline
\citecode{092b} & Précis d'une monographie d'un pêcheur-côtier, maître de barques, de l'archipel Chusan (Chine) & \citecode{s3t1\_chapt\_4.xml} \\ \hline
\citecode{093a} & Aveugle accordeur de pianos de Levallois-Perret (Seine - France) & \citecode{s3t1\_chapt\_5.xml} \\ \hline
\citecode{094a} & Bouilleur de cru du Bas-Pays de Cognac (Charente - France) & \citecode{s3t1\_chapt\_6.xml} \\ \hline
\citecode{095a} & Mineur du bassin houiller du Couchant de Mons (Borinage - Belgique) & \citecode{s3t1\_chapt\_7.xml} \\ \hline
\citecode{096a} & Fellah de Karnak (Haute-Egypte) & \citecode{s3t1\_chapt\_8.xml} \\ \hline
\citecode{097a} & Tisserand d'usine de Gladbach (Prusse rhénane) & \citecode{s3t1\_chapt\_9.xml} \\ \hline
\citecode{098a} & Décoreuse de porcelaine de Limoges (Haute-Vienne - France) & \citecode{s3t1\_chapt\_10.xml} \\ \hline
\citecode{099a} & Cantonnier-poseur de voie du chemin de fer du Nord à Paris & \citecode{s3t1\_chapt\_11.xml} \\ \hline
\end{longtable}
\end{center}

\subsection{Série 3, vol. 2 [1904-1908] (1908).}

\begin{center}
\begin{longtable}{ | c | p{9.5cm} | c | }
\hline
Id & Intitulé & Fichier \\ \hline
\citecode{469a} & Page de titre & \citecode{s3t2\_chapt\_1.xml} \\ \hline
\citecode{100a} & Cordonnier de la fabrique collective de Binche (Province de Hainaut - Belgique) & \citecode{s3t2\_chapt\_2.xml} \\ \hline
\citecode{101a} & Compositeur typographe de Québec (Canada - Amérique du Nord) & \citecode{s3t2\_chapt\_3.xml} \\ \hline
\citecode{102a} & Ardoisier du bassin d'Herbeumont (Belgique) & \citecode{s3t2\_chapt\_4.xml} \\ \hline
\citecode{103a} & Commis à l'administration centrale des chemins de fer de l'État belge (Schaerbeek-Bruxelles - Belgique) & \citecode{s3t2\_chapt\_5.xml} \\ \hline
\citecode{104a} & Teinturier de ganterie et gantières de Saint-Junien (Haute-Vienne - France) & \citecode{s3t2\_chapt\_6.xml} \\ \hline
\citecode{105a} & Jardinier-plantier de Gasseras (Commune de Montauban, Tarn-et-Garonne - France) & \citecode{s3t2\_chapt\_7.xml} \\ \hline
\citecode{106a} & Corsetière du Raincy (banlieue de Paris - France) & \citecode{s3t2\_chapt\_8.xml} \\ \hline
\citecode{107a} & Étameur sur fer-blanc des usines de Commentry (Allier - France) & \citecode{s3t2\_chapt\_9.xml} \\ \hline
\citecode{473a} & Usine hydraulique d'éclairage et de transport de force & \citecode{s3t2\_chapt\_10.xml} \\ \hline
\end{longtable}
\end{center}

\subsection{Série 3, vol. 3 [1908-1913] (1913).}

\begin{center}
\begin{longtable}{ | c | p{9.5cm} | c | }
\hline
Id & Intitulé & Fichier \\ \hline
\citecode{470a} & Page de titre & \citecode{s3t3\_chapt\_1.xml} \\ \hline
\citecode{108a} & Paysan cultivateur du Ruvo di Puglia (Province de Bari - Italie, 1903) & \citecode{s3t3\_chapt\_2.xml} \\ \hline
\end{longtable}
\end{center}

\subsection{Série 3, vol. 3bis [1928-1930] (1930).}

\begin{center}
\begin{longtable}{ | c | p{9.5cm} | c | }
\hline
Id & Intitulé & Fichier \\ \hline
\citecode{471a} & Page de titre & \citecode{s3t3-bis\_chapt\_1.xml} \\ \hline
\end{longtable}
\end{center}

\section{Structure logique}
\label{structure}

\begin{enumerate}[A.]
    \item \textbf{\textit{Titre.}}
    \item \textbf{\textit{Observations préliminaires définissant la condition des divers membres de la famille.}}
    \begin{enumerate}[I.]
        \item \textbf{\textit{Définition du lieu, de l'organisation industrielle et de la famille.}}
        \begin{enumerate}[]
            \item \textit{§ 1. État du sol, de l'industrie et de la population.}
            \item \textit{§ 2. État civil de la famille.}
            \item \textit{§ 3. Religion et habitudes morales.}
            \item \textit{§ 4. Hygiène et services de santé.}
            \item \textit{§ 5. Rang de la famille.}
        \end{enumerate}
        \item \textbf{\textit{Moyens d'existence de la famille.}}
        \begin{enumerate}[]
            \item \textit{§ 6. Propriétés.}
            \item \textit{§ 7. Subventions.}
            \item \textit{§ 8. Travaux et industries.}
        \end{enumerate}
        \item \textbf{\textit{Mode d'existence de la famille.}}
        \begin{enumerate}[]
            \item \textit{§ 9. Aliments et repas.}
            \item \textit{§ 10. Habitation, mobilier et vêtements.}
            \item \textit{§ 11. Récréations.}
        \end{enumerate}
        \item \textbf{\textit{Histoire de la famille.}}
        \begin{enumerate}[]
            \item \textit{§ 12. Phases principales de l'existence.}
            \item \textit{§ 13. M\oe{}urs et institutions assurant le bien-être physique et moral de la famille.}
        \end{enumerate}
        \item (\textbf{\textit{Budget domestique annuel}}\footnote{Cette section ne possède un titre que dans huit monographies}).
        \begin{enumerate}[]
            \item \textit{§ 14. Budget des recettes de l'année.}
            \item \textit{§ 15. Budget des dépenses de l'année.}
            \item \textit{Comptes annexés aux budgets} \footnotesize{(n° 1 à 84) puis} \textit{§ 16. Comptes annexés aux budgets.}
        \end{enumerate}
    \end{enumerate}
    \item \textbf{\textit{Notes}} \footnotesize{(n° 1 à 84) puis} \textbf{\textit{Éléments divers de la constitution sociale}}.
    \begin{enumerate}[]
            \item (A) \textbf{\textit{(titre du paragraphe)}} \footnotesize{(n° 1 à 84) puis} § 17. \textbf{\textit{(titre du paragraphe)}}.
            \item (B) \textbf{\textit{(titre du paragraphe)}} \footnotesize{(n° 1 à 84) puis} § 18. \textbf{\textit{(titre du paragraphe)}}.
            \item \textbf{\textit{etc.}}
        \end{enumerate}
\end{enumerate}