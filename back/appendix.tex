\part*{Annexes}
\addcontentsline{toc}{part}{Annexes}
\appendix
\renewcommand{\thechapter}{A}
\chapter{A. \lodm}

\section{Liste des monographies et fichiers correspondant}
\label{mapping}

\textit{La date de publication du volume est indiquée entre parenthèses, tandis que les dates de parution des fascicules le sont entre crochets droits.}

\subsection{Série 1, vol. 1 (1857).}
\label{mappings1t1}

URL sur \ia{} : 

\url{https://archive.org/details/lesouvriersdesde01sociuoft}.

\begin{center}
\begin{longtable}{ | c | p{9.5cm} | c | }
\hline
Id & Intitulé & Fichier \\ \hline
\texttt{401a} & Page de titre & \texttt{s1t1\_chapt\_1.xml} \\ \hline
\texttt{402a} & Avertissement. Considérations générales sur la Société internationale des études pratiques d'économie sociale. Son but et ses moyens d'action. & \texttt{s1t1\_chapt\_2.xml} \\ \hline
\texttt{403a} & Institution. Société internationale des études pratiques d'économie sociale. Fondation et premiers travaux. & \texttt{s1t1\_chapt\_3.xml} \\ \hline
\texttt{404a} & Définitions par ordre alphabétiques des termes à employer dans les monographies, pour désigner les ouvriers, leurs moyens d'existence, et les rapports qui les unissent soit entre eux, soit avec les autres classes. & \texttt{s1t1\_chapt\_4.xml} \\ \hline
\texttt{405a} & Explications des signes de renvoi et des abréviations. & \texttt{s1t1\_chapt\_5.xml} \\ \hline
\texttt{001a} & Charpentier de Paris (Seine - France), de la Corporation des compagnons du Devoir & \texttt{s1t1\_chapt\_6.xml} \\ \hline
\texttt{002a} & Manœuvre-Agriculteur de la Champagne pouilleuse (Marne - France) & \texttt{s1t1\_chapt\_7.xml} \\ \hline
\texttt{003a} & Paysans en communauté du Lavedan (Hautes-Pyrénées - France) & \texttt{s1t1\_chapt\_8.xml} \\ \hline
\texttt{004a} & Paysan du Labourd (Basses-Pyrénées - France) & \texttt{s1t1\_chapt\_9.xml} \\ \hline
\texttt{005a} & Métayer de la banlieue de Florence (Grand-Duché de Toscane) & \texttt{s1t1\_chapt\_10.xml} \\ \hline
\texttt{006a} & Nourrisseur de vaches de la banlieue de Londres (Middlesex - Angleterre) & \texttt{s1t1\_chapt\_11.xml} \\ \hline
\texttt{007a} & Tisseur en châles de la fabrique urbaine collective de Paris (Seine - France) & \texttt{s1t1\_chapt\_12.xml} \\ \hline
\texttt{008a} & Manœuvre-agriculteur du comté de Nottingham (Angleterre) & \texttt{s1t1\_chapt\_13.xml} \\ \hline
\texttt{009a} & Pêcheur côtier, maître de barque de Saint-Sébastien (Guipuscoa - Espagne) & \texttt{s1t1\_chapt\_14.xml} \\ \hline
\texttt{406a} & Tables alphabétique et analytique des matières contenues dans ce tome premier. & \texttt{s1t1\_chapt\_15.xml} \\ \hline
\texttt{407a} & Liste des monographies destinées aux prochaines publications de la société d'économie sociale. & \texttt{s1t1\_chapt\_16.xml} \\ \hline
\texttt{408a} & Tables des matières contenues dans ce tome premier & \texttt{s1t1\_chapt\_17.xml} \\ \hline
\end{longtable}
\end{center}

\subsection{Série 1, vol. 2 (1858).}
\label{mappings1t2}

URL sur \ia{} : 

\url{https://archive.org/details/lesouvriersdesde02sociuoft}.

\begin{center}
\begin{longtable}{ | c | p{9.5cm} | c | }
\hline
Id & Intitulé & Fichier \\ \hline
\texttt{409a} & Page de titre & \texttt{s1t2\_chapt\_1.xml} \\ \hline
\texttt{410a} & Avertissement & \texttt{s1t2\_chapt\_2.xml} \\ \hline
\texttt{010a} & Ferblantier, couvreur et vitrier d'Aix-les-Bains (Savoie - États Sardes) & \texttt{s1t2\_chapt\_3.xml} \\ \hline
\texttt{011a} & Carrier des environs de Paris (Seine - France) & \texttt{s1t2\_chapt\_4.xml} \\ \hline
\texttt{012a} & Menuisier-charpentier (Nedjar) de Tanger (Province de Tanger - Maroc) & \texttt{s1t2\_chapt\_5.xml} \\ \hline
\texttt{013a} & Tailleur d'habits de Paris (Seine - France) & \texttt{s1t2\_chapt\_6.xml} \\ \hline
\texttt{014a} & Compositeur-typographe de Bruxelles (Brabant - Belgique) & \texttt{s1t2\_chapt\_7.xml} \\ \hline
\texttt{015a} & Décapeur d'outils en acier de la fabrique d'Hérimoncourt (Doubs - France) & \texttt{s1t2\_chapt\_8.xml} \\ \hline
\texttt{016a} & Monteur d'outils en acier de la fabrique d'Hérimoncourt (Doubs - France) & \texttt{s1t2\_chapt\_9.xml} \\ \hline
\texttt{017a} & Porteur d'eau de Paris (Seine - France) & \texttt{s1t2\_chapt\_10.xml} \\ \hline
\texttt{018a} & Paysans en communauté et en polygamie de Bousrah (Esky Cham), dans le pays de Haouran (Syrie - Empire Ottoman) & \texttt{s1t2\_chapt\_11.xml} \\ \hline
\texttt{019a} & Débardeur et piocheur de craie de la banlieue de Paris (Seine - France) & \texttt{s1t2\_chapt\_12.xml} \\ \hline
\texttt{411a} & Tables alphabétique et analytique des matières contenues dans ce tome second & \texttt{s1t2\_chapt\_13.xml} \\ \hline
\texttt{412a} & Errata & \texttt{s1t2\_chapt\_14.xml} \\ \hline
\texttt{413a} & Tables des matières contenues dans ce tome second & \texttt{s1t2\_chapt\_15.xml} \\ \hline
\end{longtable}
\end{center}

\subsection{Série 1, vol. 3 (1861).}
\label{mappings1t3}

URL sur \ia{} : 

\url{https://archive.org/details/lesouvriersdesde03sociuoft}.

\begin{center}
\begin{longtable}{ | c | p{9.5cm} | c | }
\hline
Id & Intitulé & Fichier \\ \hline
\texttt{414a} & Page de titre & \texttt{s1t3\_chapt\_1.xml} \\ \hline
\texttt{415a} & Avertissement & \texttt{s1t3\_chapt\_2.xml} \\ \hline
\texttt{416a} & Rapport. Société d'économie sociale. Travaux de 1859-1860 & \texttt{s1t3\_chapt\_3.xml} \\ \hline
\texttt{417a} & Liste générale des membres de la Société internationale des études pratiques d'économie sociale & \texttt{s1t3\_chapt\_4.xml} \\ \hline
\texttt{020a} & Brodeuses des Vosges (Vosges - France) & \texttt{s1t3\_chapt\_5.xml} \\ \hline
\texttt{021a} & Paysan et savonnier de la Basse-Provence (Bouches-du-Rhône - France) & \texttt{s1t3\_chapt\_6.xml} \\ \hline
\texttt{022a} & Mineur des Placers du comté de Mariposa (Californie - États-Unis) & \texttt{s1t3\_chapt\_7.xml} \\ \hline
\texttt{023a} & Manœuvre-vigneron de l'Aunis (Charente-inférieure - France) & \texttt{s1t3\_chapt\_8.xml} \\ \hline
\texttt{024a} & Lingère de Lille (Nord - France) & \texttt{s1t3\_chapt\_9.xml} \\ \hline
\texttt{025a} & Parfumeur de Tunis (Régence de Tunis - Afrique) du bazar appelé : El Attharin-el-kebar (les grands parfumeurs) & \texttt{s1t3\_chapt\_10.xml} \\ \hline
\texttt{026a} & Instituteur primaire d'une commune rurale de la Normandie (Eure - France) & \texttt{s1t3\_chapt\_11.xml} \\ \hline
\texttt{027a} & Manœuvre à famille nombreuse de Paris (Seine - France) & \texttt{s1t3\_chapt\_12.xml} \\ \hline
\texttt{028a} & Fondeur de plomb des Alpes Apuanes (Toscane - Italie) & \texttt{s1t3\_chapt\_13.xml} \\ \hline
\texttt{418a} & Tables alphabétique et analytique des matières contenues dans ce tome troisième & \texttt{s1t3\_chapt\_14.xml} \\ \hline
\texttt{419a} & Errata de ce tome troisième & \texttt{s1t3\_chapt\_15.xml} \\ \hline
\texttt{420a} & Tables des matières contenues dans ce tome troisième & \texttt{s1t3\_chapt\_16.xml} \\ \hline
\end{longtable}
\end{center}

\subsection{Série 1, vol. 4 (1862).}
\label{mappings1t4}

URL sur \ia{} : 

\url{https://archive.org/details/lesouvriersdesde04sociuoft}.

\begin{center}
\begin{longtable}{ | c | p{9.5cm} | c | }
\hline
Id & Intitulé & Fichier \\ \hline
\texttt{421a} & Page de titre & \texttt{s1t4\_chapt\_1.xml} \\ \hline
\texttt{422a} & Explications des signes de renvoi et des abréviations. & \texttt{s1t4\_chapt\_2.xml} \\ \hline
\texttt{423a} & Avertissement & \texttt{s1t4\_chapt\_3.xml} \\ \hline
\texttt{424a} & Rapport. Société d'économie sociale. Travaux de 1860-1861 & \texttt{s1t4\_chapt\_4.xml} \\ \hline
\texttt{425a} & Instruction. Méthode d'observation des monographies de famille propre à l'ouvrage intitulé Les ouvriers européens & \texttt{s1t4\_chapt\_5.xml} \\ \hline
\texttt{426a} & Histoire de la famille. Prix fondé par M. le baron de Damas. Par la société d'économie sociale & \texttt{s1t4\_chapt\_6.xml} \\ \hline
\texttt{029a} & Paysan d'un village à banlieue morcelée du Laonnais (Aisne - France) & \texttt{s1t4\_chapt\_7.xml} \\ \hline
\texttt{030a} & Paysans en communauté du Ning-Po-Fou (province de Tché-Kian - Chine) & \texttt{s1t4\_chapt\_8.xml} \\ \hline
\texttt{031a} & Mulâtre affranchi de l'Ile de la Réunion (Océan Indien) & \texttt{s1t4\_chapt\_9.xml} \\ \hline
\texttt{032a} & Manœuvre-vigneron de la Basse-Bourgogne (Yonne - France) & \texttt{s1t4\_chapt\_10.xml} \\ \hline
\texttt{033a} & Compositeur-typographe de Paris (Seine - France) & \texttt{s1t4\_chapt\_11.xml} \\ \hline
\texttt{034a} & Auvergnat brocanteur en boutique à Paris (Seine - France) & \texttt{s1t4\_chapt\_12.xml} \\ \hline
\texttt{035a} & Mineur de la Maremme de Toscane (Toscane - Italie) & \texttt{s1t4\_chapt\_13.xml} \\ \hline
\texttt{036a} & Tisserand des Vosges (Haut-Rhin - France) & \texttt{s1t4\_chapt\_14.xml} \\ \hline
\texttt{037a} & Pêcheur côtier, maître de barques, de Marken (Hollande septentrionale - Pays-Bas) & \texttt{s1t4\_chapt\_15.xml} \\ \hline
\texttt{427a} & Société internationale des études pratiques d'économie sociale. Officiers composants les conseils d'administration et de surveillance pour la session 1863-1864. & \texttt{s1t4\_chapt\_16.xml} \\ \hline
\texttt{428a} & Liste générale des membres de la Société internationale des études pratiques d'économie sociale au 1er août 1863 & \texttt{s1t4\_chapt\_17.xml} \\ \hline
\texttt{429a} & Tables alphabétique et analytique des matières contenues dans ce tome quatrième & \texttt{s1t4\_chapt\_18.xml} \\ \hline
\texttt{430a} & Errata de ce tome quatrième & \texttt{s1t4\_chapt\_19.xml} \\ \hline
\texttt{431a} & Tables des matières contenues dans ce tome quatrième & \texttt{s1t4\_chapt\_20.xml} \\ \hline
\end{longtable}
\end{center}

\subsection{Série 1, vol. 5 [1875, 1883, 1884] (1885).}
\label{mappings1t5}

URL sur \ia{} : 

\url{https://archive.org/details/lesouvriersdesde05sociuoft}

\begin{center}
\begin{longtable}{ | c | p{9.5cm} | c | }
\hline
Id & Intitulé & Fichier \\ \hline
\texttt{432a} & Page de titre & \texttt{s1t5\_chapt\_1.xml} \\ \hline
\texttt{433a} & Oeuvres de F. Le Play & \texttt{s1t5\_chapt\_2.xml} \\ \hline
\texttt{434a} & Sommaire. Monographies de familles publiées dans ce volume & \texttt{s1t5\_chapt\_3.xml} \\ \hline
\texttt{435a} & Avertissement & \texttt{s1t5\_chapt\_4.xml} \\ \hline
\texttt{436a} & Explications des signes de renvoi et des abréviations employés dans le cours de cet ouvrage & \texttt{s1t5\_chapt\_5.xml} \\ \hline
\texttt{038a} & Fermiers à communauté taisible du Nivernais (Saône-et-Loire - France) & \texttt{s1t5\_chapt\_6.xml} \\ \hline
\texttt{039a} & Paysan de Saint-Irénée (Bas-Canada - Amérique du Nord) & \texttt{s1t5\_chapt\_7.xml} \\ \hline
\texttt{040a} & L'Ouvrier éventailliste de Sainte-Geneviève (Oise - France) & \texttt{s1t5\_chapt\_8.xml} \\ \hline
\texttt{041a} & Ouvrier cordonnier de Malakoff (Seine - France) & \texttt{s1t5\_chapt\_9.xml} \\ \hline
\texttt{041b} & Précis d'une monographie ayant pour objet un chiffonnier instable et, par alternance, mégissier fumiste et brossier de Paris (France - Seine) & \textit{In supra.} \\ \hline
\texttt{042a} & Serrurier-forgeron de Paris (Seine - France) & \texttt{s1t5\_chapt\_10.xml} \\ \hline
\texttt{042b} & Précis d'une monographie ayant pour objet le monteur en bronze de Paris & \textit{In supra.} \\ \hline
\texttt{043a} & Brigadier de la Garde républicaine de Paris (Seine - France) & \texttt{s1t5\_chapt\_11.xml} \\ \hline
\texttt{044a} & Paysan-résinier de Lévignacq (Landes - France) & \texttt{s1t5\_chapt\_12.xml} \\ \hline
\texttt{045a} & Bûcheron usager de l'ancien Comté de Dabo (Lorraine allemande) & \texttt{s1t5\_chapt\_13.xml} \\ \hline
\texttt{046a} & Paysans en communauté et colporteurs émigrants de Tabou-Douchd-El-Baar (Grande Kabylie - Province d'Alger) & \texttt{s1t5\_chapt\_14.xml} \\ \hline
\texttt{437a} & Société d'économie sociale. Conseil d'administration pour l'année 1885 & \texttt{s1t5\_chapt\_15.xml} \\ \hline
\texttt{438a} &  Liste générale des membres de la Société d'économie sociale au 15 mars 1885 & \texttt{s1t5\_chapt\_16.xml} \\ \hline
\texttt{439a} & Tables alphabétique et analytique des matières contenues dans ce tome cinquième & \texttt{s1t5\_chapt\_17.xml} \\ \hline
\texttt{440a} & Tables des matières contenues dans ce tome cinquième & \texttt{s1t5\_chapt\_18.xml} \\ \hline
\end{longtable}
\end{center}

\subsection{Série 2, vol. 1 [1885-1887] (1887).}
\label{mappings2t1}

URL sur \ia{} : 


\url{https://archive.org/details/s2lesouvriersdes01sociuoft}

\begin{center}
\begin{longtable}{ | c | p{9.5cm} | c | }
\hline
Id & Intitulé & Fichier \\ \hline
\texttt{441a} & Page de titre & \texttt{s2t1\_chapt\_1.xml} \\ \hline
\texttt{442a} & Sommaire des monographies de familles publiées dans ce volume & \texttt{s2t1\_chapt\_2.xml} \\ \hline
\texttt{443a} & Avertissement sur ce premier volume, deuxième série des Ouvriers des deux mondes & \texttt{s2t1\_chapt\_3.xml} \\ \hline
\texttt{047a} & Paysan-paludier du Bourg de Batz (Loire-Inférieure - France) & \texttt{s2t1\_chapt\_4.xml} \\ \hline
\texttt{048a} & Bordiers émancipés en communauté rurale de la Grande-Russie & \texttt{s2t1\_chapt\_5.xml} \\ \hline
\texttt{048b} & Précis d'une monographie de l'armurier des manufactures impériales de Toula (Grande-Russie) & \texttt{s2t1\_chapt\_6.xml} \\ \hline
\texttt{049a} & Charron des forges et fonderies de Montataire (Oise - France) & \texttt{s2t1\_chapt\_16.xml} \\ \hline
\texttt{050a} & Faienciers de Nevers (Nièvre - France) & \texttt{s2t1\_chapt\_17.xml} \\ \hline
\texttt{051a} & Cultivateur-maraicher de Deuil (Seine-et-Oise - France) & \texttt{s2t1\_chapt\_18.xml} \\ \hline
\texttt{052a} & Pêcheur-côtier, maître de barque, de Martigues (Bouches-du-Rhône - France) & \texttt{s2t1\_chapt\_19.xml} \\ \hline
\texttt{053a} & Métayer à famille-souche du pays d'Horte (Landes - France) & \texttt{s2t1\_chapt\_20.xml} \\ \hline
\texttt{054a} & Arabes pasteurs nomades de la tribu des Larbas (Région saharienne de l'Algérie) & \texttt{s2t1\_chapt\_21.xml} \\ \hline
\texttt{055a} & Gantier de Grenoble (Isère - France) & \texttt{s2t1\_chapt\_22.xml} \\ \hline
\texttt{444a} & Tables alphabétique et analytique des matières contenues dans le présent volume & \texttt{s2t1\_chapt\_24.xml} \\ \hline
\texttt{445a} & Table des matières dans ce tome premier (deuxième série) & \texttt{s2t1\_chapt\_25.xml} \\ \hline
\end{longtable}
\end{center}

\subsection{Série 2, vol. 2 [1887-1889] (1890).}
\label{mappings2t2}

URL sur \ia{} : 

\url{https://archive.org/details/s2lesouvriersdes02sociuoft}.

\begin{center}
\begin{longtable}{ | c | p{9.5cm} | c | }
\hline
Id & Intitulé & Fichier \\ \hline
\texttt{446a} & Page de titre & \texttt{s2t2\_chapt\_1.xml} \\ \hline
\texttt{447a} & Page de titre & \texttt{s2t2\_chapt\_2.xml} \\ \hline
\texttt{448a} & Sommaire des monographies de familles publiées dans ce volume & \texttt{s2t2\_chapt\_3.xml} \\ \hline
\texttt{449a} & Avertissement sur ce deuxième tome de la deuxième série des Ouvriers des deux mondes & \texttt{s2t2\_chapt\_4.xml} \\ \hline
\texttt{056a} & Tourneur-mécanicien des usines de la Société Cockerill, de Seraing (Belgique) & \texttt{s2t2\_chapt\_5.xml} \\ \hline
\texttt{057a} & Bordier (Fellah) berbère de la Grande-Kabylie (Province d'Alger) & \texttt{s2t2\_chapt\_6.xml} \\ \hline
\texttt{057b} & Précis d'une monographie du paysan colon du Sahel (Algérie) & \texttt{s2t2\_chapt\_7.xml} \\ \hline
\texttt{058a} & Pêcheur côtier d'Heyst (Flandre occidentale - Belgique) & \texttt{s2t2\_chapt\_8.xml} \\ \hline
\texttt{058b} & Précis d'une monographie du pêcheur côtier, maître de barque, d'Étretat (Seine-Inférieure - France) & \texttt{s2t2\_chapt\_9.xml} \\ \hline
\texttt{059a} & Paysan-métayer de la Basse Provence (Bouches-du-Rhône - France) & \texttt{s2t2\_chapt\_10.xml} \\ \hline
\texttt{059b} & Précis d'une monographie du paysan et maçon émigrant de la Marche (Creuse - France) & \texttt{s2t2\_chapt\_11.xml} \\ \hline
\texttt{060a} & Mineur silésien du bassin houiller de la Ruhr (Prusse rhénane - Allemagne) & \texttt{s2t2\_chapt\_12.xml} \\ \hline
\texttt{061a} & Mineur des soufrières de Lercara (Province de Palerme - Sicile & \texttt{s2t2\_chapt\_13.xml} \\ \hline
\texttt{062a} & Tailleur de Silex et vigneron de l'Orléanais (Loir-et-Cher - France) & \texttt{s2t2\_chapt\_14.xml} \\ \hline
\texttt{063a} & Vigneron précariste et métayer de Valmontone (Province de Rome - Italie) & \texttt{s2t2\_chapt\_15.xml} \\ \hline
\texttt{064a} & Paysans corses en communauté, porchers-bergers des montagnes de Bastelica & \texttt{s2t2\_chapt\_16.xml} \\ \hline
\texttt{450a} & Tables alphabétique et analytique des matières contenues dans le présent tome, avec index explicatif des mots employés dans un sens propre à l'économie sociale & \texttt{s2t2\_chapt\_17-1.xml} \\ \hline
\texttt{450b} & Table des matières dans ce tome deuxième (deuxième série) & \texttt{s2t2\_chapt\_17-2.xml} \\ \hline
\end{longtable}
\end{center}

\subsection{Série 2, vol. 3 [1890-1892] (1892).}
\label{mappings2t3}

URL sur \ia{} : 

\url{https://archive.org/details/s2lesouvriersdes03sociuoft}.

\begin{center}
\begin{longtable}{ | c | p{9.5cm} | c | }
\hline
Id & Intitulé & Fichier \\ \hline
\texttt{451a} & Page de titre & \texttt{s2t3\_chapt\_1.xml} \\ \hline
\texttt{452a} & Page de titre & \texttt{s2t3\_chapt\_2.xml} \\ \hline
\texttt{453a} & Sommaire des monographies de familles publiées dans ce volume & \texttt{s2t3\_chapt\_3.xml} \\ \hline
\texttt{454a} & Avertissement sur ce troisième tome de la deuxième série des Ouvriers des deux mondes & \texttt{s2t3\_chapt\_4.xml} \\ \hline
\texttt{065a} & Métayers en communauté du Confolentais (Charente - France) & \texttt{s2t3\_chapt\_5.xml} \\ \hline
\texttt{066a} & Vignerons de Ribeauvillé (Alsace) & \texttt{s2t3\_chapt\_6.xml} \\ \hline
\texttt{066b} & Précis d'une monographie du pêcheur-côtier du Finmark (Laponie - Norvège) & \texttt{s2t3\_chapt\_7.xml} \\ \hline
\texttt{066c} & Précis d'une monographie d'un tisserand d'Hilversum (Hollande septentrionale - Pays-Bas) & \texttt{s2t3\_chapt\_14.xml} \\ \hline
\texttt{067a} & Tisserand de la fabrique collective de Gand (Flandre orientale - Belgique & \texttt{s2t3\_chapt\_28.xml} \\ \hline
\texttt{068a} & Paysan agriculteur de Torremaggiore (Province de Foggia - Italie) & \texttt{s2t3\_chapt\_29.xml} \\ \hline
\texttt{069a} & Tanneur de Nottingham (Angleterre) & \texttt{s2t3\_chapt\_30.xml} \\ \hline
\texttt{070a} & Charpentier indépendant de Paris (Seine - France) & \texttt{s2t3\_chapt\_31.xml} \\ \hline
\texttt{071a} & Conducteur-typographe de l'agglomération bruxelloise (Brabant - Belgique) & \texttt{s2t3\_chapt\_32.xml} \\ \hline
\texttt{072a} & Coutelier de la fabrique collective de Gembloux (Province de Namur - Belgique) & \texttt{s2t3\_chapt\_33.xml} \\ \hline
\texttt{455a} & Tables alphabétique et analytique des matières contenues dans le présent tome, avec index explicatif des mots employés dans un sens propre à l'économie sociale & \texttt{s2t3\_chapt\_34.xml} \\ \hline
\texttt{456a} & Table des matières dans ce tome troisième (deuxième série) & \texttt{s2t3\_chapt\_35.xml} \\ \hline
\end{longtable}
\end{center}

\subsection{Série 2, vol. 4 [1892-1895] (1895).}
\label{mappings2t4}

URL sur \ia{} : 

\url{https://archive.org/details/s2lesouvriersdes04sociuoft}.

\begin{center}
\begin{longtable}{ | c | p{9.5cm} | c | }
\hline
Id & Intitulé & Fichier \\ \hline
\texttt{457a} & Page de titre & \texttt{s2t4\_chapt\_1.xml} \\ \hline
\texttt{458a} & Page de titre & \texttt{s2t4\_chapt\_2.xml} \\ \hline
\texttt{459a} & Sommaire des monographies de familles publiées dans ce volume & \texttt{s2t4\_chapt\_3.xml} \\ \hline
\texttt{460a} & Avertissement sur ce quatrième volume de la deuxième série & \texttt{s2t4\_chapt\_4.xml} \\ \hline
\texttt{073a} & Ajusteur-surveillant de l'usine de Guise (Aisne - France) & \texttt{s2t4\_chapt\_5.xml} \\ \hline
\texttt{074a} & Ébéniste parisien de haut luxe (Seine - France) & \texttt{s2t4\_chapt\_6.xml} \\ \hline
\texttt{075a} & Métayer de l'Ouest du Texas (États-Unis d'Amérique) & \texttt{s2t4\_chapt\_7.xml} \\ \hline
\texttt{076a} & Ouvrière mouleuse en cartonnage d'une fabrique collective de jouets parisiens (Seine - France) & \texttt{s2t4\_chapt\_8.xml} \\ \hline
\texttt{077a} & Savetier de Bâle (Suisse) & \texttt{s2t4\_chapt\_9.xml} \\ \hline
\texttt{078a} & Ouvrier-employé de la fabrique coopérative de papiers d'Angoulême (Charente - France) & \texttt{s2t4\_chapt\_10.xml} \\ \hline
\texttt{079a} & Tisseur de San Leucio (Province de Caserte - Italie) & \texttt{s2t4\_chapt\_11.xml} \\ \hline
\texttt{080a} & Fermiers montagnards du Haut-Forez (Loire - France) & \texttt{s2t4\_chapt\_12.xml} \\ \hline
\texttt{081a} & Allumeur de réverbères de Nancy (Meurthe-et-Moselle - France) & \texttt{s2t4\_chapt\_13.xml} \\ \hline
\texttt{461a} & Tables alphabétique et analytique des matières contenues dans le présent tome, avec index explicatif des mots employés dans un sens propre à l'économie sociale & \texttt{s2t4\_chapt\_14.xml} \\ \hline
\texttt{462a} & Table des matières contenues dans ce tome quatrième (deuxième série) & \texttt{s2t4\_chapt\_15.xml} \\ \hline
\end{longtable}
\end{center}

\subsection{Série 2, vol. 5 [1895-1899] (1899).}
\label{mappings2t5}

URL sur \ia{} : 

\url{https://archive.org/details/2serlesouvriersde05sociuoft}.

\begin{center}
\begin{longtable}{ | c | p{9.5cm} | c | }
\hline
Id & Intitulé & Fichier \\ \hline
\texttt{463a} & Page de titre & \texttt{s2t5\_chapt\_1.xml} \\ \hline
\texttt{464a} & Société d'économie sociale [nota : liste des publications] & \texttt{s2t5\_chapt\_2.xml} \\ \hline
\texttt{465a} & Sommaire des monographies de familles publiées dans ce volume & \texttt{s2t5\_chapt\_3.xml} \\ \hline
\texttt{466a} & Avertissement sur ce cinquième tome de la deuxième série & \texttt{s2t5\_chapt\_4.xml} \\ \hline
\texttt{082a} & Ouvrier garnisseur de canons de fusils de la fabrique collective d'armes à feu de Liège (Liège - Belgique) & \texttt{s2t5\_chapt\_5.xml} \\ \hline
\texttt{083a} & Fileur en peigné et régleur de métier de la Manufacture du Val-des-Bois (Marne - France) & \texttt{s2t5\_chapt\_6.xml} \\ \hline
\texttt{084a} & Cordonnier d'Iseghem (Flandre Occidentale - Belgique) & \texttt{s2t5\_chapt\_7.xml} \\ \hline
\texttt{085a} & Paysan métayer (Contadino mezzajuolo) de Roccasancasciano (Romagne Toscane - Italie) & \texttt{s2t5\_chapt\_8.xml} \\ \hline
\texttt{085b} & Précis d'une monographie d'un ouvrier agriculteur de la campagne de Ravenne (Romagne - Italie) & \texttt{s2t5\_chapt\_9.xml} \\ \hline
\texttt{086a} & Mineur des mines de houille du Pas-de-Calais (France) & \texttt{s2t5\_chapt\_10.xml} \\ \hline
\texttt{087a} & Agriculteur du Pas-de-Calais (France) & \texttt{s2t5\_chapt\_11.xml} \\ \hline
\texttt{088a} & Serrurier-forgeron du quartier de Picpus, à Paris (France) & \texttt{s2t5\_chapt\_12.xml} \\ \hline
\texttt{088b} & Précis d'une monographie du serrurier poseur de persiennes en fer de Paris & \texttt{s2t5\_chapt\_13.xml} \\ \hline
\texttt{089a} & Piqueur sociétaire de la Mine aux Mineurs de Monthieux (Loire - France) & \texttt{s2t5\_chapt\_14.xml} \\ \hline
\texttt{090a} & Petit fonctionnaire de Pnom-Penh (Cambodge) & \texttt{s2t5\_chapt\_15.xml} \\ \hline
\texttt{090b} & Précis d'une monographie d'un manœuvre-coolie de Pnom-Penh (Cambodge) & \texttt{s2t5\_chapt\_16.xml} \\ \hline
\texttt{091a} & Métayer de Corrèze (Bas Limousin - France) & \texttt{s2t5\_chapt\_17.xml} \\ \hline
\texttt{467a} & Tables alphabétique et analytique des matières contenues dans le présent tome, avec index explicatif des mots employés dans un sens propre à l'économie sociale & \texttt{s2t5\_chapt\_18-1.xml} \\ \hline
\texttt{467b} & Table des matières dans ce tome cinquième & \texttt{s2t5\_chapt\_18-2.xml} \\ \hline
\end{longtable}
\end{center}

\subsection{Série 3, vol. 1 [1900-1904] (1904).}
\label{mappings3t1}

URL sur \ia{} : 

\url{https://archive.org/details/lesouvriersdesde0108sociuoft/}.

\begin{center}
\begin{longtable}{ | c | p{9.5cm} | c | }
\hline
Id & Intitulé & Fichier \\ \hline
\texttt{468a} & [Fichier sans texte] & \texttt{s3t1\_chapt\_1.xml} \\ \hline
\texttt{472a} & La société générale des papeteries du Limousins & \texttt{s3t1\_chapt\_2.xml} \\ \hline
\texttt{092a} & Fermier normand de Jersey & \texttt{s3t1\_chapt\_3.xml} \\ \hline
\texttt{092b} & Précis d'une monographie d'un pêcheur-côtier, maître de barques, de l'archipel Chusan (Chine) & \texttt{s3t1\_chapt\_4.xml} \\ \hline
\texttt{093a} & Aveugle accordeur de pianos de Levallois-Perret (Seine - France) & \texttt{s3t1\_chapt\_5.xml} \\ \hline
\texttt{094a} & Bouilleur de cru du Bas-Pays de Cognac (Charente - France) & \texttt{s3t1\_chapt\_6.xml} \\ \hline
\texttt{095a} & Mineur du bassin houiller du Couchant de Mons (Borinage - Belgique) & \texttt{s3t1\_chapt\_7.xml} \\ \hline
\texttt{096a} & Fellah de Karnak (Haute-Egypte) & \texttt{s3t1\_chapt\_8.xml} \\ \hline
\texttt{097a} & Tisserand d'usine de Gladbach (Prusse rhénane) & \texttt{s3t1\_chapt\_9.xml} \\ \hline
\texttt{098a} & Décoreuse de porcelaine de Limoges (Haute-Vienne - France) & \texttt{s3t1\_chapt\_10.xml} \\ \hline
\texttt{099a} & Cantonnier-poseur de voie du chemin de fer du Nord à Paris & \texttt{s3t1\_chapt\_11.xml} \\ \hline
\end{longtable}
\end{center}

\subsection{Série 3, vol. 2 [1904-1908] (1908).}
\label{mappings3t2}

URL sur \ia{} : 

\url{https://archive.org/details/lesouvriersdesde916sociuoft/}.

\begin{center}
\begin{longtable}{ | c | p{9.5cm} | c | }
\hline
Id & Intitulé & Fichier \\ \hline
\texttt{469a} & Page de titre & \texttt{s3t2\_chapt\_1.xml} \\ \hline
\texttt{100a} & Cordonnier de la fabrique collective de Binche (Province de Hainaut - Belgique) & \texttt{s3t2\_chapt\_2.xml} \\ \hline
\texttt{101a} & Compositeur typographe de Québec (Canada - Amérique du Nord) & \texttt{s3t2\_chapt\_3.xml} \\ \hline
\texttt{102a} & Ardoisier du bassin d'Herbeumont (Belgique) & \texttt{s3t2\_chapt\_4.xml} \\ \hline
\texttt{103a} & Commis à l'administration centrale des chemins de fer de l'État belge (Schaerbeek-Bruxelles - Belgique) & \texttt{s3t2\_chapt\_5.xml} \\ \hline
\texttt{104a} & Teinturier de ganterie et gantières de Saint-Junien (Haute-Vienne - France) & \texttt{s3t2\_chapt\_6.xml} \\ \hline
\texttt{105a} & Jardinier-plantier de Gasseras (Commune de Montauban, Tarn-et-Garonne - France) & \texttt{s3t2\_chapt\_7.xml} \\ \hline
\texttt{106a} & Corsetière du Raincy (banlieue de Paris - France) & \texttt{s3t2\_chapt\_8.xml} \\ \hline
\texttt{107a} & Étameur sur fer-blanc des usines de Commentry (Allier - France) & \texttt{s3t2\_chapt\_9.xml} \\ \hline
\texttt{473a} & Usine hydraulique d'éclairage et de transport de force & \texttt{s3t2\_chapt\_10.xml} \\ \hline
\end{longtable}
\end{center}

\subsection{Série 3, vol. 3 [1908-1913] (1913).}
\label{mappings3t3}

URL sur \ia{} : 

\url{https://archive.org/details/lesouvriersdesde17sociuoft}.

\begin{center}
\begin{longtable}{ | c | p{9.5cm} | c | }
\hline
Id & Intitulé & Fichier \\ \hline
\texttt{470a} & Page de titre & \texttt{s3t3\_chapt\_1.xml} \\ \hline
\texttt{108a} & Paysan cultivateur du Ruvo di Puglia (Province de Bari - Italie, 1903) & \texttt{s3t3\_chapt\_2.xml} \\ \hline
\end{longtable}
\end{center}

\subsection{Série 3, vol. 3bis [1928-1930] (1930).}
\label{mappings3t3bis}

\begin{center}
\begin{longtable}{ | c | p{9.5cm} | c | }
\hline
Id & Intitulé & Fichier \\ \hline
\texttt{471a} & Page de titre & \texttt{s3t3-bis\_chapt\_1.xml} \\ \hline
\end{longtable}
\end{center}

\clearpage

\section{Structure logique}
\label{structure}

\begin{enumerate}[A.]
    \item \textbf{\textit{Titre.}}
    \item \textbf{\textit{Observations préliminaires définissant la condition des divers membres de la famille.}}
    \begin{enumerate}[I.]
        \item \textbf{\textit{Définition du lieu, de l'organisation industrielle et de la famille.}}
        \begin{enumerate}[]
            \item \textit{§ 1. État du sol, de l'industrie et de la population.}
            \item \textit{§ 2. État civil de la famille.}
            \item \textit{§ 3. Religion et habitudes morales.}
            \item \textit{§ 4. Hygiène et services de santé.}
            \item \textit{§ 5. Rang de la famille.}
        \end{enumerate}
        \item \textbf{\textit{Moyens d'existence de la famille.}}
        \begin{enumerate}[]
            \item \textit{§ 6. Propriétés.}
            \item \textit{§ 7. Subventions.}
            \item \textit{§ 8. Travaux et industries.}
        \end{enumerate}
        \item \textbf{\textit{Mode d'existence de la famille.}}
        \begin{enumerate}[]
            \item \textit{§ 9. Aliments et repas.}
            \item \textit{§ 10. Habitation, mobilier et vêtements.}
            \item \textit{§ 11. Récréations.}
        \end{enumerate}
        \item \textbf{\textit{Histoire de la famille.}}
        \begin{enumerate}[]
            \item \textit{§ 12. Phases principales de l'existence.}
            \item \textit{§ 13. M\oe{}urs et institutions assurant le bien-être physique et moral de la famille.}
        \end{enumerate}
        \item (\textbf{\textit{Budget domestique annuel}}\footnote{Cette section ne possède un titre que dans huit monographies}).
        \begin{enumerate}[]
            \item \textit{§ 14. Budget des recettes de l'année.}
            \item \textit{§ 15. Budget des dépenses de l'année.}
            \item \textit{Comptes annexés aux budgets} \footnotesize{(n° 1 à 84) puis} \textit{§ 16. Comptes annexés aux budgets.}
        \end{enumerate}
    \end{enumerate}
    \item \textbf{\textit{Notes}} \footnotesize{(n° 1 à 84) puis} \textbf{\textit{Éléments divers de la constitution sociale}}.
    \begin{enumerate}[]
            \item (A) \textbf{\textit{(titre du paragraphe)}} \footnotesize{(n° 1 à 84) puis} § 17. \textbf{\textit{(titre du paragraphe)}}.
            \item (B) \textbf{\textit{(titre du paragraphe)}} \footnotesize{(n° 1 à 84) puis} § 18. \textbf{\textit{(titre du paragraphe)}}.
            \item \textbf{\textit{etc.}}
        \end{enumerate}
\end{enumerate}

\clearpage

\section{Numérisations de \gb}\label{numgb}

Les \odm{} sont disponibles en version numérisée sur \gb. Nous listons ci-dessous les volumes en accès libre en fonction de leur lieu de conservation (les URL du domaine \url{hdl.handle.net} renvoient vers la \textit{HathiTrust Digital Library}). Les volumes sont numérotés ainsi : \textit{série (numéro du volume)}.

\subsection{Bibliothèque de l'université de Californie}
\begin{center}
\begin{tabular}{ | c | p{13cm} | }
\hline
Volume & URL \\ \hline
1 (1) & \url{https://books.google.fr/books?id=eN0WAAAAYAAJ} \\ \hline
1 (1) & \url{https://hdl.handle.net/2027/uc1.b4577103} \\ \hline
1 (2) & \url{https://books.google.fr/books?id=4GJwAAAAIAAJ} \\ \hline
1 (3) & \url{https://hdl.handle.net/2027/uc1.b4577105} \\ \hline
1 (4) & \url{https://books.google.fr/books?id=Y2hwAAAAIAAJ} \\ \hline
1 (4) & \url{https://hdl.handle.net/2027/uc1.b4577106} \\ \hline
\end{tabular}
\end{center}

\subsection{Bibliothèque nationale centrale de Florence}
\begin{center}
\begin{tabular}{ | c | p{13cm} | }
\hline
Volume & URL \\ \hline
1 (1) & \url{https://books.google.fr/books?id=rqZexB0-3V8C} \\ \hline
\end{tabular}
\end{center}

\subsection{Bibliothèque de l'université Harvard}
\begin{center}
\begin{tabular}{ | c | p{13cm} | }
\hline
Volume & URL \\ \hline
1 (1) & \url{https://hdl.handle.net/2027/hvd.32044079431714} \\ \hline
2 (2) & \url{https://hdl.handle.net/2027/hvd.32044018834879} \\ \hline
2 (5) & \url{https://hdl.handle.net/2027/hvd.32044100859230} \\ \hline
\end{tabular}
\end{center}

\subsection{Bibliothèque municipale de la ville de Lyon}
\begin{center}
\begin{tabular}{ | c | p{13cm} | }
\hline
Volume & URL \\ \hline
1 (1) & \url{https://books.google.fr/books?id=3r3hsfUlRYoC} \\ \hline
1 (2) & \url{https://books.google.fr/books?id=JSHDEpeveFgC} \\ \hline
1 (3) & \url{https://books.google.fr/books?id=3GWA_Kz5AW0C} \\ \hline
\end{tabular}
\end{center}

\subsection{\textit{Bayerische Staatsbibliothek} de Munich}
\begin{center}
\begin{tabular}{ | c | p{13cm} | }
\hline
Volume & URL \\ \hline
1 (1) & \url{https://books.google.fr/books?id=6I9LAAAAcAAJ} \\ \hline
1 (2) & \url{https://books.google.fr/books?id=apBLAAAAcAAJ} \\ \hline
1 (3) & \url{https://books.google.fr/books?id=5pBLAAAAcAAJ} \\ \hline
1 (4) & \url{https://books.google.fr/books?id=J5FLAAAAcAAJ} \\ \hline
\end{tabular}
\end{center}

\subsection{\textit{New York State College of Agriculture at Cornell University}}
\begin{center}
\begin{tabular}{ | c | p{13cm} | }
\hline
Volume & URL \\ \hline
1 (3) & \url{https://books.google.fr/books?id=10FBAAAAYAAJ} \\ \hline
\end{tabular}
\end{center}


\subsection{Bibliothèque de l'Université de Princeton}
\begin{center}
\begin{tabular}{ | c | p{13cm} | }
\hline
Volume & URL \\ \hline
1 (1) & \url{https://hdl.handle.net/2027/njp.32101064529090} \\ \hline
1 (2) & \url{https://hdl.handle.net/2027/njp.32101064529108} \\ \hline
1 (3) & \url{https://books.google.fr/books?id=fTooAAAAYAAJ} \\ \hline
1 (3) & \url{https://hdl.handle.net/2027/njp.32101064529116} \\ \hline
1 (4) & \url{https://hdl.handle.net/2027/njp.32101064529124} \\ \hline
1 (5) & \url{https://hdl.handle.net/2027/njp.32101064529132} \\ \hline
2 (1) & \url{https://hdl.handle.net/2027/njp.32101064529140} \\ \hline
2 (2) & \url{https://hdl.handle.net/2027/njp.32101064529157} \\ \hline
2 (4) & \url{https://hdl.handle.net/2027/njp.32101064529173} \\ \hline
2 (5) & \url{https://hdl.handle.net/2027/njp.32101064529181} \\ \hline
3 (1) & \url{https://hdl.handle.net/2027/njp.32101064529199} \\ \hline
3 (2) & \url{https://hdl.handle.net/2027/njp.32101064529207} \\ \hline
3 (3) & \url{https://hdl.handle.net/2027/njp.32101064529215} \\ \hline
\end{tabular}
\end{center}

\subsection{Université de Rome --- \textit{Instituto de filosofia del diritto}}
\begin{center}
\begin{tabular}{ | c | p{13cm} | }
\hline
Volume & URL \\ \hline
1 (2) & \url{https://books.google.fr/books?id=HjqApJuPx0gC} \\ \hline
\end{tabular}
\end{center}

\renewcommand{\thesection}{B.1}
\chapter{B. Feuille de route et typologie des erreurs}

\section{Feuille de route}
\label{ann:feuille_route}

Cette liste reprend le texte des \issues{} ouvertes dans le \gitlab{} des \odm{} au commencement du stage.

\begin{enumerate}
        \item \textit{Trier et renommer les fichiers} :
    \begin{itemize}
        \item  Identifier les fichiers qui correspondent aux monographies et ceux qui correspondent au paratexte ;
        \item  Donner un identifiant aux fichiers de monographie en fonction des identifiants déjà existants dans le fichier de référence ;
        \item  Créer un identifiant pour les fichiers de paratexte ;
        \item  Ajouter ces identifiants aux \texttt{@xml:id} de chaque fichier ;
        \item  Créer un \textit{mapping} de l’ensemble des fichiers sous la forme d’un CSV avec le nom du fichier et son identifiant ;
        \item  Créer un fichier \texttt{master.xml} contenant des renvois vers les autres fichiers grâce à des \texttt{<xi:includes>}.
    \end{itemize}

    \item \textit{Mettre à jour l’attribut \texttt{@url} dans la balise \texttt{<graphic>}} :
    \begin{itemize}
        \item  Les images des pages sont stockées localement sur Humanum, mais également en ligne sur Internet Archives.
        \item  Trouver l’url de chaque page de chaque volume sur \ia ;
        \item  Remplacer automatiquement le chemin local par l’url de l’image dans chaque fichier.
    \end{itemize}

    \item \textit{Tester la conformité du schéma} :
    \begin{itemize}
        \item  Écrire un script pour tester la validité des arbres XML de chaque fichier ;
        \item  Corriger les erreurs qui seraient signaler par ce script.
    \end{itemize}

    \item \textit{Intégrer les métadonnées des « enquêtés »} :
    \begin{itemize}
        \item  Créer un fichier référentiel de personnes (XML) à partir du fichier CSV de prosopographie ;
        \item  Ajouter aux paragraphes 2 des monographies ("§2. - Etat civil de la famille") des \texttt{@refs} au référentiel de personnes.
    \end{itemize}

    \item \textit{Intégrer modèle de citation dans les chapitres et créer un système de référence bibliographique} :
    \begin{itemize}
        \item  Pour chaque niveau de la structure d'une monographie  ;
        \item  Pour chaque chapitre (sans descendre dans les niveaux) ;
        \item  Par exemple en utilisant DTS.
    \end{itemize}

    \item \textit{Corrections des transcriptions} :
    \begin{itemize}
        \item  Paragraphe par paragraphe, implémenter une correction automatique des transcriptions.
    \end{itemize}

    \item \textit{Corriger le passage de source vers split} :
    \begin{itemize}
        \item  Corriger le script python de transformation des fichiers sources en une suite de fichiers XML TEI.
        \item  Éliminer les traces de teiCorpus.
    \end{itemize}

    \item \textit{Simplifier l'implémentation de la structure logique} :
    \begin{itemize}
        \item Aplatir la structure des monographies :
        \begin{itemize}
            \item  Est-il vraiment nécessaire d'utiliser les div enchâssées ? Une structure à plat avec des marqueurs signalant le début d'une nouvelle section ne suffirait-elle pas ?
            \item  Comment gérer l'articulation entre l'arbre principal (un arbre pour l'ensemble des monographies) et les sous-arbre (un sous-arbre par monographie) ? L'ensemble du corpus n'a pas de front/back mais chaque volume a un front et un back et chaque monographie a potentiellement un front et un back.
            \item  Avec quelles restrictions peut-on créer une structure similaire à celle d'un fichier \LaTeX{} avec ses imports ?
        \end{itemize}
    \end{itemize}
\end{enumerate}

\section{Relevé des erreurs dans la structure
logique}\label{ann:releve_erreurs}

Jean-Damien Généro, 8 juillet 2020\footnote{Reproduction d'une synthèse mise en ligne sur une issue \textit{GitLab}.}.

\begin{center}\rule{3in}{0.4pt}\end{center}

\subsection{Déficit dans la
transcription}\label{ann:deficit-transcr}

\begin{itemize}
\item
  \emph{Déficit partiel :} s1t2\_chapt\_11 et s1t3\_chapt\_10 (manque
  début §7), s1t2\_chapt\_4 (manquent quatre pages de la note A), \texttt{s2t3\_chapt\_14} (manquent vingt lignes).
\item
  \emph{Déficit majeur :} s1t4\_chapt\_8, 11, 15 et s1t5\_chapt\_13 et
  14 (seules les notes ont été transcrites, le reste se trouve dans des
  \texttt{\textless{}figure\textgreater{}}), s1t5\_chapt\_12
  (transcription à partir du §14, le reste dans des
  \texttt{\textless{}figure\textgreater{}}).
\end{itemize}

\begin{center}\rule{3in}{0.4pt}\end{center}

\subsection[Titres non-imprimés]{Titres manquants parce que non-imprimés dans les exemplaires
d'\ia}\label{ann:titre-non-imprimes}

\begin{itemize}
\item
  s1t1\_chapt\_12 : manque le titre de la s/section II.3 ``Mode
  d'existence\ldots{}'' ;
\item
  s2t1\_chapt\_5 : manque le titre de la s/section II.4 ``Histoire
  de\ldots{}'' ;
\item
  s2t1\_chapt\_20 : manque les titres des s/sections II.3 ``Mode
  d'existence\ldots{}'' et II.4 ``Histoire de\ldots{}'' (mais ce titre
  est repris dans l'intitulé du §12 qui suit) ;
\item
  s2t4\_chapt\_9 : manque les titres de la s/section II.1 ``Définition
  du lieu\ldots{}'' ;
\item
  s3t1\_chapt\_3 : manque le titre des s/sections II.2 ``Moyens
  d'existence\ldots{}'', II.3 ``Mode d'existence\ldots{}'' et II.4
  ``Histoire de\ldots{}'' (mais ce titre est repris dans l'intitulé du
  §12 qui suit) ;
\item
  s2t5\_chapt\_15 : manque le titre de la s/section II.4 ``Histoire
  de\ldots{}''.
\end{itemize}

\begin{center}\rule{3in}{0.4pt}\end{center}

\subsection{Structure allégée}\label{ann:structure-allegee}

Titres de s/s/section intégrés au début de paragraphe + certains titres
non-utilisés (spec. §16).

\begin{itemize}
\item
  s2t2\_chapt\_7 (précis) : pas de s/s/section §16 et de titre pour la
  partie Notes (composée d'une seule \emph{nota}) ;
\item
  s2t2\_chapt\_9 (précis) : pas de titre de section II ``Observations
  préliminaires\ldots{}'' et pas de s/s/section §16 ;
\item
  s2t2\_chapt\_11 (précis) : pas de titre de section II ``Observations
  préliminaires\ldots{}'', pas de titre pour les s/s/sections (sauf les
  paragraphes de budget §14 et §15), pas de s/s/section §16, pas de
  section Notes.
\item
  s2t3\_chapt\_7 (précis) : pas de titre de section II ``Observations
  préliminaires\ldots{}'', pas de s/s/section §16, les titres des
  s/s/section sont en début de paragraphe ;
\item
  s2t5\_chapt\_9 (précis) : dans la section II, tous les titres des
  s/s/sections sont en début de paragraphe (à l'exception des
  paragraphes de budget §14 et §15), il n'y a pas de s/s/section §16 ;
\item
  s2t5\_chapt\_16 (précis) : pas de s/s/section dans la s/section II.4
  ``Histoire de\ldots{}'', pas de s/s/section §16, la section Notes est
  occupée par la traduction d'un document.
\end{itemize}

\begin{center}\rule{3in}{0.4pt}\end{center}

\subsection{Pas de s/s/section §16 et pas de section
\textit{Notes}}\label{ann:no-notes-no-16}

\begin{itemize}
\item
  (voir \emph{supra} s2t2\_chapt\_11) ;
\item
  s2t1\_chapt\_6 et s2t5\_chapt\_12.
\end{itemize}

\begin{center}\rule{3in}{0.4pt}\end{center}

\subsection{Remarques et cas
particuliers}\label{ann:remarques-et-cas-particuliers}

\begin{itemize}
\item
  Dans s1t5\_chapt\_9, 10 et 11, une section des notes est intitulée
  ``Précis de monographie'' et organisée comme un précis (càd avec des
  sections et des titres). Néanmoins il ne s'agit pas de précis tels
  qu'on en trouve dans les séries 2 et 3, puisqu'ils ne constituent pas
  un ensemble logique indépendant de la monographie (il y a des
  paragraphes avant et des paragraphes après). Donc le ``précis'' est
  considéré comme une sub\_sub\_section et ses divisions comme des
  sub\_sub\_sub\_section.
\item
  s2t5\_chapt\_17 et 18 découpés chacun en deux fichiers (table
  analytique et table des matières).
\item
  s3t1\_chapt\_2 (``Société générale\ldots{}'') et s3t2\_chapt\_10
  (``Usine hydraulique\ldots{}'') sont des cas particuliers avec des
  structures totalement différentes des autres monographies. En plus de
  l'en-tête, le premier se divise en trois grandes sections :
  observations préliminaires (sorte d'intro), première partie, deuxième
  partie, troisième partie, conclusion et appendices ; le second est
  plus succinct avec l'en-tête, une grande section avec des s/sections
  numérotées et ensuite des appendices.
\item
  Dans s3t2\_chapt\_9 se trouve une partie ``Note sur l'état de la
  famille en 1905'' entre le §13 et §14, que j'ai structurée comme une
  s/s/section.
\item
  La s/section II.5, consacrée aux budgets (§14, §15, §16), n'est pas
  titrée sauf dans les précis de monographie s2t1\_chapt\_6,
  s2t2\_chapt\_9 et 11, s2t3\_chapt\_7 et 14, s2t5\_chapt\_9 et 13,
  s3t1\_chapt\_4 (le titre est ``Budget domestique annuel'').
\end{itemize}