\documentclass[a4paper,twoside,12pt]{book}
\usepackage[utf8]{inputenc}

\usepackage[french]{babel}

\usepackage{fontspec}

\usepackage[colorlinks=true,linkcolor=black,anchorcolor=black,citecolor=black,filecolor=black,menucolor=black,runcolor=black,urlcolor=black]{hyperref}

%Module d'usage facultatif permettant d'intégrer les tables, index, bibliographie, automatiquement à la table des matières
\usepackage{tocbibind}
%\usepackage{lscape}

\usepackage{csvsimple}

\usepackage[margin=2.5cm]{geometry}
\usepackage{setspace}
\setlength{\parindent}{1cm}
\onehalfspacing

\usepackage[backend=biber, sorting=nyt, style=enc]{biblatex}
\usepackage[autostyle]{csquotes}

\addbibresource{mybibliography.bib}

\usepackage{graphicx}
\usepackage{subcaption}
\usepackage{enumerate}
% \usepackage{enumitem} 
\usepackage{eurosym}
\usepackage{url}

\setcounter{biburllcpenalty}{7000}
\setcounter{biburlucpenalty}{8000}

%%%Pour les tableaux
\usepackage{longtable}
\usepackage{multirow}
\usepackage{array}


%%% Les index
%\usepackage{makeidx}
%\usepackage{multind} %Ou splitidx
%\usepackage{index} %…
%\makeindex
%\makeindex{edition}
%\makeindex{texte}
%\newindex{etude}{adx}{and}{Index de l'étude}
%\newindex{edition}{bdx}{bnd}{Index de l'édition}



%%%Édition critique
%\usepackage{eledmac}
%\usepackage{eledpar}

%\footparagraph{A}

%\renewcommand{\Rlineflag}{D}

\hyphenation{}

\usepackage[backend=biber, sorting=nyt, style=enc]{biblatex}
%\addbibresource{Biblio/demo.bib}
%\nocite{lachin_i_2008}

\usepackage{enumerate,lettrine}

\usepackage[multiple]{footmisc}

\usepackage{listings}

\newcommand\ajd{aujourd'hui}
\newcommand\ann{Annexe}
\newcommand\cad{c'est-à-dire}
\newcommand\cf{\textit{cf.}}
\newcommand\commit{\textit{commit}}
\newcommand\commits{\textit{commits}}
\newcommand\fig{fig.}
\newcommand\gb{\textit{Google Books}}
\newcommand\gitlab{\textit{GitLab}}
\newcommand\github{\textit{GitHub}}
\newcommand\kraken{\textit{Kraken}}
\newcommand\ia{\textit{Internet Archive}}
\newcommand\iiif{\textsc{iiif}}
\newcommand\issue{\textit{issue}}
\newcommand\issues{\textit{issues}}
\newcommand\linkeddata{\textit{Linked Data}}
\newcommand\lodm{\textit{Les Ouvriers des deux mondes}}
\newcommand\lse{\textsc{lse-od2m}}
\newcommand\markdown{\textit{Markdown}}
\newcommand\master{\textit{master}}
\newcommand\Mattermost{\textit{Mattermost}}
\newcommand\mergerequest{\textit{merge request}}
\newcommand\mergerequests{\textit{merge requests}}
\newcommand\ocr{\textsc{ocr}}
\newcommand\opensource{\textit{open source}}
\newcommand\odm{\textit{Ouvriers des deux mondes}}
\newcommand\openaccess{\textit{open access}}
\newcommand\opendata{\textit{open data}}
\newcommand\oxygen{\textit{Oxygen}}
\newcommand\pov{point de vue}
\newcommand\povs{points de vue}
\newcommand\push{\textit{push}}
\newcommand\pycharm{\textit{PyCharm}}
\newcommand\rioc{\textsc{rioc}}
\newcommand\sess{\textsc{sess}}
\newcommand\sharedocs{\textit{ShareDocs}}
\newcommand\tnah{\og Technologies numériques appliquées à l'histoire \fg{}}
\newcommand\timeus{\textit{Time~Us}}
\newcommand\transkribus{\textit{Transkribus}}
\newcommand\wikisource{\textit{Wikisource}}

\newcommand{\citecode}[1]{\texttt{\textmd{#1}}}

\title{Valoriser le traitement automatique des données : Le cas des Ouvriers des deux mondes}
\author{Jean-Damien Généro}
\date{Juillet 2020}

\begin{document}

\frontmatter
\begin{titlepage}
\begin{center}

\bigskip

\begin{large}
\sc{\'Ecole nationale des chartes}
\end{large}
\begin{center}\rule{2cm}{0.02cm}\end{center}

\bigskip
\bigskip
\begin{Large}
\textbf{Jean-Damien Généro}\\
\end{Large}
\begin{normalsize} \textit{Licencié en histoire}\\
\textit{Diplômé de Master}\\
\end{normalsize}

\bigskip
\bigskip

\begin{Huge}
\textbf{\sc{Valoriser le traitement\\ automatique des données :}}\\
\end{Huge}
\bigskip
\bigskip
\begin{LARGE}
\textbf{\emph{Le cas des Ouvriers des deux mondes} }\\
\end{LARGE}

\bigskip

\begin{figure}[h]
    \centering
    \includegraphics[width=12cm]{img/couv.png}
    \label{fig:ill_couv}
\end{figure}

\begin{large}
\end{large}
\vfill

\begin{large}
Mémoire 
pour le diplôme de Master \\
\tnah \\
\medskip
2020

\bigskip

\includegraphics[width=1.25cm]{img/etalab-logo.png}

\begin{small}
{\textit{Contenu sous licence ouverte}}
\end{small}

\end{large}
\end{center}
\end{titlepage}
\clearpage
\thispagestyle{empty}
\cleardoublepage
\section*{Résumé}
\addcontentsline{toc}{chapter}{Résumé}
\markboth{Résumé}{}

\bigbreak

\textit{Le programme ANR \timeus{} s’intéresse aux ouvriers et aux ouvrières du textile de la fin du \textsc{xviii}\ieme ~siècle au début du \textsc{xx}\ieme ~siècle et rassemble pour cela une large documentation composée de documents manuscrits et d'imprimés. Au sein de ces derniers se trouvent les monographies de familles des} Ouvriers des deux mondes \textit{publiées de 1857 à 1930 par Frédéric Le Play (1806-1882) et la Société internationale des études pratiques d’économie sociale.}

\textit{Ce corpus de treize volumes, composé de 114 monographies, a été transcrit et structuré automatiquement au format XML-TEI par un programme utilisant les logiciels d’OCR Transkribus et Kraken, et le langage de programmation Python. Le présent mémoire se propose d'analyser les actions menées au cours d'un stage de fin d'études pour valoriser les résultats de cette structuration automatique, incluant son contrôle, la correction des erreurs et l’usage des humanités numériques pour implémenter un encodage scientifique permettant l’exploitation des données et des transcriptions par les chercheurs et les chercheuses.}

\bigbreak

\bigbreak

\bigbreak

\textbf{Mots-clés:} XML ; TEI ; Python ; traitement automatique des données ; transcription automatique ; édition numérique ; ALTO ; \ocr ; \kraken{} ; \transkribus{} ; \gitlab{} ; Frédéric Le Play ; \lodm{} ; enquêtes sociologiques ; monographies de familles ; \timeus{} ; Inria.

\bigbreak

\bigbreak

\bigbreak

\textbf{Informations bibliographiques:} Jean-Damien Généro, \textit{Valoriser le traitement automatique des données : Le cas des Ouvriers des deux mondes}, mémoire du Master \og Technologies numériques appliquées à l'histoire \fg{}, dir. Alix Chagué et Vincent Jolivet, École nationale des chartes, 2020.

\bigbreak

\bigbreak

\bigbreak

\textbf{Soutenance:} mémoire présenté et soutenu publiquement le 19 octobre 2020 à l'École nationale des chartes, devant un jury composé d'Édouard Vasseur, président, professeur d’Histoire des institutions, diplomatique et archivistique contemporaines, de Vincent Jolivet, responsable de la mission projets numériques et d'Alix Chagué, ingénieure de recherche et de développement de l’équipe ALMAnaCH d’Inria ; sanctionné par une mention Trés bien et la note de 18/20.

\bigbreak

\bigbreak

\textbf{Illustration de couverture:} Émile Savoy, \textit{Chocolatier de la fabrique de chocolat au lait F.-L. Cailler à Broc (canton de Fribourg, Suisse)}, dans \lodm, Paris, série 3, 1913, p. 325, \og La fabrique Cailler \fg.
\clearpage
\thispagestyle{empty}
\cleardoublepage
\chapter*{Liste des sigles et abréviations}
\addcontentsline{toc}{chapter}{Liste des sigles et abréviations}
\markboth{Liste des sigles abréviations}{} 

\begin{center}
\textit{Institutions}
\end{center} 

\begin{itemize}
    \item ALMAnaCH : \textit{Automatic Language Modelling and Analysis \& Computational Humanities} (Inria Paris)
    \item ANR : Agence nationale de la recherche
    \item CMH : Centre Maurice-Halbwachs (EHESS et ENS Paris)
    \item CNRS : Centre National de Recherche Scientifique
    \item EA : Équipe d'accueil
    \item EHESS : École des Hautes Études en Sciences Sociales
    \item ENS : École normale supérieure
    \item ICT : Identités, Cultures et Territoires (Université de Paris)
    \item Inria : Institut national de recherche en informatique et en automatique
    \item LARHRA : Laboratoire de Recherche Historique Rhône-Alpe (Lyon 2)
    \item READ : \textit{Recognitionand Enrichment of Archival Documents}
    \item TGIR : Très grande infrastructure de recherche
    \item TELEMMe : Temps, Espaces, Langages, Europe Méridionale-Méditerranée (Université d’Aix-Marseille)
    \item UMR : Unité mixte de recherche
\end{itemize}

\bigbreak

\begin{center}---

\bigbreak

\textit{Informatique et nouvelles technologies}
\end{center} 

\bigbreak

\begin{itemize}
    \item ALTO : \textit{Analyzed Layout and Text Object}
    \item CSV : \textit{Comma-separated values}
    \item GPU : \textit{Graphics processing unit}
    \item IETF : \textit{Internet Engineering Task Force}
    \item JPEG : \textit{Joint Photographic Experts Group}
    \item JP2 : \textit{JPEG 2000}
    \item \lse{} : \textit{Logical Structure Extraction from Les Ouvriers des Deux Mondes}
    \item OCR : \textit{Optical Character Recognition}
    \item PDF : \textit{Portable Document Format}
    \item RFC : \textit{Request for comments}
    \item TEI : \textit{Text Encoding Initiative}
    \item URI : \textit{Uniform Resource Identifier}
    \item XML : \textit{Extensible Markup Language}
\end{itemize}

\clearpage
\thispagestyle{empty}
\clearpage
\thispagestyle{empty}
\cleardoublepage
% \section*{Préambule : un stage confiné ?}
\addcontentsline{toc}{chapter}{Préambule : un stage confiné ?}
\markboth{Préambule}{} 

Le stage qui a donné lieu au présent mémoire n'a pas commencé sous les meilleurs auspices. Des difficultés administratives et procédurales, puis le grand confinement et les difficultés plus grandes encore qui en ont résulté, ont failli avoir raison de lui. Il a fallu se battre pour que l'Université de Paris admette que la fermeture des établissements d'enseignement annoncée par le président de la République le 12 mars 2020, puis les mesures de confinement à partir du 17 mars, ne signifiaient en aucun cas une suspension de l'action de son administration et un report automatique et sans appel des procédures en court, dont celle concernant les stages. Je tiens donc en premier lieu à remercier Mesdames Alix Chagué et Manuela Martini, Messieurs Julien Cassefières et Thibault Clérice, qui m'ont apporté un tant soit peu de soutien et d'aide dans cette procédure extrêmement pénible, absolument anormale et pour laquelle j'ai dépensé une quantité d'énergie démesurée.

Contactée, la présidence de l'Université de Paris mit fin au marasme, apposa les cachets requis et présenta ses excuses ; le stage pu ainsi commencer, avec une semaine de retard sur la date prévue.

Intégrer une équipe d'ingénierie et de recherche tout en étant confiné chez soi n'est pas chose aisée. Les relations humaines que cet exercice suppose en temps normal ont été réduites au minimum et sont passées par des courriels, des échanges sur l'espace de discussion instantanée d'INRIA et, principalement, des visioconférences sur \textit{Zoom}.

Inconnue, immédiatement décevante et parfois pénible, cette façon de travailler a mis un certain temps à s'imposer pour moi et les premières semaines ont été difficiles. Début juin, c'est-à-dire à la moitié du stage, Alix Chagué me demanda de lui transmettre un bilan des compétences que j'avais acquises jusqu'ici. Elles étaient nombreuses du point de vue des savoir-faire, car, bon gré mal gré, je n'avais eu d'autre choix que de m'adapter à ce travail à distance et à remplir les missions qui m'avaient été assignées. Ce faisant, un des objectifs que je m'étais fixé pour ce stage fut très vite satisfait : mes savoir-faire se sont développés par l'apprentissage et la maîtrise de nouvelles techniques.

\textit{Quid} des savoir-être ? J'avais listé plusieurs points dans le bilan de mi-stage, sur un ton légèrement humoristique, et notamment saluer, faire une présentation et suivre une réunion en visioconférence. Un véritable savoir-être se cachait derrière cet humour désabusé :  ce stage m'a appris le travail à distance. Cela suppose beaucoup de choses. Outre une organisation et une discipline peut-être plus importantes que pour un travail de bureau, il s'agit d'être seul la majorité du temps. De ne pas avoir de collègue avec qui l'on peut échanger et progresser dans la résolution d'un problème. En un mot, il s'agit de travailler plus avec soi-même qu'avec les autres et donc de se \textit{débrouiller}.
% \clearpage
% \thispagestyle{empty}
% \cleardoublepage
\section*{Introduction : le projet Time Us}
\addcontentsline{toc}{chapter}{Introduction : le projet Time Us}

\bigbreak

\og Reconstituer les rémunérations et les budgets temps des travailleuses et des travailleurs du textile dans quatre villes industrielles françaises (Lille, Paris, Lyon, Marseille) dans une perspective européenne et de longue durée \fg{}\footnote{Présentation du programme sur le site de l'ANR (\url{https://anr.fr/Projet-ANR-16-CE26-0018}, consulté le \today).} est l'objectif du programme ANR \textit{Time Us}. Son intitulé exact, \og Rémunérations et usages du temps des femmes et des hommes en France de la fin du \textsc{xvii}\ieme ~siècle au début du \textsc{xx}\ieme ~siècle \fg{}\footnote{Présentation du programme sur le site du LARHA (\url{http://larhra.ish-lyon.cnrs.fr/anr-time-us}, consulté le \today).} indique qu'il s'articule à la fois autour d'un temps long qui court depuis les premières manufactures de l'époque Moderne jusqu'à la fin de la révolution industrielle au tournant des \textsc{xix}\ieme  ~et~\textsc{xx}\ieme ~siècles, mais aussi sur le temps individuel et quotidien qui est celui d'un ouvrier ou d'une ouvrière au sein de ces grands mouvements historiques. Les relations entre la rémunération et le temps sont au c\oe{}ur des questionnements du programme. Cette attention portée à l'usage du temps se traduit dans la dénomination courante du programme, \og Time us \fg{}, abréviation de l'anglais \og \textit{time usage} \fg{}.

\textit{Time Us} se concentre principalement sur les femmes et plus encore sur les ouvrières du textile, industrie dans laquelle \og elles sont présentes dans toutes les phases du processus productif \fg{}\footnote{\cite[p. 1]{inria}.}. Le programme tend à combler le biais des genres dans l'historiographie du travail industriel en réalisant une opération de collecte et de traitement de la documentation manuscrite et imprimée relative à l'emploi et aux activités quotidiennes des femmes\footnote{\textit{Ibid}.}.  Ainsi, le moteur de \textit{Time us} est moins la production d'une réflexion scientifique autour du travail des femmes que la constitution d'un corpus documentaire sériel et prêt à être exploité par des chercheurs  d'horizons multiples. La pluridisciplinarité est en effet un aspect majeur du programme, qui souhaite une utilisation de ses données dans un maximum de champs de recherche des sciences humaines et sociales, notamment \og en  histoire économique et sociale, en histoire de la famille et du genre, en histoire des conflits du travail et de la culture des classes populaires \fg{}\footnote{\textit{Ibid}, p. 2.}.

La documentation est essentiellement constituée de documents manuscrits conservés par la Bibliothèque municipale de Lyon et les dépôts d'archives départementaux et municipaux lillois, parisien, lyonnais et marseillais. Très diverse, elle est issue d'organes officiels, à l'instar des conseils prud'homaux, des chambres de commerce ou encore des tribunaux de commerce, mais aussi d'organismes ou d'individus privés tels que les archives du tisseur lyonnais Pierre Charnier (1795-1857)\footnote{Les papiers du canut Pierre Charnier font partie du fonds \og Fernand Rude \fg{} de la Bibliothèque municipales de Lyon, nommé d'après l'historien qui les possédait avant leur versement (\url{https://www.bm-lyon.fr/collections-patrimoniales-et-specialisees/explorer-les-collections/article/fernand-rude}, consulté le \today).}, les registres comptables (1817-1821) et le dossier de faillite (1821-1822) de la filature parisienne Dupuis-Drouet\footnote{Archives de Paris, D 12U1, n° 375-376 (\url{http://archives.paris.fr/arkotheque/inventaires/ead_ir_consult.php?a=4&ref=FRAD075_000727}, consulté le \today).}, etc.

En sus de cette documentation manuscrite, le programme \textit{Time Us} s'appuie également sur trois grands corpus d'imprimés. Le premier se compose de neuf titres de la presse ouvrière lyonnaise, entièrement numérisés sur le site \textit{Numelyo}, espace digital de la Bibliothèque municipale de  Lyon\footnote{ Le corpus se trouve à l'adresse \url{https://collections.bm-lyon.fr/PER003}, (consulté le \today).}. L'intérêt du programme pour ce corpus porte sur les nombreux compte rendus d'audience du Conseil des prud'hommes qui s'y trouvent, mais aussi pour des reproductions ou des extraits de discours, des lettres ou des analyses économiques et sociales relayés par ces journaux. Quatre des titres sont publiés dans la première moitié des années 1830 (\textit{L'Écho de la fabrique}, \textit{L’Écho des travailleurs}, \textit{L'Indicateur} et \textit{La Tribune prolétaire}), les cinq restant dans les années 1840 (\textit{L’Écho des ouvriers}, \textit{L’Écho de la Fabrique de 1841}, \textit{L’Écho de l'industrie}, \textit{L'Avenir}, \textit{La Tribune Lyonnaise})\footnote{Présentation de ce corpus sur le wiki du programme : \url{http://timeusage.paris.inria.fr/mediawiki/index.php/Documentation_régionale_-_Presse_lyonnaise} (consulté le \today).}.

Un second corpus d'imprimés est constitué de monographies collectées sur le site \textit{Gallica} de la Bibliothèque nationale de France au format PDF. Très divers, on y trouve à la fois \textit{L'ouvrière} de Jules Simon (1861), un \textit{Dictionnaire général des tissus anciens et modernes} (1859-1863) ou encore un \textit{Traité complet sur la fabrication des étoffes de soie} (1859)\footnote{Liste complète sur le wiki du programme : \url{http://timeusage.paris.inria.fr/mediawiki/index.php/Aperçu_des_états\#Imprimés_divers} (consulté le \today).}. Il s'agit d'une base complémentaire au projet, qui doit permettre de contextualiser la base archivistique principale, notamment en permettant de suivre l'évolution du  vocabulaire afférent au textile sur la période étudiée.

Le troisième et dernier corpus d'imprimés est composé des treize volumes de la série des \textit{Ouvriers des deux mondes}.

\begin{center}
$\star$
\end{center} 

Initiées par le sociologue Frédéric Le Play (1806-1882), les \textit{Ouvriers des deux mondes} sont des enquêtes sociologiques --- couramment désignées comme les \og monographies \fg{} --- conduites par les membres de la Société internationale des études pratiques d’économie sociale\footnote{Actuelle Société d'économie et de sciences sociales, désormais abrégée en SESS.} de 1857 à 1928. Répartis en trois séries comptant un total de cent vingt-six monographies\footnote{\cite[p. 95]{lorry}.}, \textit{Les Ouvriers des deux mondes} sont la deuxième entreprise d'étude empirique de Le Play, après celle des \textit{Ouvriers européens} dont la première édition a lieu en 1855\footnote{\textit{Ibid}.}.

Dans sa préface du numéro spécial de la revue \textit{Les Études sociales} consacré aux monographies leplaysiennes, Alain Chenu relève la prégnance de l'assimilation de celles-ci à des \og mines \fg{} dans lesquelles les chercheurs peuvent \og [puiser] \fg{} à leur guise, tant les sujets abordés et les étendues géographiques traitées sont nombreux\footnote{\cite[p. 5]{chenu}.}. \textit{Les Ouvriers des deux mondes}, dont le titre fait écho à \textit{La revue des deux mondes} fondée en 1829, présentent en effet une succession de \og trajectoires et récits de vie de familles ouvrières \fg{}\footnote{\cite[p. 193]{baciocchi}.} établies de part et d'autre de la Méditerranée. Se succèdent ainsi une enquête consacrée à un charpentier de Paris\footcite{mono001a}, à un métayer de la banlieue de Florence\footcite{mono005a} et à un menuisier-charpentier de Tanger\footcite{mono012a}.

Alain Chenu, tout en appuyant la métaphore des \og mines \fg{} leplaysiennes, dénonce son caractère restrictif, et notamment l'idée selon laquelle les monographies seraient une succession d'enquêtes indépendantes. Le Play et la SESS ont en effet conservé une même \og grille d'observation \fg{} depuis la première (1856) jusqu'à la dernière enquête (1928), construisant \textit{de facto} un \og système dont les éléments prennent sens les uns par rapport aux autres \fg{}\footnote{Alain Chenu, \textit{op. cit.}, p. 5.}.

\textit{Les Ouvriers des deux mondes} présentent un double intérêt pour \textit{Time Us}. D'une part, leur attention dirigée de manière exclusive envers les familles ouvrières en font un matériau privilégié pour ce programme, d'autant que les monographies se focalisent sur le budget et son usage\footnote{Le budget est \og à la fois la méthode et le résultat \fg{} des monographies : \cite[p. 11]{cardoni}}. Aucun individu de la cellule familiale n'est ignoré : l'ouvrier --- il s'agit le plus souvent du père, qui peut être accompagné de ses frères ou de ses fils ---, ses descendants, ses ascendants, sa femme évidemment, mais aussi les domestiques\footcite{mono018a} et parfois les esclaves\footnote{Narcisse Cotte, \og Menuisier-charpentier... \fg{}, \textit{op. cit.}}. Ainsi, si seulement quatorze enquêtes ont pour sujet des familles travaillant dans l'industrie du textile, l'ensemble reste important à l'échelle du programme ANR dans la mesure où chaque enquête s'attache à établir le budget réservé aux matières textiles et à leur utilisation par la famille enquêtée.

D'autre part, le fait qu'il s'agit d'imprimés et la reproduction systématique de la même structure logique dans chaque monographie permettent d'envisager un traitement informatique de chaque volume.
\cleardoublepage
\part*{Bibliographie}
\chapter{Bibliographie}

\printbibliography[heading=subbibliography,keyword=bib,title={Bibliographie générale}]
\addcontentsline{toc}{section}{Bibliographie générale}

\printbibliography[heading=subbibliography,keyword=od2m,title={\lodm}]
\addcontentsline{toc}{section}{\lodm}

\newpage
\thispagestyle{empty}
\mbox{}
\newpage

\mainmatter
\part{Un corpus déjà structuré}

\chapter{Un corpus d'imprimés}

Début du premier chapitre

\chapter{Des numérisations multiples}

Début du deuxième chapitre

\chapter{Un encodage automatique}

Début du troisième chapitre
\cleardoublepage
\part{Une structuration à reprendre : des tâches manuelles et semi-automatiques}

\clearpage
\thispagestyle{empty}
\cleardoublepage

\chapter{Gestion de projet : méthodologie de travail}


\section{Outils de développement}

\subsection{\gitlab}

La gestion quotidienne du programme \timeus{} se fait grâce à un dépôt sur \gitlab{}, un logiciel libre permettant aux entités comme Inria de créer une plate-forme interne de développement informatique. Sur \gitlab{} se trouve le dépôt central, organisé en plusieurs dossiers, dont un est réservé aux \odm. La technologie \textit{git} permet d'administrer les différentes versions du projet, sur une échelle à la fois verticale (les anciennes versions, appelées \commits, restant accessibles à travers un historique) et horizontale (le travail peut s'effectuer sur des \textit{branches} divergentes de la branche principale dite \master{} sans affecter l'état de cette dernière). Les \commits possèdent tous un commentaire où l'utilisateur peut résumer ses modifications.

Chaque ingénieur peut rapatrier le dépôt \gitlab{} en local pour travailler dessus ; ce rapatriement est un \textit{pull}. Il peut ensuite effectuer l'opération inverse consistant à mettre ses \commits{} en ligne. Son équipe peut ainsi prendre connaissance des dernières avancées de la branche sur laquelle il travaille. Cette opération de mise en ligne s'effectue \textit{via} un \textit{push}.

Une fois le travail dans une branche achevé, celle-ci peut être fusionnée avec \master. \gitlab{} propose une interface pour effectuer des \mergerequests{}. Il s'agit d'une demande de fusion, où \gitlab{} affiche l'historique de la branche. Les utilisateurs peuvent ainsi contrôler l'ensemble des \commits{} de la branche locale et vérifier qu'ils ne vont pas corrompre \master{} en créant des conflits (\textit{merge conflicts}) ; si des problèmes sont détectés, ils ont la possibilité d'échanger par messages afin de les résoudre.

\gitlab{} permet enfin à ses utilisateurs d'échanger sur des éléments posant problème, d'effectuer des suggestions ou encore de porter à l'attention de leur équipe une ressource utile à travers des \issues. Les \issues{} et les \mergerequests{} sont numérotées (\citecode{\#\textbackslash d} et \citecode{!\textbackslash d}\footnote{Par exemple \citecode{\#5} fait référence à la cinquième \issue{} et \citecode{!5} à la cinquième \mergerequest{}.}), écrire leurs numéros dans \gitlab{} ou dans le commentaire de modification d'un \commit{} permettant de faire automatique référence à elles à travers un lien interne.

\subsection{Le dépôt \gitlab{} des \odm}

L'équipe ALMAnaCH possède un espace sur la plate-forme \gitlab{} d'Inria, où un dossier (en accès restreint) est réservé au programme \timeus{}. C'est ici, dans un sous-ensemble, que se trouve le dépôt des \odm. Le script \lse{} est déposé dans un dossier externe.

La branche \master{} compte quatre sous-dossiers :

\begin{itemize}
    \item \citecode{source} contient les treize fichiers XML-TEI des volumes des \odm{} ;
    \item \citecode{script} contient les scripts développés afin d'automatiser le traitement des fichiers ;
    \item \citecode{files} contient les fichiers XML-TEI générés à partir des fichiers sources et modifiés automatiquement ou à la main en vue de leur publication (monographies et fichiers de paratexte) ;
    \item \citecode{metadata} contient des fichiers de métadonnées sur le projet.
\end{itemize}

Ce dossier était notre espace de travail principal, notre première action ayant consisté en son rapatriement au niveau local. Notre méthodologie de travail était la suivante :

\begin{enumerate}
    \item Une branche était créée pour chaque mission ;
    \item Des \commits{} étaient effectués en local sur cette branche ;
    \item Les \commits{} d'une journée étaient mise en ligne sur \gitlab{} par un \textit{push} ;
    \item Lorsque la mission était achevée, une \mergerequest{} était ouverte ;
    \item La \mergerequest{} était acceptée et la branche fusionnée avec \master.
\end{enumerate}

\section{Espaces de discussion}

La situation de confinement d'avril à mai et le maintien de la fermeture aux stagiaires des locaux d'Inria de juin à juillet a conduit à la mise en place d'outils de discussion, ou à l'exploitation intensive d'outils pré-existants.

\section{Feuille de route}

bla

\chapter{Contrôle du découpage des fichiers source}

Début du chapitre 5.

\chapter[Correction des erreurs d'implémentation]{Correction des erreurs d'implémentation de la structure logique}

Début du chapitre 6.
\cleardoublepage
\part{Un corpus à valoriser}

\clearpage
\thispagestyle{empty}
\cleardoublepage

\chapter{Une valorisation plurielle}

Le processus de valorisation des données de la recherche peut emprunter deux voies différentes. L'une consiste à inscrire ces données dans le temps long et à favoriser leur ré-utilisation grâce à une structuration standardisée et documentée, l'autre entraîne leur publication dans un cadre institutionnel et budgétaire conjoncturel qui, s'il peut produire des résultats visibles rapidement, n'est pas assuré d'être reconduit et pérennisé.

Les fichiers des \odm{} se trouvent face à cette injonction qui, sans être réellement paradoxale, conjugue une obligation --- publier avant la fin du programme ANR \timeus{} --- et une ambition --- assurer la pérennité d'un corpus numérique.

\section{Édition papier, édition numérique}

Un postulat commun veut qu'une édition numérique offre à son utilisateur plus de possibilité qu'une édition papier n'en offre à son lecteur\footnote{\og \textit{Often these descriptions glance at their print predecessors, usually with expressions of how much more these digital editions can contain than ever could be included in print editions, and how much more the reader can do with them} \fg{} : \cite[p. 105-106]{robinson}.}. Le numérique permet en effet de concevoir des plate-formes sur mesure pour la consultation des textes, et les outils des humanités numériques permettent que \og de nombreuses données non interrogeables jusqu’à présent [fassent] l’objet d’enquêtes \fg\footcite[p. 20]{duval}. \og des structures cachées, des faits de système difficilement décelables à l’œil et à la main \fg{} deviennent ainsi accessibles\footcite{duval}.

Dans le programme \timeus{}, l'édition numérique des \odm{} a éveillé l'intérêt d'au moins trois participants. Le LARHRA de l'université de Lyon 2, dans le strict respect des objectifs du programme, souhaite utiliser les informations économiques fournies par les tableaux de budget et celles d'ordre prosopographique contenues dans le paragraphe \og §2 --- État civil de la famille \fg. L'équipe ALMAnaCH d'Inria a pour sa part la volonté de puiser dans les champs lexicaux des mondes ouvrier et industriel afin d'alimenter des algorithmes de traitement automatique du langage (TAL). Enfin, le Centre Maurice-Halbwachs (CMH) cherche à fournir à la communauté scientifique une édition numérique des \odm{} qui intégrerait des informations matérielles sur la constitution du corpus et le travail de la \sess.

On le voit, les objectifs poursuivis par ces entités sont très divers. Les matériaux qui permettront de les réaliser le sont tout autant : le LARHRA et ALMAnaCH ont besoin de donner brutes issues des tableaux statistiques (des chiffres) et du texte (des mots), tandis que le CMS rassemble des métadonnées nouvelles qui proviennent de plusieurs corpus. Le LARHRA a également besoin d'une transcription de qualité pour s'assurer de la viabilité des informations prosopographiques du second paragraphe.

Cette pluralité de directions illustre la tension qui traverse les éditions numériques : elles portent avant tout sur un document et non sur une \oe{}uvre\footnote{\og \textit{Two decades of making digital editions, and recent papers about digital editions, have moved the needle away from the “work” to the “document”, to the point where we might need only think of “documents”} \fg{} : \cite[p. 107]{robinson}}. Cette distinction est issue de la triade document, texte, \oe{}uvre (\textit{document}, \textit{text}, \textit{work}) qui désigne les dimensions matérielle, linguistique et intellectuelle d'un écrit\footnote{\og \textit{Work} désigne le texte de l’auteur, éventuellement le texte correspondant à la volonté de l’auteur, et implique la notion d’authenticité ; \textit{text} dénomme la séquence linguistique attestée dans un document transmettant l’œuvre ; enfin \textit{document} est une manifestation physique d’un text \fg{} : \cite[p. 15-16]{duval}.}. De fait, l'encodage que nous avons décrit dans la partie précédente fait la part belle au document digital \odm{} d'\ia. \transkribus{} et le script \lse{} mobilisent l'une des fonctionnalités majeures de la TEI, la reproduction fidèle d'un document grâce à des ensembles \texttt{<facsimile>}\footcite[p. 124]{robinson}. L'\oe{}uvre n'est pas pour autant oubliée. Elle se rencontre dans la structure logique leplaysienne, là encore reproduite fidèlement par le biais des divisions et des titres.

Cette coexistence apparente n'a cependant pas vocation à durer : dans la vision de \timeus, c'est bien l'\oe{}uvre et non le document qui doit prendre le dessus. Il n'est pas question de concevoir un support de consultation qui présenterait des échantillons successifs correspondant au contenu d'une page : cela reviendrait à reproduire l'interface de visualisation d'\ia. Le premier essai d'édition des fichiers XML, mené par Alix Chagué, consiste ainsi en un document HTML où l'intégralité du texte du premier volume est reproduit, organisé en fonction de la structure logique et non des zones de segmentation\footnote{Cette démonstration est visible à cette adresse : \url{http://demo-leplay.herokuapp.com/volume_parsed_test.html} (consulté le \today).}. Elle pourrait cependant être améliorée avec des informations issues du document, à commencer par la traduction, par exemple entre crochets droits, des balises \texttt{<pb>} à chaque changement de page.

Ces observations montrent qu'une édition numérique se doit d'être équilibrée et de rendre compte à la fois du document-texte et de l'\oe{}uvre-texte\footnote{\og \textit{A scholarly edition must, so far as it can, illuminate both aspects of the text, both text-as-work and text-as-document} \fg{} : \cite[p. 123]{robinson}.}, sans quoi elle risque de perdre tant ses lecteurs\footnote{\og \textit{But there are dangers here. (...) Facsimile editions in print form are of very little use to the reader, or even to scholars, whose interest (...) is likely to be in questions of how the received text changed over time, how it was received, how it was altered, transformed, passed into different currencies. If we make only digital documentary editions, we will distance ourselves and our editions from the readers} \fg{} : \cite[p. 127]{robinson}.} que ses utilisateurs\footnote{\og La lisibilité des éditions électroniques n’a rien à envier à celle des éditions papier. (...) Parfois, les interfaces ne sont pas intuitives et requièrent une longue familiarisation ; d’autres fois, des aides à la lecture systématiquement présentes dans les éditions papier disparaissent \fg{} : \cite[p. 21]{duval}.}.

\section{Retrouver les fascicules sous le volume}

Au premier abord, il est aisé de considérer les \odm{} comme une \oe{}uvre. Mais plus on progresse dans l'histoire de cette entreprise, plus cette idée première se fragilise. En effet, si ce corpus s'incarne aujourd'hui dans des volumes, ceux-ci étaient autrefois des fascicules. Il est en outre constitué de monographies réparties en trois série, tout en formant un ensemble cohérent dont les membres \og prennent sens les uns par rapport aux autres \fg{}\footcite[p. 5]{chenu}. D'une certaine manière, la volonté du CMH d'explorer l'histoire matérielle du corpus pour compléter les métadonnées des fichiers XML remet en question l'existence de \og l'\oe{}uvre \odm{} \fg{}.

Les enjeux de l'édition numérique d'un texte imprimé aux \textsc{xix}\ieme{} et \textsc{xx}\ieme~siècles divergent de ceux sous-entendus par l'édition de texte imprimé lors de siècles antérieurs. Aucune ré-impression n'est ici attestée : le texte est le même d'un exemplaire à l'autre. C'est la raison pour laquelle l'encodage ne fait aucun effort de lématisation et qu'il se base, de fait, sur les seules numérisations des exemplaires de Toronto.

Dans l'encodage d'un texte imprimé tel que \lodm{}, le défi se situe non pas au niveau du texte et de ses différentes versions, mais bien dans la restitution de la tradition matérielle qui a amené à la constitution des volumes. Comment traduire dans l'encodage les stratégies mises en place par les différents relieurs pour fondre les fascicules dans le volume ? Dans les exemplaires de la Bibliothèque nationale de France, les feuillets liminaires des fascicules ont été placés après les tables, là où ailleurs ils ont été conservés dans le flux du texte. Il y a ici une subtilité dont l'encodage de niveau \og document \fg{} ne se préoccupe pas, et qui portant est essentiel pour les deux autres niveaux.

Rappelons qu'un document TEI est divisé en deux grandes parties, que sont le \texttt{<teiHeader>} dévolu aux métadonnées et le \texttt{<text>} contenant le document. C'est dans le premier que les observations relevées par le CMH doivent prendre place, et plus précisément dans la section \texttt{<sourceDesc>}. Plusieurs ensembles de description des exemplaires (\texttt{<msDesc>}) peuvent en effet y être mobilisés afin de rendre compte des différences entre les exemplaires parisiens et ceux de Toronto.

conséquence sur les interfaces de visualisation ?

\section{\textit{Quid} de la donnée ?}


\chapter{Les données graphiques dans un flux textuel}

\chapter{Qualité des transcriptions}
\cleardoublepage
\part*{Conclusion}
\addcontentsline{toc}{part}{Conclusion}
\markboth{Conclusion}{Conclusion} 

Blbla

\backmatter
\part*{Annexes}
\addcontentsline{toc}{part}{Annexes}
\appendix
\renewcommand{\thechapter}{A}
\chapter{A. \lodm}

\section{Liste des monographies et fichiers correspondant}
\label{mapping}

\textit{La date de publication du volume est indiquée entre parenthèses, tandis que les dates de parution des fascicules le sont entre crochets droits.}

\subsection{Série 1, vol. 1 (1857).}
\label{mappings1t1}

URL sur \ia{} : 

\url{https://archive.org/details/lesouvriersdesde01sociuoft}.

\begin{center}
\begin{longtable}{ | c | p{9.5cm} | c | }
\hline
Id & Intitulé & Fichier \\ \hline
\texttt{401a} & Page de titre & \texttt{s1t1\_chapt\_1.xml} \\ \hline
\texttt{402a} & Avertissement. Considérations générales sur la Société internationale des études pratiques d'économie sociale. Son but et ses moyens d'action. & \texttt{s1t1\_chapt\_2.xml} \\ \hline
\texttt{403a} & Institution. Société internationale des études pratiques d'économie sociale. Fondation et premiers travaux. & \texttt{s1t1\_chapt\_3.xml} \\ \hline
\texttt{404a} & Définitions par ordre alphabétiques des termes à employer dans les monographies, pour désigner les ouvriers, leurs moyens d'existence, et les rapports qui les unissent soit entre eux, soit avec les autres classes. & \texttt{s1t1\_chapt\_4.xml} \\ \hline
\texttt{405a} & Explications des signes de renvoi et des abréviations. & \texttt{s1t1\_chapt\_5.xml} \\ \hline
\texttt{001a} & Charpentier de Paris (Seine - France), de la Corporation des compagnons du Devoir & \texttt{s1t1\_chapt\_6.xml} \\ \hline
\texttt{002a} & Manœuvre-Agriculteur de la Champagne pouilleuse (Marne - France) & \texttt{s1t1\_chapt\_7.xml} \\ \hline
\texttt{003a} & Paysans en communauté du Lavedan (Hautes-Pyrénées - France) & \texttt{s1t1\_chapt\_8.xml} \\ \hline
\texttt{004a} & Paysan du Labourd (Basses-Pyrénées - France) & \texttt{s1t1\_chapt\_9.xml} \\ \hline
\texttt{005a} & Métayer de la banlieue de Florence (Grand-Duché de Toscane) & \texttt{s1t1\_chapt\_10.xml} \\ \hline
\texttt{006a} & Nourrisseur de vaches de la banlieue de Londres (Middlesex - Angleterre) & \texttt{s1t1\_chapt\_11.xml} \\ \hline
\texttt{007a} & Tisseur en châles de la fabrique urbaine collective de Paris (Seine - France) & \texttt{s1t1\_chapt\_12.xml} \\ \hline
\texttt{008a} & Manœuvre-agriculteur du comté de Nottingham (Angleterre) & \texttt{s1t1\_chapt\_13.xml} \\ \hline
\texttt{009a} & Pêcheur côtier, maître de barque de Saint-Sébastien (Guipuscoa - Espagne) & \texttt{s1t1\_chapt\_14.xml} \\ \hline
\texttt{406a} & Tables alphabétique et analytique des matières contenues dans ce tome premier. & \texttt{s1t1\_chapt\_15.xml} \\ \hline
\texttt{407a} & Liste des monographies destinées aux prochaines publications de la société d'économie sociale. & \texttt{s1t1\_chapt\_16.xml} \\ \hline
\texttt{408a} & Tables des matières contenues dans ce tome premier & \texttt{s1t1\_chapt\_17.xml} \\ \hline
\end{longtable}
\end{center}

\subsection{Série 1, vol. 2 (1858).}
\label{mappings1t2}

URL sur \ia{} : 

\url{https://archive.org/details/lesouvriersdesde02sociuoft}.

\begin{center}
\begin{longtable}{ | c | p{9.5cm} | c | }
\hline
Id & Intitulé & Fichier \\ \hline
\texttt{409a} & Page de titre & \texttt{s1t2\_chapt\_1.xml} \\ \hline
\texttt{410a} & Avertissement & \texttt{s1t2\_chapt\_2.xml} \\ \hline
\texttt{010a} & Ferblantier, couvreur et vitrier d'Aix-les-Bains (Savoie - États Sardes) & \texttt{s1t2\_chapt\_3.xml} \\ \hline
\texttt{011a} & Carrier des environs de Paris (Seine - France) & \texttt{s1t2\_chapt\_4.xml} \\ \hline
\texttt{012a} & Menuisier-charpentier (Nedjar) de Tanger (Province de Tanger - Maroc) & \texttt{s1t2\_chapt\_5.xml} \\ \hline
\texttt{013a} & Tailleur d'habits de Paris (Seine - France) & \texttt{s1t2\_chapt\_6.xml} \\ \hline
\texttt{014a} & Compositeur-typographe de Bruxelles (Brabant - Belgique) & \texttt{s1t2\_chapt\_7.xml} \\ \hline
\texttt{015a} & Décapeur d'outils en acier de la fabrique d'Hérimoncourt (Doubs - France) & \texttt{s1t2\_chapt\_8.xml} \\ \hline
\texttt{016a} & Monteur d'outils en acier de la fabrique d'Hérimoncourt (Doubs - France) & \texttt{s1t2\_chapt\_9.xml} \\ \hline
\texttt{017a} & Porteur d'eau de Paris (Seine - France) & \texttt{s1t2\_chapt\_10.xml} \\ \hline
\texttt{018a} & Paysans en communauté et en polygamie de Bousrah (Esky Cham), dans le pays de Haouran (Syrie - Empire Ottoman) & \texttt{s1t2\_chapt\_11.xml} \\ \hline
\texttt{019a} & Débardeur et piocheur de craie de la banlieue de Paris (Seine - France) & \texttt{s1t2\_chapt\_12.xml} \\ \hline
\texttt{411a} & Tables alphabétique et analytique des matières contenues dans ce tome second & \texttt{s1t2\_chapt\_13.xml} \\ \hline
\texttt{412a} & Errata & \texttt{s1t2\_chapt\_14.xml} \\ \hline
\texttt{413a} & Tables des matières contenues dans ce tome second & \texttt{s1t2\_chapt\_15.xml} \\ \hline
\end{longtable}
\end{center}

\subsection{Série 1, vol. 3 (1861).}
\label{mappings1t3}

URL sur \ia{} : 

\url{https://archive.org/details/lesouvriersdesde03sociuoft}.

\begin{center}
\begin{longtable}{ | c | p{9.5cm} | c | }
\hline
Id & Intitulé & Fichier \\ \hline
\texttt{414a} & Page de titre & \texttt{s1t3\_chapt\_1.xml} \\ \hline
\texttt{415a} & Avertissement & \texttt{s1t3\_chapt\_2.xml} \\ \hline
\texttt{416a} & Rapport. Société d'économie sociale. Travaux de 1859-1860 & \texttt{s1t3\_chapt\_3.xml} \\ \hline
\texttt{417a} & Liste générale des membres de la Société internationale des études pratiques d'économie sociale & \texttt{s1t3\_chapt\_4.xml} \\ \hline
\texttt{020a} & Brodeuses des Vosges (Vosges - France) & \texttt{s1t3\_chapt\_5.xml} \\ \hline
\texttt{021a} & Paysan et savonnier de la Basse-Provence (Bouches-du-Rhône - France) & \texttt{s1t3\_chapt\_6.xml} \\ \hline
\texttt{022a} & Mineur des Placers du comté de Mariposa (Californie - États-Unis) & \texttt{s1t3\_chapt\_7.xml} \\ \hline
\texttt{023a} & Manœuvre-vigneron de l'Aunis (Charente-inférieure - France) & \texttt{s1t3\_chapt\_8.xml} \\ \hline
\texttt{024a} & Lingère de Lille (Nord - France) & \texttt{s1t3\_chapt\_9.xml} \\ \hline
\texttt{025a} & Parfumeur de Tunis (Régence de Tunis - Afrique) du bazar appelé : El Attharin-el-kebar (les grands parfumeurs) & \texttt{s1t3\_chapt\_10.xml} \\ \hline
\texttt{026a} & Instituteur primaire d'une commune rurale de la Normandie (Eure - France) & \texttt{s1t3\_chapt\_11.xml} \\ \hline
\texttt{027a} & Manœuvre à famille nombreuse de Paris (Seine - France) & \texttt{s1t3\_chapt\_12.xml} \\ \hline
\texttt{028a} & Fondeur de plomb des Alpes Apuanes (Toscane - Italie) & \texttt{s1t3\_chapt\_13.xml} \\ \hline
\texttt{418a} & Tables alphabétique et analytique des matières contenues dans ce tome troisième & \texttt{s1t3\_chapt\_14.xml} \\ \hline
\texttt{419a} & Errata de ce tome troisième & \texttt{s1t3\_chapt\_15.xml} \\ \hline
\texttt{420a} & Tables des matières contenues dans ce tome troisième & \texttt{s1t3\_chapt\_16.xml} \\ \hline
\end{longtable}
\end{center}

\subsection{Série 1, vol. 4 (1862).}
\label{mappings1t4}

URL sur \ia{} : 

\url{https://archive.org/details/lesouvriersdesde04sociuoft}.

\begin{center}
\begin{longtable}{ | c | p{9.5cm} | c | }
\hline
Id & Intitulé & Fichier \\ \hline
\texttt{421a} & Page de titre & \texttt{s1t4\_chapt\_1.xml} \\ \hline
\texttt{422a} & Explications des signes de renvoi et des abréviations. & \texttt{s1t4\_chapt\_2.xml} \\ \hline
\texttt{423a} & Avertissement & \texttt{s1t4\_chapt\_3.xml} \\ \hline
\texttt{424a} & Rapport. Société d'économie sociale. Travaux de 1860-1861 & \texttt{s1t4\_chapt\_4.xml} \\ \hline
\texttt{425a} & Instruction. Méthode d'observation des monographies de famille propre à l'ouvrage intitulé Les ouvriers européens & \texttt{s1t4\_chapt\_5.xml} \\ \hline
\texttt{426a} & Histoire de la famille. Prix fondé par M. le baron de Damas. Par la société d'économie sociale & \texttt{s1t4\_chapt\_6.xml} \\ \hline
\texttt{029a} & Paysan d'un village à banlieue morcelée du Laonnais (Aisne - France) & \texttt{s1t4\_chapt\_7.xml} \\ \hline
\texttt{030a} & Paysans en communauté du Ning-Po-Fou (province de Tché-Kian - Chine) & \texttt{s1t4\_chapt\_8.xml} \\ \hline
\texttt{031a} & Mulâtre affranchi de l'Ile de la Réunion (Océan Indien) & \texttt{s1t4\_chapt\_9.xml} \\ \hline
\texttt{032a} & Manœuvre-vigneron de la Basse-Bourgogne (Yonne - France) & \texttt{s1t4\_chapt\_10.xml} \\ \hline
\texttt{033a} & Compositeur-typographe de Paris (Seine - France) & \texttt{s1t4\_chapt\_11.xml} \\ \hline
\texttt{034a} & Auvergnat brocanteur en boutique à Paris (Seine - France) & \texttt{s1t4\_chapt\_12.xml} \\ \hline
\texttt{035a} & Mineur de la Maremme de Toscane (Toscane - Italie) & \texttt{s1t4\_chapt\_13.xml} \\ \hline
\texttt{036a} & Tisserand des Vosges (Haut-Rhin - France) & \texttt{s1t4\_chapt\_14.xml} \\ \hline
\texttt{037a} & Pêcheur côtier, maître de barques, de Marken (Hollande septentrionale - Pays-Bas) & \texttt{s1t4\_chapt\_15.xml} \\ \hline
\texttt{427a} & Société internationale des études pratiques d'économie sociale. Officiers composants les conseils d'administration et de surveillance pour la session 1863-1864. & \texttt{s1t4\_chapt\_16.xml} \\ \hline
\texttt{428a} & Liste générale des membres de la Société internationale des études pratiques d'économie sociale au 1er août 1863 & \texttt{s1t4\_chapt\_17.xml} \\ \hline
\texttt{429a} & Tables alphabétique et analytique des matières contenues dans ce tome quatrième & \texttt{s1t4\_chapt\_18.xml} \\ \hline
\texttt{430a} & Errata de ce tome quatrième & \texttt{s1t4\_chapt\_19.xml} \\ \hline
\texttt{431a} & Tables des matières contenues dans ce tome quatrième & \texttt{s1t4\_chapt\_20.xml} \\ \hline
\end{longtable}
\end{center}

\subsection{Série 1, vol. 5 [1875, 1883, 1884] (1885).}
\label{mappings1t5}

URL sur \ia{} : 

\url{https://archive.org/details/lesouvriersdesde05sociuoft}

\begin{center}
\begin{longtable}{ | c | p{9.5cm} | c | }
\hline
Id & Intitulé & Fichier \\ \hline
\texttt{432a} & Page de titre & \texttt{s1t5\_chapt\_1.xml} \\ \hline
\texttt{433a} & Oeuvres de F. Le Play & \texttt{s1t5\_chapt\_2.xml} \\ \hline
\texttt{434a} & Sommaire. Monographies de familles publiées dans ce volume & \texttt{s1t5\_chapt\_3.xml} \\ \hline
\texttt{435a} & Avertissement & \texttt{s1t5\_chapt\_4.xml} \\ \hline
\texttt{436a} & Explications des signes de renvoi et des abréviations employés dans le cours de cet ouvrage & \texttt{s1t5\_chapt\_5.xml} \\ \hline
\texttt{038a} & Fermiers à communauté taisible du Nivernais (Saône-et-Loire - France) & \texttt{s1t5\_chapt\_6.xml} \\ \hline
\texttt{039a} & Paysan de Saint-Irénée (Bas-Canada - Amérique du Nord) & \texttt{s1t5\_chapt\_7.xml} \\ \hline
\texttt{040a} & L'Ouvrier éventailliste de Sainte-Geneviève (Oise - France) & \texttt{s1t5\_chapt\_8.xml} \\ \hline
\texttt{041a} & Ouvrier cordonnier de Malakoff (Seine - France) & \texttt{s1t5\_chapt\_9.xml} \\ \hline
\texttt{041b} & Précis d'une monographie ayant pour objet un chiffonnier instable et, par alternance, mégissier fumiste et brossier de Paris (France - Seine) & \textit{In supra.} \\ \hline
\texttt{042a} & Serrurier-forgeron de Paris (Seine - France) & \texttt{s1t5\_chapt\_10.xml} \\ \hline
\texttt{042b} & Précis d'une monographie ayant pour objet le monteur en bronze de Paris & \textit{In supra.} \\ \hline
\texttt{043a} & Brigadier de la Garde républicaine de Paris (Seine - France) & \texttt{s1t5\_chapt\_11.xml} \\ \hline
\texttt{044a} & Paysan-résinier de Lévignacq (Landes - France) & \texttt{s1t5\_chapt\_12.xml} \\ \hline
\texttt{045a} & Bûcheron usager de l'ancien Comté de Dabo (Lorraine allemande) & \texttt{s1t5\_chapt\_13.xml} \\ \hline
\texttt{046a} & Paysans en communauté et colporteurs émigrants de Tabou-Douchd-El-Baar (Grande Kabylie - Province d'Alger) & \texttt{s1t5\_chapt\_14.xml} \\ \hline
\texttt{437a} & Société d'économie sociale. Conseil d'administration pour l'année 1885 & \texttt{s1t5\_chapt\_15.xml} \\ \hline
\texttt{438a} &  Liste générale des membres de la Société d'économie sociale au 15 mars 1885 & \texttt{s1t5\_chapt\_16.xml} \\ \hline
\texttt{439a} & Tables alphabétique et analytique des matières contenues dans ce tome cinquième & \texttt{s1t5\_chapt\_17.xml} \\ \hline
\texttt{440a} & Tables des matières contenues dans ce tome cinquième & \texttt{s1t5\_chapt\_18.xml} \\ \hline
\end{longtable}
\end{center}

\subsection{Série 2, vol. 1 [1885-1887] (1887).}
\label{mappings2t1}

URL sur \ia{} : 


\url{https://archive.org/details/s2lesouvriersdes01sociuoft}

\begin{center}
\begin{longtable}{ | c | p{9.5cm} | c | }
\hline
Id & Intitulé & Fichier \\ \hline
\texttt{441a} & Page de titre & \texttt{s2t1\_chapt\_1.xml} \\ \hline
\texttt{442a} & Sommaire des monographies de familles publiées dans ce volume & \texttt{s2t1\_chapt\_2.xml} \\ \hline
\texttt{443a} & Avertissement sur ce premier volume, deuxième série des Ouvriers des deux mondes & \texttt{s2t1\_chapt\_3.xml} \\ \hline
\texttt{047a} & Paysan-paludier du Bourg de Batz (Loire-Inférieure - France) & \texttt{s2t1\_chapt\_4.xml} \\ \hline
\texttt{048a} & Bordiers émancipés en communauté rurale de la Grande-Russie & \texttt{s2t1\_chapt\_5.xml} \\ \hline
\texttt{048b} & Précis d'une monographie de l'armurier des manufactures impériales de Toula (Grande-Russie) & \texttt{s2t1\_chapt\_6.xml} \\ \hline
\texttt{049a} & Charron des forges et fonderies de Montataire (Oise - France) & \texttt{s2t1\_chapt\_16.xml} \\ \hline
\texttt{050a} & Faienciers de Nevers (Nièvre - France) & \texttt{s2t1\_chapt\_17.xml} \\ \hline
\texttt{051a} & Cultivateur-maraicher de Deuil (Seine-et-Oise - France) & \texttt{s2t1\_chapt\_18.xml} \\ \hline
\texttt{052a} & Pêcheur-côtier, maître de barque, de Martigues (Bouches-du-Rhône - France) & \texttt{s2t1\_chapt\_19.xml} \\ \hline
\texttt{053a} & Métayer à famille-souche du pays d'Horte (Landes - France) & \texttt{s2t1\_chapt\_20.xml} \\ \hline
\texttt{054a} & Arabes pasteurs nomades de la tribu des Larbas (Région saharienne de l'Algérie) & \texttt{s2t1\_chapt\_21.xml} \\ \hline
\texttt{055a} & Gantier de Grenoble (Isère - France) & \texttt{s2t1\_chapt\_22.xml} \\ \hline
\texttt{444a} & Tables alphabétique et analytique des matières contenues dans le présent volume & \texttt{s2t1\_chapt\_24.xml} \\ \hline
\texttt{445a} & Table des matières dans ce tome premier (deuxième série) & \texttt{s2t1\_chapt\_25.xml} \\ \hline
\end{longtable}
\end{center}

\subsection{Série 2, vol. 2 [1887-1889] (1890).}
\label{mappings2t2}

URL sur \ia{} : 

\url{https://archive.org/details/s2lesouvriersdes02sociuoft}.

\begin{center}
\begin{longtable}{ | c | p{9.5cm} | c | }
\hline
Id & Intitulé & Fichier \\ \hline
\texttt{446a} & Page de titre & \texttt{s2t2\_chapt\_1.xml} \\ \hline
\texttt{447a} & Page de titre & \texttt{s2t2\_chapt\_2.xml} \\ \hline
\texttt{448a} & Sommaire des monographies de familles publiées dans ce volume & \texttt{s2t2\_chapt\_3.xml} \\ \hline
\texttt{449a} & Avertissement sur ce deuxième tome de la deuxième série des Ouvriers des deux mondes & \texttt{s2t2\_chapt\_4.xml} \\ \hline
\texttt{056a} & Tourneur-mécanicien des usines de la Société Cockerill, de Seraing (Belgique) & \texttt{s2t2\_chapt\_5.xml} \\ \hline
\texttt{057a} & Bordier (Fellah) berbère de la Grande-Kabylie (Province d'Alger) & \texttt{s2t2\_chapt\_6.xml} \\ \hline
\texttt{057b} & Précis d'une monographie du paysan colon du Sahel (Algérie) & \texttt{s2t2\_chapt\_7.xml} \\ \hline
\texttt{058a} & Pêcheur côtier d'Heyst (Flandre occidentale - Belgique) & \texttt{s2t2\_chapt\_8.xml} \\ \hline
\texttt{058b} & Précis d'une monographie du pêcheur côtier, maître de barque, d'Étretat (Seine-Inférieure - France) & \texttt{s2t2\_chapt\_9.xml} \\ \hline
\texttt{059a} & Paysan-métayer de la Basse Provence (Bouches-du-Rhône - France) & \texttt{s2t2\_chapt\_10.xml} \\ \hline
\texttt{059b} & Précis d'une monographie du paysan et maçon émigrant de la Marche (Creuse - France) & \texttt{s2t2\_chapt\_11.xml} \\ \hline
\texttt{060a} & Mineur silésien du bassin houiller de la Ruhr (Prusse rhénane - Allemagne) & \texttt{s2t2\_chapt\_12.xml} \\ \hline
\texttt{061a} & Mineur des soufrières de Lercara (Province de Palerme - Sicile & \texttt{s2t2\_chapt\_13.xml} \\ \hline
\texttt{062a} & Tailleur de Silex et vigneron de l'Orléanais (Loir-et-Cher - France) & \texttt{s2t2\_chapt\_14.xml} \\ \hline
\texttt{063a} & Vigneron précariste et métayer de Valmontone (Province de Rome - Italie) & \texttt{s2t2\_chapt\_15.xml} \\ \hline
\texttt{064a} & Paysans corses en communauté, porchers-bergers des montagnes de Bastelica & \texttt{s2t2\_chapt\_16.xml} \\ \hline
\texttt{450a} & Tables alphabétique et analytique des matières contenues dans le présent tome, avec index explicatif des mots employés dans un sens propre à l'économie sociale & \texttt{s2t2\_chapt\_17-1.xml} \\ \hline
\texttt{450b} & Table des matières dans ce tome deuxième (deuxième série) & \texttt{s2t2\_chapt\_17-2.xml} \\ \hline
\end{longtable}
\end{center}

\subsection{Série 2, vol. 3 [1890-1892] (1892).}
\label{mappings2t3}

URL sur \ia{} : 

\url{https://archive.org/details/s2lesouvriersdes03sociuoft}.

\begin{center}
\begin{longtable}{ | c | p{9.5cm} | c | }
\hline
Id & Intitulé & Fichier \\ \hline
\texttt{451a} & Page de titre & \texttt{s2t3\_chapt\_1.xml} \\ \hline
\texttt{452a} & Page de titre & \texttt{s2t3\_chapt\_2.xml} \\ \hline
\texttt{453a} & Sommaire des monographies de familles publiées dans ce volume & \texttt{s2t3\_chapt\_3.xml} \\ \hline
\texttt{454a} & Avertissement sur ce troisième tome de la deuxième série des Ouvriers des deux mondes & \texttt{s2t3\_chapt\_4.xml} \\ \hline
\texttt{065a} & Métayers en communauté du Confolentais (Charente - France) & \texttt{s2t3\_chapt\_5.xml} \\ \hline
\texttt{066a} & Vignerons de Ribeauvillé (Alsace) & \texttt{s2t3\_chapt\_6.xml} \\ \hline
\texttt{066b} & Précis d'une monographie du pêcheur-côtier du Finmark (Laponie - Norvège) & \texttt{s2t3\_chapt\_7.xml} \\ \hline
\texttt{066c} & Précis d'une monographie d'un tisserand d'Hilversum (Hollande septentrionale - Pays-Bas) & \texttt{s2t3\_chapt\_14.xml} \\ \hline
\texttt{067a} & Tisserand de la fabrique collective de Gand (Flandre orientale - Belgique & \texttt{s2t3\_chapt\_28.xml} \\ \hline
\texttt{068a} & Paysan agriculteur de Torremaggiore (Province de Foggia - Italie) & \texttt{s2t3\_chapt\_29.xml} \\ \hline
\texttt{069a} & Tanneur de Nottingham (Angleterre) & \texttt{s2t3\_chapt\_30.xml} \\ \hline
\texttt{070a} & Charpentier indépendant de Paris (Seine - France) & \texttt{s2t3\_chapt\_31.xml} \\ \hline
\texttt{071a} & Conducteur-typographe de l'agglomération bruxelloise (Brabant - Belgique) & \texttt{s2t3\_chapt\_32.xml} \\ \hline
\texttt{072a} & Coutelier de la fabrique collective de Gembloux (Province de Namur - Belgique) & \texttt{s2t3\_chapt\_33.xml} \\ \hline
\texttt{455a} & Tables alphabétique et analytique des matières contenues dans le présent tome, avec index explicatif des mots employés dans un sens propre à l'économie sociale & \texttt{s2t3\_chapt\_34.xml} \\ \hline
\texttt{456a} & Table des matières dans ce tome troisième (deuxième série) & \texttt{s2t3\_chapt\_35.xml} \\ \hline
\end{longtable}
\end{center}

\subsection{Série 2, vol. 4 [1892-1895] (1895).}
\label{mappings2t4}

URL sur \ia{} : 

\url{https://archive.org/details/s2lesouvriersdes04sociuoft}.

\begin{center}
\begin{longtable}{ | c | p{9.5cm} | c | }
\hline
Id & Intitulé & Fichier \\ \hline
\texttt{457a} & Page de titre & \texttt{s2t4\_chapt\_1.xml} \\ \hline
\texttt{458a} & Page de titre & \texttt{s2t4\_chapt\_2.xml} \\ \hline
\texttt{459a} & Sommaire des monographies de familles publiées dans ce volume & \texttt{s2t4\_chapt\_3.xml} \\ \hline
\texttt{460a} & Avertissement sur ce quatrième volume de la deuxième série & \texttt{s2t4\_chapt\_4.xml} \\ \hline
\texttt{073a} & Ajusteur-surveillant de l'usine de Guise (Aisne - France) & \texttt{s2t4\_chapt\_5.xml} \\ \hline
\texttt{074a} & Ébéniste parisien de haut luxe (Seine - France) & \texttt{s2t4\_chapt\_6.xml} \\ \hline
\texttt{075a} & Métayer de l'Ouest du Texas (États-Unis d'Amérique) & \texttt{s2t4\_chapt\_7.xml} \\ \hline
\texttt{076a} & Ouvrière mouleuse en cartonnage d'une fabrique collective de jouets parisiens (Seine - France) & \texttt{s2t4\_chapt\_8.xml} \\ \hline
\texttt{077a} & Savetier de Bâle (Suisse) & \texttt{s2t4\_chapt\_9.xml} \\ \hline
\texttt{078a} & Ouvrier-employé de la fabrique coopérative de papiers d'Angoulême (Charente - France) & \texttt{s2t4\_chapt\_10.xml} \\ \hline
\texttt{079a} & Tisseur de San Leucio (Province de Caserte - Italie) & \texttt{s2t4\_chapt\_11.xml} \\ \hline
\texttt{080a} & Fermiers montagnards du Haut-Forez (Loire - France) & \texttt{s2t4\_chapt\_12.xml} \\ \hline
\texttt{081a} & Allumeur de réverbères de Nancy (Meurthe-et-Moselle - France) & \texttt{s2t4\_chapt\_13.xml} \\ \hline
\texttt{461a} & Tables alphabétique et analytique des matières contenues dans le présent tome, avec index explicatif des mots employés dans un sens propre à l'économie sociale & \texttt{s2t4\_chapt\_14.xml} \\ \hline
\texttt{462a} & Table des matières contenues dans ce tome quatrième (deuxième série) & \texttt{s2t4\_chapt\_15.xml} \\ \hline
\end{longtable}
\end{center}

\subsection{Série 2, vol. 5 [1895-1899] (1899).}
\label{mappings2t5}

URL sur \ia{} : 

\url{https://archive.org/details/2serlesouvriersde05sociuoft}.

\begin{center}
\begin{longtable}{ | c | p{9.5cm} | c | }
\hline
Id & Intitulé & Fichier \\ \hline
\texttt{463a} & Page de titre & \texttt{s2t5\_chapt\_1.xml} \\ \hline
\texttt{464a} & Société d'économie sociale [nota : liste des publications] & \texttt{s2t5\_chapt\_2.xml} \\ \hline
\texttt{465a} & Sommaire des monographies de familles publiées dans ce volume & \texttt{s2t5\_chapt\_3.xml} \\ \hline
\texttt{466a} & Avertissement sur ce cinquième tome de la deuxième série & \texttt{s2t5\_chapt\_4.xml} \\ \hline
\texttt{082a} & Ouvrier garnisseur de canons de fusils de la fabrique collective d'armes à feu de Liège (Liège - Belgique) & \texttt{s2t5\_chapt\_5.xml} \\ \hline
\texttt{083a} & Fileur en peigné et régleur de métier de la Manufacture du Val-des-Bois (Marne - France) & \texttt{s2t5\_chapt\_6.xml} \\ \hline
\texttt{084a} & Cordonnier d'Iseghem (Flandre Occidentale - Belgique) & \texttt{s2t5\_chapt\_7.xml} \\ \hline
\texttt{085a} & Paysan métayer (Contadino mezzajuolo) de Roccasancasciano (Romagne Toscane - Italie) & \texttt{s2t5\_chapt\_8.xml} \\ \hline
\texttt{085b} & Précis d'une monographie d'un ouvrier agriculteur de la campagne de Ravenne (Romagne - Italie) & \texttt{s2t5\_chapt\_9.xml} \\ \hline
\texttt{086a} & Mineur des mines de houille du Pas-de-Calais (France) & \texttt{s2t5\_chapt\_10.xml} \\ \hline
\texttt{087a} & Agriculteur du Pas-de-Calais (France) & \texttt{s2t5\_chapt\_11.xml} \\ \hline
\texttt{088a} & Serrurier-forgeron du quartier de Picpus, à Paris (France) & \texttt{s2t5\_chapt\_12.xml} \\ \hline
\texttt{088b} & Précis d'une monographie du serrurier poseur de persiennes en fer de Paris & \texttt{s2t5\_chapt\_13.xml} \\ \hline
\texttt{089a} & Piqueur sociétaire de la Mine aux Mineurs de Monthieux (Loire - France) & \texttt{s2t5\_chapt\_14.xml} \\ \hline
\texttt{090a} & Petit fonctionnaire de Pnom-Penh (Cambodge) & \texttt{s2t5\_chapt\_15.xml} \\ \hline
\texttt{090b} & Précis d'une monographie d'un manœuvre-coolie de Pnom-Penh (Cambodge) & \texttt{s2t5\_chapt\_16.xml} \\ \hline
\texttt{091a} & Métayer de Corrèze (Bas Limousin - France) & \texttt{s2t5\_chapt\_17.xml} \\ \hline
\texttt{467a} & Tables alphabétique et analytique des matières contenues dans le présent tome, avec index explicatif des mots employés dans un sens propre à l'économie sociale & \texttt{s2t5\_chapt\_18-1.xml} \\ \hline
\texttt{467b} & Table des matières dans ce tome cinquième & \texttt{s2t5\_chapt\_18-2.xml} \\ \hline
\end{longtable}
\end{center}

\subsection{Série 3, vol. 1 [1900-1904] (1904).}
\label{mappings3t1}

URL sur \ia{} : 

\url{https://archive.org/details/lesouvriersdesde0108sociuoft/}.

\begin{center}
\begin{longtable}{ | c | p{9.5cm} | c | }
\hline
Id & Intitulé & Fichier \\ \hline
\texttt{468a} & [Fichier sans texte] & \texttt{s3t1\_chapt\_1.xml} \\ \hline
\texttt{472a} & La société générale des papeteries du Limousins & \texttt{s3t1\_chapt\_2.xml} \\ \hline
\texttt{092a} & Fermier normand de Jersey & \texttt{s3t1\_chapt\_3.xml} \\ \hline
\texttt{092b} & Précis d'une monographie d'un pêcheur-côtier, maître de barques, de l'archipel Chusan (Chine) & \texttt{s3t1\_chapt\_4.xml} \\ \hline
\texttt{093a} & Aveugle accordeur de pianos de Levallois-Perret (Seine - France) & \texttt{s3t1\_chapt\_5.xml} \\ \hline
\texttt{094a} & Bouilleur de cru du Bas-Pays de Cognac (Charente - France) & \texttt{s3t1\_chapt\_6.xml} \\ \hline
\texttt{095a} & Mineur du bassin houiller du Couchant de Mons (Borinage - Belgique) & \texttt{s3t1\_chapt\_7.xml} \\ \hline
\texttt{096a} & Fellah de Karnak (Haute-Egypte) & \texttt{s3t1\_chapt\_8.xml} \\ \hline
\texttt{097a} & Tisserand d'usine de Gladbach (Prusse rhénane) & \texttt{s3t1\_chapt\_9.xml} \\ \hline
\texttt{098a} & Décoreuse de porcelaine de Limoges (Haute-Vienne - France) & \texttt{s3t1\_chapt\_10.xml} \\ \hline
\texttt{099a} & Cantonnier-poseur de voie du chemin de fer du Nord à Paris & \texttt{s3t1\_chapt\_11.xml} \\ \hline
\end{longtable}
\end{center}

\subsection{Série 3, vol. 2 [1904-1908] (1908).}
\label{mappings3t2}

URL sur \ia{} : 

\url{https://archive.org/details/lesouvriersdesde916sociuoft/}.

\begin{center}
\begin{longtable}{ | c | p{9.5cm} | c | }
\hline
Id & Intitulé & Fichier \\ \hline
\texttt{469a} & Page de titre & \texttt{s3t2\_chapt\_1.xml} \\ \hline
\texttt{100a} & Cordonnier de la fabrique collective de Binche (Province de Hainaut - Belgique) & \texttt{s3t2\_chapt\_2.xml} \\ \hline
\texttt{101a} & Compositeur typographe de Québec (Canada - Amérique du Nord) & \texttt{s3t2\_chapt\_3.xml} \\ \hline
\texttt{102a} & Ardoisier du bassin d'Herbeumont (Belgique) & \texttt{s3t2\_chapt\_4.xml} \\ \hline
\texttt{103a} & Commis à l'administration centrale des chemins de fer de l'État belge (Schaerbeek-Bruxelles - Belgique) & \texttt{s3t2\_chapt\_5.xml} \\ \hline
\texttt{104a} & Teinturier de ganterie et gantières de Saint-Junien (Haute-Vienne - France) & \texttt{s3t2\_chapt\_6.xml} \\ \hline
\texttt{105a} & Jardinier-plantier de Gasseras (Commune de Montauban, Tarn-et-Garonne - France) & \texttt{s3t2\_chapt\_7.xml} \\ \hline
\texttt{106a} & Corsetière du Raincy (banlieue de Paris - France) & \texttt{s3t2\_chapt\_8.xml} \\ \hline
\texttt{107a} & Étameur sur fer-blanc des usines de Commentry (Allier - France) & \texttt{s3t2\_chapt\_9.xml} \\ \hline
\texttt{473a} & Usine hydraulique d'éclairage et de transport de force & \texttt{s3t2\_chapt\_10.xml} \\ \hline
\end{longtable}
\end{center}

\subsection{Série 3, vol. 3 [1908-1913] (1913).}
\label{mappings3t3}

URL sur \ia{} : 

\url{https://archive.org/details/lesouvriersdesde17sociuoft}.

\begin{center}
\begin{longtable}{ | c | p{9.5cm} | c | }
\hline
Id & Intitulé & Fichier \\ \hline
\texttt{470a} & Page de titre & \texttt{s3t3\_chapt\_1.xml} \\ \hline
\texttt{108a} & Paysan cultivateur du Ruvo di Puglia (Province de Bari - Italie, 1903) & \texttt{s3t3\_chapt\_2.xml} \\ \hline
\end{longtable}
\end{center}

\subsection{Série 3, vol. 3bis [1928-1930] (1930).}
\label{mappings3t3bis}

\begin{center}
\begin{longtable}{ | c | p{9.5cm} | c | }
\hline
Id & Intitulé & Fichier \\ \hline
\texttt{471a} & Page de titre & \texttt{s3t3-bis\_chapt\_1.xml} \\ \hline
\end{longtable}
\end{center}

\clearpage

\section{Structure logique}
\label{structure}

\begin{enumerate}[A.]
    \item \textbf{\textit{Titre.}}
    \item \textbf{\textit{Observations préliminaires définissant la condition des divers membres de la famille.}}
    \begin{enumerate}[I.]
        \item \textbf{\textit{Définition du lieu, de l'organisation industrielle et de la famille.}}
        \begin{enumerate}[]
            \item \textit{§ 1. État du sol, de l'industrie et de la population.}
            \item \textit{§ 2. État civil de la famille.}
            \item \textit{§ 3. Religion et habitudes morales.}
            \item \textit{§ 4. Hygiène et services de santé.}
            \item \textit{§ 5. Rang de la famille.}
        \end{enumerate}
        \item \textbf{\textit{Moyens d'existence de la famille.}}
        \begin{enumerate}[]
            \item \textit{§ 6. Propriétés.}
            \item \textit{§ 7. Subventions.}
            \item \textit{§ 8. Travaux et industries.}
        \end{enumerate}
        \item \textbf{\textit{Mode d'existence de la famille.}}
        \begin{enumerate}[]
            \item \textit{§ 9. Aliments et repas.}
            \item \textit{§ 10. Habitation, mobilier et vêtements.}
            \item \textit{§ 11. Récréations.}
        \end{enumerate}
        \item \textbf{\textit{Histoire de la famille.}}
        \begin{enumerate}[]
            \item \textit{§ 12. Phases principales de l'existence.}
            \item \textit{§ 13. M\oe{}urs et institutions assurant le bien-être physique et moral de la famille.}
        \end{enumerate}
        \item (\textbf{\textit{Budget domestique annuel}}\footnote{Cette section ne possède un titre que dans huit monographies}).
        \begin{enumerate}[]
            \item \textit{§ 14. Budget des recettes de l'année.}
            \item \textit{§ 15. Budget des dépenses de l'année.}
            \item \textit{Comptes annexés aux budgets} \footnotesize{(n° 1 à 84) puis} \textit{§ 16. Comptes annexés aux budgets.}
        \end{enumerate}
    \end{enumerate}
    \item \textbf{\textit{Notes}} \footnotesize{(n° 1 à 84) puis} \textbf{\textit{Éléments divers de la constitution sociale}}.
    \begin{enumerate}[]
            \item (A) \textbf{\textit{(titre du paragraphe)}} \footnotesize{(n° 1 à 84) puis} § 17. \textbf{\textit{(titre du paragraphe)}}.
            \item (B) \textbf{\textit{(titre du paragraphe)}} \footnotesize{(n° 1 à 84) puis} § 18. \textbf{\textit{(titre du paragraphe)}}.
            \item \textbf{\textit{etc.}}
        \end{enumerate}
\end{enumerate}

\clearpage

\section{Numérisations de \gb}\label{numgb}

Les \odm{} sont disponibles en version numérisée sur \gb. Nous listons ci-dessous les volumes en accès libre en fonction de leur lieu de conservation (les URL du domaine \url{hdl.handle.net} renvoient vers la \textit{HathiTrust Digital Library}). Les volumes sont numérotés ainsi : \textit{série (numéro du volume)}.

\subsection{Bibliothèque de l'université de Californie}
\begin{center}
\begin{tabular}{ | c | p{13cm} | }
\hline
Volume & URL \\ \hline
1 (1) & \url{https://books.google.fr/books?id=eN0WAAAAYAAJ} \\ \hline
1 (1) & \url{https://hdl.handle.net/2027/uc1.b4577103} \\ \hline
1 (2) & \url{https://books.google.fr/books?id=4GJwAAAAIAAJ} \\ \hline
1 (3) & \url{https://hdl.handle.net/2027/uc1.b4577105} \\ \hline
1 (4) & \url{https://books.google.fr/books?id=Y2hwAAAAIAAJ} \\ \hline
1 (4) & \url{https://hdl.handle.net/2027/uc1.b4577106} \\ \hline
\end{tabular}
\end{center}

\subsection{Bibliothèque nationale centrale de Florence}
\begin{center}
\begin{tabular}{ | c | p{13cm} | }
\hline
Volume & URL \\ \hline
1 (1) & \url{https://books.google.fr/books?id=rqZexB0-3V8C} \\ \hline
\end{tabular}
\end{center}

\subsection{Bibliothèque de l'université Harvard}
\begin{center}
\begin{tabular}{ | c | p{13cm} | }
\hline
Volume & URL \\ \hline
1 (1) & \url{https://hdl.handle.net/2027/hvd.32044079431714} \\ \hline
2 (2) & \url{https://hdl.handle.net/2027/hvd.32044018834879} \\ \hline
2 (5) & \url{https://hdl.handle.net/2027/hvd.32044100859230} \\ \hline
\end{tabular}
\end{center}

\subsection{Bibliothèque municipale de la ville de Lyon}
\begin{center}
\begin{tabular}{ | c | p{13cm} | }
\hline
Volume & URL \\ \hline
1 (1) & \url{https://books.google.fr/books?id=3r3hsfUlRYoC} \\ \hline
1 (2) & \url{https://books.google.fr/books?id=JSHDEpeveFgC} \\ \hline
1 (3) & \url{https://books.google.fr/books?id=3GWA_Kz5AW0C} \\ \hline
\end{tabular}
\end{center}

\subsection{\textit{Bayerische Staatsbibliothek} de Munich}
\begin{center}
\begin{tabular}{ | c | p{13cm} | }
\hline
Volume & URL \\ \hline
1 (1) & \url{https://books.google.fr/books?id=6I9LAAAAcAAJ} \\ \hline
1 (2) & \url{https://books.google.fr/books?id=apBLAAAAcAAJ} \\ \hline
1 (3) & \url{https://books.google.fr/books?id=5pBLAAAAcAAJ} \\ \hline
1 (4) & \url{https://books.google.fr/books?id=J5FLAAAAcAAJ} \\ \hline
\end{tabular}
\end{center}

\subsection{\textit{New York State College of Agriculture at Cornell University}}
\begin{center}
\begin{tabular}{ | c | p{13cm} | }
\hline
Volume & URL \\ \hline
1 (3) & \url{https://books.google.fr/books?id=10FBAAAAYAAJ} \\ \hline
\end{tabular}
\end{center}


\subsection{Bibliothèque de l'Université de Princeton}
\begin{center}
\begin{tabular}{ | c | p{13cm} | }
\hline
Volume & URL \\ \hline
1 (1) & \url{https://hdl.handle.net/2027/njp.32101064529090} \\ \hline
1 (2) & \url{https://hdl.handle.net/2027/njp.32101064529108} \\ \hline
1 (3) & \url{https://books.google.fr/books?id=fTooAAAAYAAJ} \\ \hline
1 (3) & \url{https://hdl.handle.net/2027/njp.32101064529116} \\ \hline
1 (4) & \url{https://hdl.handle.net/2027/njp.32101064529124} \\ \hline
1 (5) & \url{https://hdl.handle.net/2027/njp.32101064529132} \\ \hline
2 (1) & \url{https://hdl.handle.net/2027/njp.32101064529140} \\ \hline
2 (2) & \url{https://hdl.handle.net/2027/njp.32101064529157} \\ \hline
2 (4) & \url{https://hdl.handle.net/2027/njp.32101064529173} \\ \hline
2 (5) & \url{https://hdl.handle.net/2027/njp.32101064529181} \\ \hline
3 (1) & \url{https://hdl.handle.net/2027/njp.32101064529199} \\ \hline
3 (2) & \url{https://hdl.handle.net/2027/njp.32101064529207} \\ \hline
3 (3) & \url{https://hdl.handle.net/2027/njp.32101064529215} \\ \hline
\end{tabular}
\end{center}

\subsection{Université de Rome --- \textit{Instituto de filosofia del diritto}}
\begin{center}
\begin{tabular}{ | c | p{13cm} | }
\hline
Volume & URL \\ \hline
1 (2) & \url{https://books.google.fr/books?id=HjqApJuPx0gC} \\ \hline
\end{tabular}
\end{center}

\renewcommand{\thesection}{B.1}
\chapter{B. Feuille de route et typologie des erreurs}

\section{Feuille de route}
\label{ann:feuille_route}

Cette liste reprend le texte des \issues{} ouvertes dans le \gitlab{} des \odm{} au commencement du stage.

\begin{enumerate}
        \item \textit{Trier et renommer les fichiers} :
    \begin{itemize}
        \item  Identifier les fichiers qui correspondent aux monographies et ceux qui correspondent au paratexte ;
        \item  Donner un identifiant aux fichiers de monographie en fonction des identifiants déjà existants dans le fichier de référence ;
        \item  Créer un identifiant pour les fichiers de paratexte ;
        \item  Ajouter ces identifiants aux \texttt{@xml:id} de chaque fichier ;
        \item  Créer un \textit{mapping} de l’ensemble des fichiers sous la forme d’un CSV avec le nom du fichier et son identifiant ;
        \item  Créer un fichier \texttt{master.xml} contenant des renvois vers les autres fichiers grâce à des \texttt{<xi:includes>}.
    \end{itemize}

    \item \textit{Mettre à jour l’attribut \texttt{@url} dans la balise \texttt{<graphic>}} :
    \begin{itemize}
        \item  Les images des pages sont stockées localement sur Humanum, mais également en ligne sur Internet Archives.
        \item  Trouver l’url de chaque page de chaque volume sur \ia ;
        \item  Remplacer automatiquement le chemin local par l’url de l’image dans chaque fichier.
    \end{itemize}

    \item \textit{Tester la conformité du schéma} :
    \begin{itemize}
        \item  Écrire un script pour tester la validité des arbres XML de chaque fichier ;
        \item  Corriger les erreurs qui seraient signaler par ce script.
    \end{itemize}

    \item \textit{Intégrer les métadonnées des « enquêtés »} :
    \begin{itemize}
        \item  Créer un fichier référentiel de personnes (XML) à partir du fichier CSV de prosopographie ;
        \item  Ajouter aux paragraphes 2 des monographies ("§2. - Etat civil de la famille") des \texttt{@refs} au référentiel de personnes.
    \end{itemize}

    \item \textit{Intégrer modèle de citation dans les chapitres et créer un système de référence bibliographique} :
    \begin{itemize}
        \item  Pour chaque niveau de la structure d'une monographie  ;
        \item  Pour chaque chapitre (sans descendre dans les niveaux) ;
        \item  Par exemple en utilisant DTS.
    \end{itemize}

    \item \textit{Corrections des transcriptions} :
    \begin{itemize}
        \item  Paragraphe par paragraphe, implémenter une correction automatique des transcriptions.
    \end{itemize}

    \item \textit{Corriger le passage de source vers split} :
    \begin{itemize}
        \item  Corriger le script python de transformation des fichiers sources en une suite de fichiers XML TEI.
        \item  Éliminer les traces de teiCorpus.
    \end{itemize}

    \item \textit{Simplifier l'implémentation de la structure logique} :
    \begin{itemize}
        \item Aplatir la structure des monographies :
        \begin{itemize}
            \item  Est-il vraiment nécessaire d'utiliser les div enchâssées ? Une structure à plat avec des marqueurs signalant le début d'une nouvelle section ne suffirait-elle pas ?
            \item  Comment gérer l'articulation entre l'arbre principal (un arbre pour l'ensemble des monographies) et les sous-arbre (un sous-arbre par monographie) ? L'ensemble du corpus n'a pas de front/back mais chaque volume a un front et un back et chaque monographie a potentiellement un front et un back.
            \item  Avec quelles restrictions peut-on créer une structure similaire à celle d'un fichier \LaTeX{} avec ses imports ?
        \end{itemize}
    \end{itemize}
\end{enumerate}

\section{Relevé des erreurs dans la structure
logique}\label{ann:releve_erreurs}

Jean-Damien Généro, 8 juillet 2020\footnote{Reproduction d'une synthèse mise en ligne sur une issue \textit{GitLab}.}.

\begin{center}\rule{3in}{0.4pt}\end{center}

\subsection{Déficit dans la
transcription}\label{ann:deficit-transcr}

\begin{itemize}
\item
  \emph{Déficit partiel :} s1t2\_chapt\_11 et s1t3\_chapt\_10 (manque
  début §7), s1t2\_chapt\_4 (manquent quatre pages de la note A), \texttt{s2t3\_chapt\_14} (manquent vingt lignes).
\item
  \emph{Déficit majeur :} s1t4\_chapt\_8, 11, 15 et s1t5\_chapt\_13 et
  14 (seules les notes ont été transcrites, le reste se trouve dans des
  \texttt{\textless{}figure\textgreater{}}), s1t5\_chapt\_12
  (transcription à partir du §14, le reste dans des
  \texttt{\textless{}figure\textgreater{}}).
\end{itemize}

\begin{center}\rule{3in}{0.4pt}\end{center}

\subsection[Titres non-imprimés]{Titres manquants parce que non-imprimés dans les exemplaires
d'\ia}\label{ann:titre-non-imprimes}

\begin{itemize}
\item
  s1t1\_chapt\_12 : manque le titre de la s/section II.3 ``Mode
  d'existence\ldots{}'' ;
\item
  s2t1\_chapt\_5 : manque le titre de la s/section II.4 ``Histoire
  de\ldots{}'' ;
\item
  s2t1\_chapt\_20 : manque les titres des s/sections II.3 ``Mode
  d'existence\ldots{}'' et II.4 ``Histoire de\ldots{}'' (mais ce titre
  est repris dans l'intitulé du §12 qui suit) ;
\item
  s2t4\_chapt\_9 : manque les titres de la s/section II.1 ``Définition
  du lieu\ldots{}'' ;
\item
  s3t1\_chapt\_3 : manque le titre des s/sections II.2 ``Moyens
  d'existence\ldots{}'', II.3 ``Mode d'existence\ldots{}'' et II.4
  ``Histoire de\ldots{}'' (mais ce titre est repris dans l'intitulé du
  §12 qui suit) ;
\item
  s2t5\_chapt\_15 : manque le titre de la s/section II.4 ``Histoire
  de\ldots{}''.
\end{itemize}

\begin{center}\rule{3in}{0.4pt}\end{center}

\subsection{Structure allégée}\label{ann:structure-allegee}

Titres de s/s/section intégrés au début de paragraphe + certains titres
non-utilisés (spec. §16).

\begin{itemize}
\item
  s2t2\_chapt\_7 (précis) : pas de s/s/section §16 et de titre pour la
  partie Notes (composée d'une seule \emph{nota}) ;
\item
  s2t2\_chapt\_9 (précis) : pas de titre de section II ``Observations
  préliminaires\ldots{}'' et pas de s/s/section §16 ;
\item
  s2t2\_chapt\_11 (précis) : pas de titre de section II ``Observations
  préliminaires\ldots{}'', pas de titre pour les s/s/sections (sauf les
  paragraphes de budget §14 et §15), pas de s/s/section §16, pas de
  section Notes.
\item
  s2t3\_chapt\_7 (précis) : pas de titre de section II ``Observations
  préliminaires\ldots{}'', pas de s/s/section §16, les titres des
  s/s/section sont en début de paragraphe ;
\item
  s2t5\_chapt\_9 (précis) : dans la section II, tous les titres des
  s/s/sections sont en début de paragraphe (à l'exception des
  paragraphes de budget §14 et §15), il n'y a pas de s/s/section §16 ;
\item
  s2t5\_chapt\_16 (précis) : pas de s/s/section dans la s/section II.4
  ``Histoire de\ldots{}'', pas de s/s/section §16, la section Notes est
  occupée par la traduction d'un document.
\end{itemize}

\begin{center}\rule{3in}{0.4pt}\end{center}

\subsection{Pas de s/s/section §16 et pas de section
\textit{Notes}}\label{ann:no-notes-no-16}

\begin{itemize}
\item
  (voir \emph{supra} s2t2\_chapt\_11) ;
\item
  s2t1\_chapt\_6 et s2t5\_chapt\_12.
\end{itemize}

\begin{center}\rule{3in}{0.4pt}\end{center}

\subsection{Remarques et cas
particuliers}\label{ann:remarques-et-cas-particuliers}

\begin{itemize}
\item
  Dans s1t5\_chapt\_9, 10 et 11, une section des notes est intitulée
  ``Précis de monographie'' et organisée comme un précis (càd avec des
  sections et des titres). Néanmoins il ne s'agit pas de précis tels
  qu'on en trouve dans les séries 2 et 3, puisqu'ils ne constituent pas
  un ensemble logique indépendant de la monographie (il y a des
  paragraphes avant et des paragraphes après). Donc le ``précis'' est
  considéré comme une sub\_sub\_section et ses divisions comme des
  sub\_sub\_sub\_section.
\item
  s2t5\_chapt\_17 et 18 découpés chacun en deux fichiers (table
  analytique et table des matières).
\item
  s3t1\_chapt\_2 (``Société générale\ldots{}'') et s3t2\_chapt\_10
  (``Usine hydraulique\ldots{}'') sont des cas particuliers avec des
  structures totalement différentes des autres monographies. En plus de
  l'en-tête, le premier se divise en trois grandes sections :
  observations préliminaires (sorte d'intro), première partie, deuxième
  partie, troisième partie, conclusion et appendices ; le second est
  plus succinct avec l'en-tête, une grande section avec des s/sections
  numérotées et ensuite des appendices.
\item
  Dans s3t2\_chapt\_9 se trouve une partie ``Note sur l'état de la
  famille en 1905'' entre le §13 et §14, que j'ai structurée comme une
  s/s/section.
\item
  La s/section II.5, consacrée aux budgets (§14, §15, §16), n'est pas
  titrée sauf dans les précis de monographie s2t1\_chapt\_6,
  s2t2\_chapt\_9 et 11, s2t3\_chapt\_7 et 14, s2t5\_chapt\_9 et 13,
  s3t1\_chapt\_4 (le titre est ``Budget domestique annuel'').
\end{itemize}

\cleardoublepage

\listoffigures

\newpage
\thispagestyle{empty}
\mbox{}
\newpage

\tableofcontents

\newpage
\thispagestyle{empty}
\mbox{}
\newpage

\end{document}
