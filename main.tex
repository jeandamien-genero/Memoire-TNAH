\documentclass[a4paper,twoside,12pt]{book}
\usepackage[utf8]{inputenc}

\usepackage[french]{babel}

\usepackage{fontspec}

\usepackage[colorlinks=true,linkcolor=black,anchorcolor=black,citecolor=black,filecolor=black,menucolor=black,runcolor=black,urlcolor=black]{hyperref}

\usepackage{eurosym}

%Module d'usage facultatif permettant d'intégrer les tables, index, bibliographie, automatiquement à la table des matières
\usepackage{tocbibind}
%\usepackage{lscape}

\usepackage[margin=2.5cm]{geometry}
\usepackage{setspace}
\setlength{\parindent}{1cm}
\onehalfspacing

\usepackage[backend=biber, sorting=nyt, style=enc]{biblatex}
\usepackage[autostyle]{csquotes}

\addbibresource{mybibliography.bib}
\addbibresource{OD2M.bib}

%%%Pour les tableaux
%\usepackage{multirow}


%%% Les index
%\usepackage{makeidx}
%\usepackage{multind} %Ou splitidx
%\usepackage{index} %…
%\makeindex
%\makeindex{edition}
%\makeindex{texte}
%\newindex{etude}{adx}{and}{Index de l'étude}
%\newindex{edition}{bdx}{bnd}{Index de l'édition}



%%%Édition critique
%\usepackage{eledmac}
%\usepackage{eledpar}

%\footparagraph{A}

%\renewcommand{\Rlineflag}{D}

\hyphenation{}

\usepackage[backend=biber, sorting=nyt, style=enc]{biblatex}
%\addbibresource{Biblio/demo.bib}
%\nocite{lachin_i_2008}

\usepackage{enumerate,lettrine}

\newcommand\cad{c'est-à-dire}
\newcommand\ia{\textit{Internet Archive}}
\newcommand\odm{\textit{Ouvriers des deux mondes}}
\newcommand\tnah{\og Technologies numériques appliquées à l'histoire \fg{}}
\newcommand\timeus{\textit{Time~Us}}

\title{Valoriser le traitement automatique des données : Le cas des Ouvriers des deux mondes}
\author{Jean-Damien Généro}
\date{Juillet 2020}

\begin{document}

\frontmatter
\begin{titlepage}
\begin{center}

\bigskip

\begin{large}
\sc{\'Ecole nationale des chartes}
\end{large}
\begin{center}\rule{2cm}{0.02cm}\end{center}

\bigskip
\bigskip
\begin{Large}
\textbf{Jean-Damien Généro}\\
\end{Large}
\begin{normalsize} \textit{Licencié en histoire}\\
\textit{Diplômé de Master}\\
\end{normalsize}

\bigskip
\bigskip

\begin{Huge}
\textbf{\sc{Valoriser le traitement\\ automatique des données :}}\\
\end{Huge}
\bigskip
\bigskip
\begin{LARGE}
\textbf{\emph{Le cas des Ouvriers des deux mondes} }\\
\end{LARGE}

\bigskip

\begin{figure}[h]
    \centering
    \includegraphics[width=12cm]{img/couv.png}
    \label{fig:ill_couv}
\end{figure}

\begin{large}
\end{large}
\vfill

\begin{large}
Mémoire 
pour le diplôme de Master \\
\tnah \\
\medskip
2020

\bigskip

\includegraphics[width=1.25cm]{img/etalab-logo.png}

\begin{small}
{\textit{Contenu sous licence ouverte}}
\end{small}

\end{large}
\end{center}
\end{titlepage}
\clearpage
\thispagestyle{empty}
\cleardoublepage
\section*{Résumé}
\addcontentsline{toc}{chapter}{Résumé}
\markboth{Résumé}{}

\bigbreak

\textit{Le programme ANR \timeus{} s’intéresse aux ouvriers et aux ouvrières du textile de la fin du \textsc{xviii}\ieme ~siècle au début du \textsc{xx}\ieme ~siècle et rassemble pour cela une large documentation composée de documents manuscrits et d'imprimés. Au sein de ces derniers se trouvent les monographies de familles des} Ouvriers des deux mondes \textit{publiées de 1857 à 1930 par Frédéric Le Play (1806-1882) et la Société internationale des études pratiques d’économie sociale.}

\textit{Ce corpus de treize volumes, composé de 114 monographies, a été transcrit et structuré automatiquement au format XML-TEI par un programme utilisant les logiciels d’OCR Transkribus et Kraken, et le langage de programmation Python. Le présent mémoire se propose d'analyser les actions menées au cours d'un stage de fin d'études pour valoriser les résultats de cette structuration automatique, incluant son contrôle, la correction des erreurs et l’usage des humanités numériques pour implémenter un encodage scientifique permettant l’exploitation des données et des transcriptions par les chercheurs et les chercheuses.}

\bigbreak

\bigbreak

\bigbreak

\textbf{Mots-clés:} XML ; TEI ; Python ; traitement automatique des données ; transcription automatique ; édition numérique ; ALTO ; \ocr ; \kraken{} ; \transkribus{} ; \gitlab{} ; Frédéric Le Play ; \lodm{} ; enquêtes sociologiques ; monographies de familles ; \timeus{} ; Inria.

\bigbreak

\bigbreak

\bigbreak

\textbf{Informations bibliographiques:} Jean-Damien Généro, \textit{Valoriser le traitement automatique des données : Le cas des Ouvriers des deux mondes}, mémoire du Master \og Technologies numériques appliquées à l'histoire \fg{}, dir. Alix Chagué et Vincent Jolivet, École nationale des chartes, 2020.

\bigbreak

\bigbreak

\bigbreak

\textbf{Soutenance:} mémoire présenté et soutenu publiquement le 19 octobre 2020 à l'École nationale des chartes, devant un jury composé d'Édouard Vasseur, président, professeur d’Histoire des institutions, diplomatique et archivistique contemporaines, de Vincent Jolivet, responsable de la mission projets numériques et d'Alix Chagué, ingénieure de recherche et de développement de l’équipe ALMAnaCH d’Inria ; sanctionné par une mention Trés bien et la note de 18/20.

\bigbreak

\bigbreak

\textbf{Illustration de couverture:} Émile Savoy, \textit{Chocolatier de la fabrique de chocolat au lait F.-L. Cailler à Broc (canton de Fribourg, Suisse)}, dans \lodm, Paris, série 3, 1913, p. 325, \og La fabrique Cailler \fg.
\clearpage
\thispagestyle{empty}
\cleardoublepage
\chapter*{Liste des sigles et abréviations}
\addcontentsline{toc}{chapter}{Liste des sigles et abréviations}
\markboth{Liste des sigles abréviations}{} 

\begin{center}
\textit{Institutions}
\end{center} 

\begin{itemize}
    \item ALMAnaCH : \textit{Automatic Language Modelling and Analysis \& Computational Humanities} (Inria Paris)
    \item ANR : Agence nationale de la recherche
    \item CMH : Centre Maurice-Halbwachs (EHESS et ENS Paris)
    \item CNRS : Centre National de Recherche Scientifique
    \item EA : Équipe d'accueil
    \item EHESS : École des Hautes Études en Sciences Sociales
    \item ENS : École normale supérieure
    \item ICT : Identités, Cultures et Territoires (Université de Paris)
    \item Inria : Institut national de recherche en informatique et en automatique
    \item LARHRA : Laboratoire de Recherche Historique Rhône-Alpe (Lyon 2)
    \item READ : \textit{Recognitionand Enrichment of Archival Documents}
    \item TGIR : Très grande infrastructure de recherche
    \item TELEMMe : Temps, Espaces, Langages, Europe Méridionale-Méditerranée (Université d’Aix-Marseille)
    \item UMR : Unité mixte de recherche
\end{itemize}

\bigbreak

\begin{center}---

\bigbreak

\textit{Informatique et nouvelles technologies}
\end{center} 

\bigbreak

\begin{itemize}
    \item ALTO : \textit{Analyzed Layout and Text Object}
    \item CSV : \textit{Comma-separated values}
    \item GPU : \textit{Graphics processing unit}
    \item IETF : \textit{Internet Engineering Task Force}
    \item JPEG : \textit{Joint Photographic Experts Group}
    \item JP2 : \textit{JPEG 2000}
    \item \lse{} : \textit{Logical Structure Extraction from Les Ouvriers des Deux Mondes}
    \item OCR : \textit{Optical Character Recognition}
    \item PDF : \textit{Portable Document Format}
    \item RFC : \textit{Request for comments}
    \item TEI : \textit{Text Encoding Initiative}
    \item URI : \textit{Uniform Resource Identifier}
    \item XML : \textit{Extensible Markup Language}
\end{itemize}

\clearpage
\thispagestyle{empty}
\clearpage
\thispagestyle{empty}
\cleardoublepage
\section*{Préambule : un stage confiné ?}
\addcontentsline{toc}{chapter}{Préambule : un stage confiné ?}
\markboth{Préambule}{} 

Le stage qui a donné lieu au présent mémoire n'a pas commencé sous les meilleurs auspices. Des difficultés administratives et procédurales, puis le grand confinement et les difficultés plus grandes encore qui en ont résulté, ont failli avoir raison de lui. Il a fallu se battre pour que l'Université de Paris admette que la fermeture des établissements d'enseignement annoncée par le président de la République le 12 mars 2020, puis les mesures de confinement à partir du 17 mars, ne signifiaient en aucun cas une suspension de l'action de son administration et un report automatique et sans appel des procédures en court, dont celle concernant les stages. Je tiens donc en premier lieu à remercier Mesdames Alix Chagué et Manuela Martini, Messieurs Julien Cassefières et Thibault Clérice, qui m'ont apporté un tant soit peu de soutien et d'aide dans cette procédure extrêmement pénible, absolument anormale et pour laquelle j'ai dépensé une quantité d'énergie démesurée.

Contactée, la présidence de l'Université de Paris mit fin au marasme, apposa les cachets requis et présenta ses excuses ; le stage pu ainsi commencer, avec une semaine de retard sur la date prévue.

Intégrer une équipe d'ingénierie et de recherche tout en étant confiné chez soi n'est pas chose aisée. Les relations humaines que cet exercice suppose en temps normal ont été réduites au minimum et sont passées par des courriels, des échanges sur l'espace de discussion instantanée d'INRIA et, principalement, des visioconférences sur \textit{Zoom}.

Inconnue, immédiatement décevante et parfois pénible, cette façon de travailler a mis un certain temps à s'imposer pour moi et les premières semaines ont été difficiles. Début juin, c'est-à-dire à la moitié du stage, Alix Chagué me demanda de lui transmettre un bilan des compétences que j'avais acquises jusqu'ici. Elles étaient nombreuses du point de vue des savoir-faire, car, bon gré mal gré, je n'avais eu d'autre choix que de m'adapter à ce travail à distance et à remplir les missions qui m'avaient été assignées. Ce faisant, un des objectifs que je m'étais fixé pour ce stage fut très vite satisfait : mes savoir-faire se sont développés par l'apprentissage et la maîtrise de nouvelles techniques.

\textit{Quid} des savoir-être ? J'avais listé plusieurs points dans le bilan de mi-stage, sur un ton légèrement humoristique, et notamment saluer, faire une présentation et suivre une réunion en visioconférence. Un véritable savoir-être se cachait derrière cet humour désabusé :  ce stage m'a appris le travail à distance. Cela suppose beaucoup de choses. Outre une organisation et une discipline peut-être plus importantes que pour un travail de bureau, il s'agit d'être seul la majorité du temps. De ne pas avoir de collègue avec qui l'on peut échanger et progresser dans la résolution d'un problème. En un mot, il s'agit de travailler plus avec soi-même qu'avec les autres et donc de se \textit{débrouiller}.
\clearpage
\thispagestyle{empty}
\cleardoublepage
\section*{Introduction : le projet Time Us}
\addcontentsline{toc}{chapter}{Introduction : le projet Time Us}

\bigbreak

\og Reconstituer les rémunérations et les budgets temps des travailleuses et des travailleurs du textile dans quatre villes industrielles françaises (Lille, Paris, Lyon, Marseille) dans une perspective européenne et de longue durée \fg{}\footnote{Présentation du programme sur le site de l'ANR (\url{https://anr.fr/Projet-ANR-16-CE26-0018}, consulté le \today).} est l'objectif du programme ANR \textit{Time Us}. Son intitulé exact, \og Rémunérations et usages du temps des femmes et des hommes en France de la fin du \textsc{xvii}\ieme ~siècle au début du \textsc{xx}\ieme ~siècle \fg{}\footnote{Présentation du programme sur le site du LARHA (\url{http://larhra.ish-lyon.cnrs.fr/anr-time-us}, consulté le \today).} indique qu'il s'articule à la fois autour d'un temps long qui court depuis les premières manufactures de l'époque Moderne jusqu'à la fin de la révolution industrielle au tournant des \textsc{xix}\ieme  ~et~\textsc{xx}\ieme ~siècles, mais aussi sur le temps individuel et quotidien qui est celui d'un ouvrier ou d'une ouvrière au sein de ces grands mouvements historiques. Les relations entre la rémunération et le temps sont au c\oe{}ur des questionnements du programme. Cette attention portée à l'usage du temps se traduit dans la dénomination courante du programme, \og Time us \fg{}, abréviation de l'anglais \og \textit{time usage} \fg{}.

\textit{Time Us} se concentre principalement sur les femmes et plus encore sur les ouvrières du textile, industrie dans laquelle \og elles sont présentes dans toutes les phases du processus productif \fg{}\footnote{\cite[p. 1]{inria}.}. Le programme tend à combler le biais des genres dans l'historiographie du travail industriel en réalisant une opération de collecte et de traitement de la documentation manuscrite et imprimée relative à l'emploi et aux activités quotidiennes des femmes\footnote{\textit{Ibid}.}.  Ainsi, le moteur de \textit{Time us} est moins la production d'une réflexion scientifique autour du travail des femmes que la constitution d'un corpus documentaire sériel et prêt à être exploité par des chercheurs  d'horizons multiples. La pluridisciplinarité est en effet un aspect majeur du programme, qui souhaite une utilisation de ses données dans un maximum de champs de recherche des sciences humaines et sociales, notamment \og en  histoire économique et sociale, en histoire de la famille et du genre, en histoire des conflits du travail et de la culture des classes populaires \fg{}\footnote{\textit{Ibid}, p. 2.}.

La documentation est essentiellement constituée de documents manuscrits conservés par la Bibliothèque municipale de Lyon et les dépôts d'archives départementaux et municipaux lillois, parisien, lyonnais et marseillais. Très diverse, elle est issue d'organes officiels, à l'instar des conseils prud'homaux, des chambres de commerce ou encore des tribunaux de commerce, mais aussi d'organismes ou d'individus privés tels que les archives du tisseur lyonnais Pierre Charnier (1795-1857)\footnote{Les papiers du canut Pierre Charnier font partie du fonds \og Fernand Rude \fg{} de la Bibliothèque municipales de Lyon, nommé d'après l'historien qui les possédait avant leur versement (\url{https://www.bm-lyon.fr/collections-patrimoniales-et-specialisees/explorer-les-collections/article/fernand-rude}, consulté le \today).}, les registres comptables (1817-1821) et le dossier de faillite (1821-1822) de la filature parisienne Dupuis-Drouet\footnote{Archives de Paris, D 12U1, n° 375-376 (\url{http://archives.paris.fr/arkotheque/inventaires/ead_ir_consult.php?a=4&ref=FRAD075_000727}, consulté le \today).}, etc.

En sus de cette documentation manuscrite, le programme \textit{Time Us} s'appuie également sur trois grands corpus d'imprimés. Le premier se compose de neuf titres de la presse ouvrière lyonnaise, entièrement numérisés sur le site \textit{Numelyo}, espace digital de la Bibliothèque municipale de  Lyon\footnote{ Le corpus se trouve à l'adresse \url{https://collections.bm-lyon.fr/PER003}, (consulté le \today).}. L'intérêt du programme pour ce corpus porte sur les nombreux compte rendus d'audience du Conseil des prud'hommes qui s'y trouvent, mais aussi pour des reproductions ou des extraits de discours, des lettres ou des analyses économiques et sociales relayés par ces journaux. Quatre des titres sont publiés dans la première moitié des années 1830 (\textit{L'Écho de la fabrique}, \textit{L’Écho des travailleurs}, \textit{L'Indicateur} et \textit{La Tribune prolétaire}), les cinq restant dans les années 1840 (\textit{L’Écho des ouvriers}, \textit{L’Écho de la Fabrique de 1841}, \textit{L’Écho de l'industrie}, \textit{L'Avenir}, \textit{La Tribune Lyonnaise})\footnote{Présentation de ce corpus sur le wiki du programme : \url{http://timeusage.paris.inria.fr/mediawiki/index.php/Documentation_régionale_-_Presse_lyonnaise} (consulté le \today).}.

Un second corpus d'imprimés est constitué de monographies collectées sur le site \textit{Gallica} de la Bibliothèque nationale de France au format PDF. Très divers, on y trouve à la fois \textit{L'ouvrière} de Jules Simon (1861), un \textit{Dictionnaire général des tissus anciens et modernes} (1859-1863) ou encore un \textit{Traité complet sur la fabrication des étoffes de soie} (1859)\footnote{Liste complète sur le wiki du programme : \url{http://timeusage.paris.inria.fr/mediawiki/index.php/Aperçu_des_états\#Imprimés_divers} (consulté le \today).}. Il s'agit d'une base complémentaire au projet, qui doit permettre de contextualiser la base archivistique principale, notamment en permettant de suivre l'évolution du  vocabulaire afférent au textile sur la période étudiée.

Le troisième et dernier corpus d'imprimés est composé des treize volumes de la série des \textit{Ouvriers des deux mondes}.

\begin{center}
$\star$
\end{center} 

Initiées par le sociologue Frédéric Le Play (1806-1882), les \textit{Ouvriers des deux mondes} sont des enquêtes sociologiques --- couramment désignées comme les \og monographies \fg{} --- conduites par les membres de la Société internationale des études pratiques d’économie sociale\footnote{Actuelle Société d'économie et de sciences sociales, désormais abrégée en SESS.} de 1857 à 1928. Répartis en trois séries comptant un total de cent vingt-six monographies\footnote{\cite[p. 95]{lorry}.}, \textit{Les Ouvriers des deux mondes} sont la deuxième entreprise d'étude empirique de Le Play, après celle des \textit{Ouvriers européens} dont la première édition a lieu en 1855\footnote{\textit{Ibid}.}.

Dans sa préface du numéro spécial de la revue \textit{Les Études sociales} consacré aux monographies leplaysiennes, Alain Chenu relève la prégnance de l'assimilation de celles-ci à des \og mines \fg{} dans lesquelles les chercheurs peuvent \og [puiser] \fg{} à leur guise, tant les sujets abordés et les étendues géographiques traitées sont nombreux\footnote{\cite[p. 5]{chenu}.}. \textit{Les Ouvriers des deux mondes}, dont le titre fait écho à \textit{La revue des deux mondes} fondée en 1829, présentent en effet une succession de \og trajectoires et récits de vie de familles ouvrières \fg{}\footnote{\cite[p. 193]{baciocchi}.} établies de part et d'autre de la Méditerranée. Se succèdent ainsi une enquête consacrée à un charpentier de Paris\footcite{mono001a}, à un métayer de la banlieue de Florence\footcite{mono005a} et à un menuisier-charpentier de Tanger\footcite{mono012a}.

Alain Chenu, tout en appuyant la métaphore des \og mines \fg{} leplaysiennes, dénonce son caractère restrictif, et notamment l'idée selon laquelle les monographies seraient une succession d'enquêtes indépendantes. Le Play et la SESS ont en effet conservé une même \og grille d'observation \fg{} depuis la première (1856) jusqu'à la dernière enquête (1928), construisant \textit{de facto} un \og système dont les éléments prennent sens les uns par rapport aux autres \fg{}\footnote{Alain Chenu, \textit{op. cit.}, p. 5.}.

\textit{Les Ouvriers des deux mondes} présentent un double intérêt pour \textit{Time Us}. D'une part, leur attention dirigée de manière exclusive envers les familles ouvrières en font un matériau privilégié pour ce programme, d'autant que les monographies se focalisent sur le budget et son usage\footnote{Le budget est \og à la fois la méthode et le résultat \fg{} des monographies : \cite[p. 11]{cardoni}}. Aucun individu de la cellule familiale n'est ignoré : l'ouvrier --- il s'agit le plus souvent du père, qui peut être accompagné de ses frères ou de ses fils ---, ses descendants, ses ascendants, sa femme évidemment, mais aussi les domestiques\footcite{mono018a} et parfois les esclaves\footnote{Narcisse Cotte, \og Menuisier-charpentier... \fg{}, \textit{op. cit.}}. Ainsi, si seulement quatorze enquêtes ont pour sujet des familles travaillant dans l'industrie du textile, l'ensemble reste important à l'échelle du programme ANR dans la mesure où chaque enquête s'attache à établir le budget réservé aux matières textiles et à leur utilisation par la famille enquêtée.

D'autre part, le fait qu'il s'agit d'imprimés et la reproduction systématique de la même structure logique dans chaque monographie permettent d'envisager un traitement informatique de chaque volume.

\mainmatter
\part{Un corpus déjà structuré}

\chapter{Un corpus d'imprimés}

Début du premier chapitre

\chapter{Des numérisations multiples}

Début du deuxième chapitre

\chapter{Un encodage automatique}

Début du troisième chapitre

\backmatter
\part*{Bibliographie}
\addcontentsline{toc}{part}{Bibliographie}

\cite{inria}.

\cite{baciocchi}.

\cite{cardoni}.

\cite{chague}.

\cite{chenu}.

\cite{hincker}.

\cite{lefourner}

\cite{lorry}.

\cite{savoye}.

\cite{savoye2}.

\tableofcontents

\end{document}
