\chapter*{Liste des sigles et abréviations}
\addcontentsline{toc}{chapter}{Liste des sigles et abréviations}
\markboth{Liste des sigles abréviations}{} 

\begin{center}
\textit{Institutions}
\end{center} 

\begin{itemize}
    \item ALMAnaCH : \textit{Automatic Language Modelling and Analysis \& Computational Humanities} (Inria Paris)
    \item ANR : Agence nationale de la recherche
    \item CMH : Centre Maurice-Halbwachs (EHESS et ENS Paris)
    \item CNRS : Centre National de Recherche Scientifique
    \item CRH : Centre de recherches historiques (EHESS)
    \item EA : Équipe d'accueil
    \item EHESS : École des Hautes Études en Sciences Sociales
    \item ENS : École normale supérieure
    \item ICT : Identités, Cultures et Territoires (Université de Paris)
    \item Inria : Institut national de recherche en informatique et en automatique
    \item LARHRA : Laboratoire de Recherche Historique Rhône-Alpe (Lyon 2)
    \item TGIR : Très grande infrastructure de recherche
    \item TELEMMe : Temps, Espaces, Langages, Europe Méridionale-Méditerranée (Université d’Aix-Marseille)
    \item UMR : Unité mixte de recherche
\end{itemize}

\bigbreak

\begin{center}---

\bigbreak

\textit{Programmes de recherche}
\end{center} 

\begin{itemize}
    \item DAHN :\textit{Digital edition of historical manuscripts}
    \item MetaLEX : Métalexicographie numérique des langues historiques du
droit en Europe
    \item READ : \textit{Recognition and Enrichment of Archival Documents}
    \item \timeus{} :\textit{ time usage} (Rémunérations et usages du temps des femmes et des hommes en France de la fin du \textsc{xvii}\ieme ~siècle au début du \textsc{xx}\ieme ~siècle)
\end{itemize}

\begin{center}---

\bigbreak

\textit{Informatique et nouvelles technologies}
\end{center} 

\bigbreak

\begin{itemize}
    \item ALTO : \textit{Analyzed Layout and Text Object}
    \item API : \textit{Application Programming Interface}
    \item CSV : \textit{Comma-separated values}
    \item GPU : \textit{Graphics processing unit}
    \item HTML : \textit{Hypertext Markup Language}
    \item IETF : \textit{Internet Engineering Task Force}
    \item IIIF : \textit{International Image Interoperability Framework}
    \item JPEG : \textit{Joint Photographic Experts Group}
    \item JP2 : \textit{JPEG 2000}
    \item JSON : \textit{JavaScript Object Notation}
    \item \lse{} : \textit{Logical Structure Extraction from Les Ouvriers des Deux Mondes}
    \item OCR : \textit{Optical Character Recognition}
    \item ODD: \textit{One Document Does it all}
    \item PDF : \textit{Portable Document Format}
    \item Relax NG : \textit{Regular Language for XML Next Generation}
    \item RFC : \textit{Request for comments}
    \item RNG : \cf{} \textit{relax NG}
    \item TEI : \textit{Text Encoding Initiative}
    \item URI : \textit{Uniform Resource Identifier}
    \item XML : \textit{eXtensible Markup Language}
    \item XSL : \textit{eXtensible Stylesheet Language}
\end{itemize}

\clearpage
\thispagestyle{empty}