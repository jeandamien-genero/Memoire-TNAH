\section*{Résumé}
\addcontentsline{toc}{chapter}{Résumé}
\markboth{Résumé}{}

\bigbreak

 Résumé d'une trentaine de lignes à placer en tête du mémoire, accompagné d'une dizaine de mots-clés destinés à décrire le mémoire et des informations bibliographiques nécessaires pour le citer. Ce résumé et ces mots-clés sont destinés à compléter la notice bibliographique du mémoire dans la future bibliothèque numérique des mémoires.\\
N.B.: ne pas dépasser une page pour le tout.

\bigbreak

\textbf{Mots-clés:} XML ; TEI ; Python ; traitement automatique des données ; transcription automatique ; édition numérique ; ALTO ; \ocr ; \kraken{} ; \transkribus{} ; \gitlab{} ; Frédéric Le Play ; \lodm{} ; enquêtes sociologiques ; monographies de famille ; \timeus{} ; Inria.

\bigbreak

\textbf{Informations bibliographiques:} Jean-Damien Généro, \textit{Valoriser le traitement automatique des données : Le cas des Ouvriers des deux mondes}, mémoire du Master \og Technologies numériques appliquées à l'histoire \fg{}, dir. A. Chagué et V. Jolivet, École nationale des chartes, 2020.