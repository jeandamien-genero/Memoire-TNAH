\section*{Résumé}
\addcontentsline{toc}{chapter}{Résumé}
\markboth{Résumé}{}

\bigbreak

\textit{Le programme ANR \timeus{} s’intéresse aux ouvriers et aux ouvrières du textile de la fin du \textsc{xviii}\ieme ~siècle au début du \textsc{xx}\ieme ~siècle et rassemble pour cela une large documentation composée de documents manuscrits et d'imprimés. Au sein de ces derniers se trouvent les monographies de familles des} Ouvriers des deux mondes \textit{publiées de 1857 à 1930 par Frédéric Le Play (1806-1882) et la Société internationale des études pratiques d’économie sociale.}

\textit{Ce corpus de treize volumes, composé de 114 monographies, a été transcrit et structuré automatiquement au format XML-TEI par un programme utilisant les logiciels d’OCR Transkribus et Kraken, et le langage de programmation Python. Le présent mémoire se propose d'analyser les actions menées au cours d'un stage de fin d'études pour valoriser les résultats de cette structuration automatique, incluant son contrôle, la correction des erreurs et l’usage des humanités numériques pour implémenter un encodage scientifique permettant l’exploitation des données et des transcriptions par les chercheurs et les chercheuses.}

\bigbreak

\bigbreak

\bigbreak

\textbf{Mots-clés:} XML ; TEI ; Python ; traitement automatique des données ; transcription automatique ; édition numérique ; ALTO ; \ocr ; \kraken{} ; \transkribus{} ; \gitlab{} ; Frédéric Le Play ; \lodm{} ; enquêtes sociologiques ; monographies de familles ; \timeus{} ; Inria.

\bigbreak

\bigbreak

\bigbreak

\textbf{Informations bibliographiques:} Jean-Damien Généro, \textit{Valoriser le traitement automatique des données : Le cas des Ouvriers des deux mondes}, mémoire du Master \og Technologies numériques appliquées à l'histoire \fg{}, dir. Alix Chagué et Vincent Jolivet, École nationale des chartes, 2020.

\bigbreak

\bigbreak

\bigbreak

\textbf{Soutenance:} mémoire présenté et soutenu publiquement le 19 octobre 2020 à l'École nationale des chartes, devant un jury composé d'Édouard Vasseur, président, professeur d’Histoire des institutions, diplomatique et archivistique contemporaines, de Vincent Jolivet, responsable de la mission projets numériques et d'Alix Chagué, ingénieure de recherche et de développement de l’équipe ALMAnaCH d’Inria ; sanctionné par une mention Trés bien et la note de 18/20.

\bigbreak

\bigbreak

\textbf{Illustration de couverture:} Émile Savoy, \textit{Chocolatier de la fabrique de chocolat au lait F.-L. Cailler à Broc (canton de Fribourg, Suisse)}, dans \lodm, Paris, série 3, 1913, p. 325, \og La fabrique Cailler \fg.