\section*{Préambule : un stage confiné ?}
\addcontentsline{toc}{chapter}{Préambule : un stage confiné ?}
\markboth{Préambule}{} 

Le stage qui a donné lieu au présent mémoire n'a pas commencé sous les meilleurs auspices. Des difficultés administratives et procédurales, puis le grand confinement et les difficultés plus grandes encore qui en ont résulté, ont failli avoir raison de lui. Il a fallu se battre pour que l'Université de Paris admette que la fermeture des établissements d'enseignement annoncée par le président de la République le 12 mars 2020, puis les mesures de confinement à partir du 17 mars, ne signifiaient en aucun cas une suspension de l'action de son administration et un report automatique et sans appel des procédures en court, dont celle concernant les stages. Je tiens donc en premier lieu à remercier Mesdames Alix Chagué et Manuela Martini, Messieurs Julien Cassefières et Thibault Clérice, qui m'ont apporté un tant soit peu de soutien et d'aide dans cette procédure extrêmement pénible, absolument anormale et pour laquelle j'ai dépensé une quantité d'énergie démesurée.

Contactée, la présidence de l'Université de Paris mit fin au marasme, apposa les cachets requis et présenta ses excuses ; le stage pu ainsi commencer, avec une semaine de retard sur la date prévue.

Intégrer une équipe d'ingénierie et de recherche tout en étant confiné chez soi n'est pas chose aisée. Les relations humaines que cet exercice suppose en temps normal ont été réduites au minimum et sont passées par des courriels, des échanges sur l'espace de discussion instantanée d'INRIA et, principalement, des visioconférences sur \textit{Zoom}.

Inconnue, immédiatement décevante et parfois pénible, cette façon de travailler a mis un certain temps à s'imposer pour moi et les premières semaines ont été difficiles. Début juin, c'est-à-dire à la moitié du stage, Alix Chagué me demanda de lui transmettre un bilan des compétences que j'avais acquises jusqu'ici. Elles étaient nombreuses du point de vue des savoir-faire, car, bon gré mal gré, je n'avais eu d'autre choix que de m'adapter à ce travail à distance et à remplir les missions qui m'avaient été assignées. Ce faisant, un des objectifs que je m'étais fixé pour ce stage fut très vite satisfait : mes savoir-faire se sont développés par l'apprentissage et la maîtrise de nouvelles techniques.

\textit{Quid} des savoir-être ? J'avais listé plusieurs points dans le bilan de mi-stage, sur un ton légèrement humoristique, et notamment saluer, faire une présentation et suivre une réunion en visioconférence. Un véritable savoir-être se cachait derrière cet humour désabusé :  ce stage m'a appris le travail à distance. Cela suppose beaucoup de choses. Outre une organisation et une discipline peut-être plus importantes que pour un travail de bureau, il s'agit d'être seul la majorité du temps. De ne pas avoir de collègue avec qui l'on peut échanger et progresser dans la résolution d'un problème. En un mot, il s'agit de travailler plus avec soi-même qu'avec les autres et donc de se \textit{débrouiller}.