\section*{Préambule : un stage confiné ?}
\addcontentsline{toc}{chapter}{Préambule : un stage confiné ?}
\markboth{Préambule}{} 

Le stage qui a donné lieu au présent mémoire n'a pas commencé sous les meilleurs auspices. Des difficultés administratives et procédurales, puis le grand confinement et les difficultés plus grandes encore qui en ont résulté, ont failli avoir raison de lui. Il a fallu se battre pour que l'Université de Paris admette que la fermeture des établissements d'enseignement annoncée par le président de la République le 12 mars 2020, puis les mesures de confinement à partir du 17 mars, ne signifiaient en aucun cas une suspension de l'action de son administration et un report automatique et sans appel des procédures en cours, dont celles concernant les stages. Je tiens donc en premier lieu à remercier Mesdames Alix Chagué et Manuela Martini, Messieurs Julien Cassefières et Thibault Clérice, qui m'ont apporté un tant soit peu de soutien et d'aide dans cette procédure extrêmement pénible, absolument anormale et pour laquelle j'ai dépensé une quantité d'énergie démesurée.

Contactée, la présidence de l'Université de Paris mit fin au marasme, apposa les cachets requis et présenta ses excuses ; le stage pu ainsi commencer, avec une semaine de retard sur la date prévue.

Intégrer une équipe d'ingénierie et de recherche tout en étant confiné chez soi n'est pas chose aisée. Les relations humaines que cet exercice suppose en temps normal ont été réduites au minimum et sont passées par des courriels, des échanges sur l'espace de discussion instantanée d'INRIA et, principalement, des visioconférences sur \textit{Zoom}.

Ce stage confiné a ainsi été une expérience singulière mais tout de même enrichissante et professionalisante. J'ai pu échanger (certes par écran interposé) avec de nombreuses personnes auprès de qui j'ai beaucoup appris, tant au niveau technique que professionnel.

Je remercie Alix Chagué, ingénieure de recherche et de développement de l'équipe ALMAnaCH d'Inria, qui a encadré ce stage jour après jour, et Vincent Jolivet, responsable de la mission projets numériques de l'École nationale des chartes, qui a également assuré un encadrement professionnel et technique. Leurs conseils et leurs expériences m'ont été extrêmement profitables. Je remercie également Madame Manuela Martini, professeure de l'Université Lumière Lyon 2 et coordinatrice du programme ANR \timeus{}, qui s'est rendue disponible pour organiser des réunions régulières afin de suivre l'avancement de mon travail. Stéphane Baciocchi, ingénieur de recherche du Centre de recherches historiques, et Anne Lhuissier, chercheuse au Centre Maurice-Halbwachs, m'ont également apporté une aide précieuse dans la compréhension de la collection des \odm.

Un stage ne s'effectue pas dans de telles conditions sans un soutien amical fort : merci à mes camarades de promotion, Anne Brunet, Mathilde Daugas, Edward Gray, Chloé Fize, Morgane Rousselot, Lucas Terriel ; merci enfin à ceux qui sont avec moi depuis plus longtemps encore, Jeanne Dupiot, Hadrien de Visme, Alice Montaut d'Argy.