\section*{Préambule : un stage confiné ?}
\addcontentsline{toc}{chapter}{Préambule : un stage confiné ?}
\markboth{Préambule}{} 

Le stage qui a donné lieu au présent mémoire n'a pas commencé sous les meilleurs auspices. Des difficultés administratives et procédurales, puis le grand confinement et les difficultés plus grandes encore qui en ont résulté, ont failli avoir raison de lui. Il a fallu se battre pour que l'Université de Paris admette que la fermeture des établissements d'enseignement annoncée par le président de la République le 12 mars 2020, puis les mesures de confinement à partir du 17 mars, ne signifiaient en aucun cas une suspension de l'action de son administration et un report automatique et sans appel des procédures en cours, dont celles concernant les stages. La situation a fini par se débloquer et le stage a commencé avec une semaine de retard sur la date prévue.

Intégrer une équipe d'ingénierie, de recherche et de développement tout en étant confiné chez soi n'est pas chose aisée. Les relations humaines que cet exercice suppose en temps normal ont été réduites au minimum et sont passées par des courriels, des échanges sur l'espace de discussion instantanée d'Inria et, principalement, des visioconférences sur \textit{Zoom}. Ce stage confiné a ainsi été une expérience singulière mais tout de même enrichissante et professionalisante. J'ai pu échanger (certes par écran interposé) avec de nombreuses personnes auprès desquelles j'ai beaucoup appris, tant au niveau technique que professionnel.

Alix Chagué, ingénieure de recherche et de développement de l'équipe ALMAnaCH d'Inria, a encadré ce stage jour après jour, et Vincent Jolivet, responsable de la mission projets numériques de l'École nationale des chartes, en a assuré l'encadrement professionnel et technique. Leurs conseils et leurs expériences m'ont été extrêmement profitables : je leur exprime ma plus vive gratitude.

Je remercie également Madame Manuela Martini, professeure de l'Université Lumière Lyon 2 et coordinatrice du programme ANR \timeus{}, qui a fait preuve d'une grande disponibilité pour organiser des réunions régulières afin de suivre l'avancement de mon travail. Ma reconnaissance va également à Stéphane Baciocchi, ingénieur de recherche du Centre de recherches historiques, et à Anne Lhuissier, chargée de recherche de l'INRAE affectée au Centre Maurice-Halbwachs. Leur aide précieuse m'a permis de comprendre la collection des \odm. Il m'est du reste indispensable de remercier Madame Liliane Hilaire-Pérez, professeure de l'Université de Paris, qui a assuré l'encadrement administratif du stage et s'est rendue disponible pour valider et signer tous les documents que ce rôle suppose.

Un stage ne s'effectue pas dans de telles conditions sans un soutien amical fort : merci à mes condisciples, Anne Brunet, Mathilde Daugas, Edward Gray, Chloé Fize, Morgane Rousselot, Lucas Terriel ; merci enfin à ceux qui sont avec moi depuis plus longtemps encore, Jeanne Dupiot, Hadrien de Visme, Alice Montaut d'Argy, et quelques autres encore.