\section*{Introduction : le projet Time Us}
\addcontentsline{toc}{chapter}{Introduction : le projet Time Us}

\bigbreak

\og Reconstituer les rémunérations et les budgets temps des travailleuses et des travailleurs du textile dans quatre villes industrielles françaises (Lille, Paris, Lyon, Marseille) dans une perspective européenne et de longue durée \fg{}\footnote{Présentation du programme sur le site de l'ANR (\url{https://anr.fr/Projet-ANR-16-CE26-0018}, consulté le \today).} est l'objectif du programme ANR \textit{Time Us}. Son intitulé exact, \og Rémunérations et usages du temps des femmes et des hommes en France de la fin du \textsc{xvii}\ieme ~siècle au début du \textsc{xx}\ieme ~siècle \fg{}\footnote{Présentation du programme sur le site du LARHA (\url{http://larhra.ish-lyon.cnrs.fr/anr-time-us}, consulté le \today).}, met en évidence une double articulation autour d'un temps long qui court depuis les premières manufactures de l'époque Moderne jusqu'à la fin de la révolution industrielle au tournant des \textsc{xix}\ieme  ~et~\textsc{xx}\ieme ~siècles, mais aussi sur le temps individuel et quotidien qui est celui d'un ouvrier ou d'une ouvrière au sein de ces grands mouvements historiques. Cette attention portée à l'usage du temps se traduit dans la dénomination courante du programme, \og Time us \fg{}, abréviation de l'anglais \og \textit{time usage} \fg{}.

\textit{Time Us} se concentre principalement sur les femmes et plus encore sur les ouvrières du textile, industrie dans laquelle \og elles sont présentes dans toutes les phases du processus productif \fg{}\footnote{\cite[p. 1]{inria}.}. Le programme tend à combler le biais des genres dans l'historiographie du travail industriel en réalisant une opération de collecte et de traitement de la documentation manuscrite et imprimée relative à l'emploi et aux activités quotidiennes des femmes\footnote{\textit{Ibid}.}.  Ainsi, le moteur de \textit{Time us} est moins la production d'une réflexion scientifique autour du travail des femmes que la constitution d'un corpus documentaire sériel et prêt à être exploité par des chercheurs  d'horizons multiples. La pluridisciplinarité est en effet un aspect majeur du programme, qui souhaite une utilisation de ses données dans un maximum de champs de recherche des sciences humaines et sociales, notamment \og en  histoire économique et sociale, en histoire de la famille et du genre, en histoire des conflits du travail et de la culture des classes populaires \fg{}\footnote{\textit{Ibid}, p. 2.}.

La documentation est essentiellement constituée de documents manuscrits conservés dans la Bibliothèque municipale de Lyon et dans les dépôts d'archives départementaux et municipaux lillois, parisien, lyonnais et marseillais.  Très diverse, elle issue d'organes officiels, à l'instar des conseils prud'homaux, des chambres de commerce ou encore des tribunaux de commerce, mais aussi d'organismes ou d'individus privés tels que la presse ouvrière, les archives du tisseur lyonnais Pierre Charnier\footnote{Les papiers de Pierre Charnier font partie du fonds \og Fernand Rude \fg{} de la Bibliothèque municipales de Lyon, nommé d'après l'historien qui les possédait avant leur versement (\url{https://www.bm-lyon.fr/collections-patrimoniales-et-specialisees/explorer-les-collections/article/fernand-rude}, consulté le \today).}, les livres de compte et le dossier de faillite de la filature parisienne Dupuis-Drouet, etc.).
