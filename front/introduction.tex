\part*{Introduction}

\begin{center}
    \section*{Les \odm\\au sein de l'ANR \timeus}
\end{center}
\addcontentsline{toc}{chapter}{Introduction}
\markboth{Introduction}{} 

\bigbreak

\bigbreak

\og Reconstituer les rémunérations et les budgets temps des travailleuses et des travailleurs du textile dans quatre villes industrielles françaises (Lille, Paris, Lyon, Marseille) dans une perspective européenne et de longue durée \fg{}\footnote{Présentation du programme sur le site de l'ANR (\url{https://anr.fr/Projet-ANR-16-CE26-0018}, consulté le \today).} est l'objectif du programme ANR \timeus. Son intitulé exact, \og Rémunérations et usages du temps des femmes et des hommes en France de la fin du \textsc{xvii}\ieme ~siècle au début du \textsc{xx}\ieme ~siècle \fg{}\footnote{Présentation du programme sur le site du LARHA (\url{http://larhra.ish-lyon.cnrs.fr/anr-time-us}, consulté le \today).} met en évidence sa double articulation autour d'un temps long qui court depuis les premières manufactures de l'époque Moderne jusqu'à la fin de la révolution industrielle au tournant des \textsc{xix}\ieme  ~et~\textsc{xx}\ieme ~siècles, et du temps individuel et quotidien qui est celui d'un ouvrier ou d'une ouvrière au sein de ces grands mouvements historiques. Les relations entre la rémunération et le temps sont au c\oe{}ur des questionnements du programme. Cette attention portée à l'usage du temps se traduit dans la dénomination courante du programme, \og \timeus{} \fg{}, abréviation de l'anglais \og \textit{time usage} \fg{}.

Initialement financé sur une durée de trente-six mois (janvier 2017-janvier 2020), le programme ANR, coordonné par le professeur Manuela Martini de l'Université Lumière Lyon 2, est porté par une équipe pluri-institutionnelle.  Celle-ci est composée de quatre unités mixtes de recherche (UMR) formées d'universitaires et de chercheurs du Centre National de la Recherche Scientifique\footnote{Le Laboratoire de recherches historiques Rhône-Alpes (LARHRA, Lyon~2), le laboratoire Temps, Espaces, Langages, Europe Méridionale-Méditerranée (TELEMMe, Université d'Aix-Marseille), l'Institut de Recherches Historiques du Septentrion (Université de Lille) et le Centre Maurice-Halbwachs (CMH, EHESS et ENS Paris).}, d'une équipe d'accueil (EA)\footnote{Le laboratoire Identités, Cultures et Territoires (ICT) .} de l'Université de Paris\footnote{Université instituée par le décret n° 2019-209 du 20 mars 2019 et regroupant les établissements Paris~V-Descartes et Paris~VII-Diderot.} et de l'équipe projet ALMAnaCH\footnote{\textit{Automatic Language Modelling and Analysis \& Computational Humanities}.} de l'Institut national de recherche en informatique et en automatique (Inria\footnote{\og Inria \fg{} est depuis 2011 une marque ; aussi l'élision par le biais d'un \og l' \fg{} liminaire n'est-elle plus requise dans son écriture courante (\og Rappel : désormais \textit{L'INRIA} s'écrit \textit{Inria }», \url{https://web.archive.org/web/20111113064215/https://www.inria.fr/institut/inria-en-bref/charte-logo-inria/charte}, consulté le \today).} Paris). Le stage qui a donné lieu au présent mémoire était placé sous la responsabilité administrative du laboratoire ICT  de l'Université de Paris et a été effectué sous le tutorat scientifique et professionnel d'Inria, en la personne d'Alix Chagué, ingénieure de recherche et de développement au sein de l'équipe ALMAnaCH.

\timeus{} se concentre principalement sur les femmes et plus encore sur les ouvrières du textile, industrie dans laquelle \og elles sont présentes dans toutes les phases du processus productif \fg{}\footnote{\cite[p. 1]{inria}.}. Le programme tend à combler le biais des genres dans l'historiographie du travail industriel en réalisant une opération de collecte et de traitement de la documentation manuscrite et imprimée relative à l'emploi et aux activités quotidiennes des femmes\footnote{\textit{Ibid}.}.  Ainsi, le moteur de \timeus{} est moins la production d'une réflexion scientifique autour du travail des femmes que la constitution d'un corpus documentaire sériel et prêt à être exploité par des chercheurs  d'horizons multiples. La pluridisciplinarité est un aspect majeur du programme, qui souhaite une utilisation de ses données dans un maximum de champs de recherche des sciences humaines et sociales, notamment \og en  histoire économique et sociale, en histoire de la famille et du genre, en histoire des conflits du travail et de la culture des classes populaires \fg{}\footnote{\textit{Ibid}, p. 2.}.

La documentation est essentiellement constituée de documents manuscrits conservés par la Bibliothèque municipale de Lyon et les dépôts d'archives départementaux et municipaux lillois, parisien, lyonnais et marseillais. Très diverse, elle est issue d'organes officiels, à l'instar des conseils prud'homaux, des chambres de commerce ou encore des tribunaux de commerce, mais aussi des archives de personnes physiques ou morales de droit privé, telles que celles du tisseur lyonnais Pierre Charnier (1795-1857)\footnote{Les papiers du canut Pierre Charnier font partie du fonds \og Fernand Rude \fg{} de la Bibliothèque municipales de Lyon, nommé d'après l'historien qui les possédait avant leur versement (\url{https://www.bm-lyon.fr/collections-patrimoniales-et-specialisees/explorer-les-collections/article/fernand-rude}, consulté le \today).}, ou les registres comptables (1817-1821) et le dossier de faillite (1821-1822) de la filature parisienne Dupuis-Drouet\footnote{Archives de Paris, D 12U1, n° 375-376 (\url{http://archives.paris.fr/arkotheque/inventaires/ead_ir_consult.php?a=4&ref=FRAD075_000727}, consulté le \today).}.

En sus de cette documentation manuscrite, le programme \timeus{} s'appuie sur trois grands corpus d'imprimés. Le premier se compose de neuf titres de la presse ouvrière lyonnaise, entièrement numérisés sur le site \textit{Numelyo} de la Bibliothèque municipale de  Lyon\footnote{Le corpus se trouve à l'adresse \url{https://collections.bm-lyon.fr/PER003} (consulté le \today).}. L'intérêt du programme pour ce corpus porte sur les nombreux compte rendus d'audience du Conseil des prud'hommes qui s'y trouvent, ainsi que sur les reproductions ou extraits de discours, les lettres ou encore les analyses économiques et sociales relayés par ces journaux. Quatre des titres sont publiés dans la première moitié des années 1830 (\textit{L'Écho de la fabrique}, \textit{L’Écho des travailleurs}, \textit{L'Indicateur} et \textit{La Tribune prolétaire}), les cinq restant l'étant dans les années 1840 (\textit{L’Écho des ouvriers}, \textit{L’Écho de la Fabrique de 1841}, \textit{L’Écho de l'industrie}, \textit{L'Avenir}, \textit{La Tribune Lyonnaise})\footnote{Présentation de ce corpus sur le wiki du programme : \url{http://timeusage.paris.inria.fr/mediawiki/index.php/Documentation_régionale_-_Presse_lyonnaise} (consulté le \today).}.

Un second corpus d'imprimés est constitué de monographies collectées sur le site \textit{Gallica} de la Bibliothèque nationale de France au format PDF. Très divers, on y trouve à la fois \textit{L'ouvrière} de Jules Simon (1861), un \textit{Dictionnaire général des tissus anciens et modernes} (1859-1863) ou encore un \textit{Traité complet sur la fabrication des étoffes de soie} (1859)\footnote{Liste complète sur le wiki du programme : \url{http://timeusage.paris.inria.fr/mediawiki/index.php/Aperçu_des_états\#Imprimés_divers} (consulté le \today).}. Pour le LARHRA, il s'agit d'une base complémentaire au projet, qui doit faciliter la contextualisation de la base archivistique principale, notamment en permettant de suivre l'évolution du  vocabulaire afférent au textile sur la période étudiée. Inria, dans un optique de traitement automatique du langage (TAL) considère quant à lui que ces ouvrages servent à la \og connaissance du domaine \fg{}, \cad{} que les modèles de langues en tireront le vocabulaire spécifique, les tournures syntaxiques et les associations fréquentes de mots propres au domaine du textile.

Le troisième et dernier corpus d'imprimés est composé des treize volumes de la série des \odm.

\begin{center}
$\star$
\end{center} 

Initiées par le sociologue Frédéric Le Play (1806-1882), les \odm{} sont des enquêtes sociologiques conduites par les membres de la Société internationale des études pratiques d’économie sociale\footnote{Actuelle Société d'économie et de sciences sociales, désormais abrégée en \sess.} de 1857 à 1928. Répartie en trois séries comptant un total de cent vingt-six monographies\footnote{\cite[p. 95]{lorry}.}, il s'agit de la deuxième entreprise d'étude empirique de Le Play, après celle des \textit{Ouvriers européens} dont la première édition a lieu en 1855\footnote{\textit{Ibid}.}. Les enquêtes sont couramment désignées sous le vocable de \og monographies de famille \fg{} ou plus simplement \og monographies \fg{}.

Dans sa préface du numéro spécial de la revue \textit{Les Études sociales} consacré aux monographies leplaysiennes, Alain Chenu relève la prégnance de l'assimilation de celles-ci à des \og mines \fg{} dans lesquelles les chercheurs peuvent \og [puiser] \fg{} à leur guise, tant les sujets abordés et les étendues géographiques traitées sont nombreux\footnote{\cite[p. 5]{chenu}.}. \textit{Les Ouvriers des deux mondes}, dont le titre fait écho à \textit{La revue des deux mondes} fondée en 1829, présentent en effet une succession de \og trajectoires et récits de vie de familles ouvrières \fg{}\footnote{\cite[p. 193]{baciocchi}.} établies de part et d'autre de la Méditerranée, et parfois plus loin encore. Se succèdent ainsi une enquête consacrée à un charpentier de Paris\footcite{mono001a}, à un métayer de la banlieue de Florence\footcite{mono005a} et à un menuisier-charpentier de Tanger\footcite{mono012a}.

Les \odm{} présentent un double intérêt pour \timeus. D'une part, leur attention dirigée de manière exclusive envers les familles ouvrières en font un matériau privilégié pour ce programme, d'autant que les monographies se focalisent sur le budget et son usage\footnote{Le budget est \og à la fois la méthode et le résultat \fg{} des monographies : \cite[p. 11]{cardoni}}. Aucun individu de la cellule familiale n'est ignoré : l'ouvrier --- il s'agit le plus souvent du père, qui peut être accompagné de ses frères ou de ses fils ---, ses descendants, ses ascendants, sa femme évidemment, mais aussi les domestiques\footcite{mono018a} et parfois les esclaves\footnote{Narcisse Cotte, \og Menuisier-charpentier... \fg{}, \textit{op. cit.}}. Ainsi, si seulement quatorze enquêtes ont pour sujet des familles travaillant dans l'industrie du textile, l'ensemble reste intéressant à l'échelle du programme ANR dans la mesure où chaque enquête s'attache à établir le budget affecté aux matières textiles et à son utilisation par la famille enquêtée. L'angle de lecture adopté par \timeus{} concerne donc la totalité du texte pour quatorze monographies, et se focalise sur les paragraphes et tableaux relatifs aux individus et à leurs budgets pour les cent douze restantes.

D'autre part, la reproduction systématique d'une même structure logique dans chaque monographie permet d'envisager un traitement informatique de l'ensemble du corpus et, à travers celui-ci, la mise à disposition des textes structurés dans des fichiers XML. L'idée des \og mines \fg{} leplaysiennes est en effet inexacte, et Alain Chenu dénonce le caractère restrictif de cette image : les monographies ne sont pas une succession d'enquêtes indépendantes. Le Play et la \textsc{sess} ont conservé une même \og grille d'observation \fg{} depuis la première (1856) jusqu'à la dernière enquête (1928), construisant \textit{de facto} un \og système dont les éléments prennent sens les uns par rapport aux autres \fg{}\footcite[p. 5]{chenu}.

\timeus{} a souhaité transcrire automatiquement les volumes des \odm{} à partir de leurs numérisations, avant d'implémenter la structure logique d'origine au sein du texte brut obtenu. Le but visé est de mettre à disposition de la communauté scientifique ces textes afin qu'elle puisse y porter un regard nouveau à l'aide des outils numériques ; il s'agit de faire entrer ce \og matériau exceptionnel sur les sociétés des cinq continents \fg{}\footcite{savoye} dans le champ des humanités numériques.

\begin{center}
$\star$
\end{center} 

Les rapports entre le programme ANR \timeus{} et le Master \tnah{} de l'École nationale des chartes remontent à 2018. Deux étudiantes, Alix Chagué (prom. 2018) et Victoria Le Fourner (prom. 2019) y ont effectué leur stage de fin d'étude. Alix Chagué a travaillé sur la collecte et le traitement de la documentation\footcite{chague}, tandis que Victoria Le Fournier s'est intéressée à la structuration de cette documentation en prenant pour exemple la juridiction prud'homale\footcite{lefourner}. Leurs mémoires correspondent à deux phases successives du projet, et la présente étude entend décrire sa phase finale de valorisation des fichiers obtenus. Si cet aspect l'inscrit dans la suite des précédents, il n'en reste pas moins fondamentalement différent.

En premier lieu, il porte une attention exclusive à un corpus du programme, les \odm. Alix Chagué balayait pour sa part l'ensemble de la documentation et Victoria Le Fourner s'appuyait quant à elle sur une de ses composantes majeures et transversales. Les documents afférents aux conseils des prud'hommes se trouvent en effet à la fois dans les archives propres à ces institutions et, pour la juridiction lyonnaise, sous forme de comptes rendus d'audience publiés dans la presse. Le présent travail ne cherche pas à replacer les \odm{} au sein de la documentation globale du programme.

Au commencement du stage ayant donné lieu à ce mémoire en avril 2020, les \odm{} étaient déjà entièrement mis en forme et structurés au format XML-TEI. Ainsi --- c'est la deuxième différence d'avec nos prédécesseures --- nous sommes intervenus dans la phase de post-traitement, \cad{} que l'essentiel du stage a été consacré au contrôle qualité des fichiers, à la réparation des erreurs issues du traitement automatique et à une réflexion sur la valorisation à mener des données ainsi produites.

Plusieurs objectifs ont été poursuivis durant ce stage, et les questionnements qu'ils portent sous-tendent le propos que nous allons tenir entre ces pages. Il faut tout d'abord rappeler que le mémoire du Master \tnah{} est un exercice à mi-chemin entre le mémoire de recherche et le rapport de stage. Il ne s'agit pas d'établir un compte rendu des missions reçues et des actions menées, ni d'exposer une réflexion scientifique sur un sujet délimité. Il s'agit de mettre en perspective de manière critique l'intervention du stagiaire en la replaçant au sein des dynamiques scientifiques propres au projet et à l'équipe où le stage a été effectué.

Dans cet optique, notre intervention avait un premier objectif technique et professionnel. Il s'agissait d'analyser et de reprendre des fichiers XML-TEI afin de corriger leurs erreurs structurelles dans le but d'amorcer leur valorisation. Pour mener à bien cette opération, nous avions la possibilité de recourir à des techniques automatiques ou semi-automatiques en appliquant des scripts Python, ou bien d'effectuer les corrections en manuel. Le second objectif a résulté de cette façon de faire et interrogeait la valeur ajoutée de l'automatisation pour l'édition numérique d'une documentation imprimée dans le cadre d'un programme de recherche.

Nous tiendrons dans ce mémoire un propos en trois parties. Pour commencer, nous présenterons les différents états du corpus des \odm{}, qui est à la fois imprimé, numérisé et structuré en fichiers informatiques. Cela sera pour nous l'occasion de décrire la structure logique mise en place par Le Play et de faire la part entre la volonté de continuité de ses successeurs dans la réutilisation de celle-ci et les innovations qu'ils ont pu mettre en place au fil du temps. Nous conclurons cette première partie par la description des opérations qui ont permis d'obtenir les fichiers des monographies, qui nous furent transmis sous la forme d'un dépôt au sein de l'espace GitLab d'Inria.

La seconde partie abordera les reprises que nous avons effectuées, en commençant par définir la méthode de travail suivie et notamment les outils de développements et les espaces d'échange  mis en place pour la gestion du projet. Après avoir présenté une typologie des erreurs identifiées à l'issue de l'analyse des documents, nous exposerons nos interventions tant manuelles qu'automatiques en procédant du général au particulier, \cad{} du niveau global du corpus à celui, plus fin, des fichiers.

Nous réfléchirons enfin aux différentes manières de valoriser ces fichiers dans la troisième partie, en insistant sur trois voies possibles. La première consiste à déterminer l'opportunité de conserver un lien entre le texte de la page et son image d'origine, ainsi que la nature que celui-ci pourrait prendre (stockage local, utilisation des ressources de la plate-forme d'hébergement des numérisations, recours à un protocole \textsc{iiif}\footnote{Abréviation d'\textit{International Image Interoperability Framework}, généralement prononcée \og~triple~\textsc{i}~\textsc{f}~\fg{} en français.}). La seconde concerne l'indexation des individus enquêtés et la place que pourrait tenir l'automatisation dans cette opération. La dernière porte enfin sur la possibilité de corriger les transcriptions pour que les fichiers soient exploitables non seulement au niveau des données qu'ils contiennent, mais aussi au niveau du texte en lui-même, \cad{} de permettre et à la machine et à l'humain de les utiliser\footnote{Ce mémoire est entièrement rédigé avec le langage \LaTeX. Les fichiers qui le constituent sont disponibles sur le GitHub de l'auteur, consultables et téléchargeables à cette adresse : \url{https://github.com/jeandamien-genero/Memoire-TNAH}.}.

