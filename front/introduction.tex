\section*{Introduction : le projet Time Us}
\addcontentsline{toc}{chapter}{Introduction : le projet Time Us}

\bigbreak

\og Reconstituer les rémunérations et les budgets temps des travailleuses et des travailleurs du textile dans quatre villes industrielles françaises (Lille, Paris, Lyon, Marseille) dans une perspective européenne et de longue durée \fg{}\footnote{Présentation du programme sur le site de l'ANR (\url{https://anr.fr/Projet-ANR-16-CE26-0018}, consulté le \today).} est l'objectif du programme ANR \textit{Time Us}. Son intitulé exact, \og Rémunérations et usages du temps des femmes et des hommes en France de la fin du \textsc{xvii}\ieme ~siècle au début du \textsc{xx}\ieme ~siècle \fg{}\footnote{Présentation du programme sur le site du LARHA (\url{http://larhra.ish-lyon.cnrs.fr/anr-time-us}, consulté le \today).}, souligne la place centrale qu'occupe le temps au sein des questionnements qui animent ce programme. Ils se fondent sur le temps long qui court depuis les premières manufactures de l'époque Moderne jusqu'à la fin de la révolution industrielle au tournant des \textsc{xix}\ieme  ~et~\textsc{xx}\ieme ~siècles, mais aussi sur le temps individuel et quotidien qui est celui d'un ouvrier ou d'une ouvrière au sein de ces grands mouvements historiques. Cette attention portée à l'usage du temps se traduit dans la dénomination courante du programme, \og Time us \fg{}, abréviation de l'anglais \og \textit{time usage} \fg{}.

\textit{Time Us} se concentre principalement sur les femmes et plus encore sur les ouvrières du textile, industrie dans laquelle \og elles sont présentes dans toutes les phases du processus productif \fg{}\footnote{\cite[p. 1]{inria}.}.

