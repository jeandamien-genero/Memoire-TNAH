\part*{Conclusion}
\addcontentsline{toc}{part}{Conclusion}
\markboth{Conclusion}{Conclusion} 

\vspace*{\stretch{2}}

Au commencement de notre stage, nous avons trouvé un corpus de 223 fichiers XML-TEI pesant 25,7 Mo : nous en rendons 194 et 18,6 Mo. Ces chiffres ne résument qu'imparfaitement notre action ; au niveau du dépôt \gitlab, le poids total est passé de 51,6 Mo à 58,8 Mo. Si les fichiers XML ont en effet subi un effort de rationalisation de leur structure, de nombreux fichiers de métadonnées --- JSON, CSV et XML --- ont été ajoutés, ainsi que les scripts python ayant permis l'automatisation des tâches.

L'automatisation était l'enjeu central de notre stage. Il s'agissait de déterminer quelle était sa valeur ajoutée pour la production d'un corpus numérique destiné à la recherche. Il faut nuancer d'emblée le mot \og automatisation \fg{} et le remplacer par celui de \og semi-automatisation \fg{}. La technologie apporte un gain de temps important et appréciable pour certaines tâches --- ainsi, un index XML constitué en quelques secondes, 194 fichiers modifiés en à peine plus de temps --- mais elle n'est pas apte à régler tout les problèmes posés par la constitution d'un corpus numérique pour la recherche.  D'un \pov{} professionnel, l'enseignement principal que nous retirons de ce stage est qu'un tel projet est à la confluence de tant d'intérêts différents, qu'une validation humaine est toujours nécessaire pour satisfaire, ou du moins tenter de satisfaire les chercheurs, chercheuses et ingénieur(e)s qui en sont parties prenantes et dont les vues sont souvent éloignées.

Ceci posé, il est évident que des écueils majeurs ont pu être surmontés. La structure logique est le principal. Ce monument emblématique des monographies de familles est désormais présent dans les fichiers XML et peut être interrogé \textit{via} des requêtes XPath, qui permettent de se déplacer le long de l'arbre XML.

Nous ne laissons par pour autant un travail achevé : plusieurs actions doivent encore être menées afin de parvenir à un corpus dont les données seraient, d'une part, exploitables par les chercheurs et les chercheuses, et, d'autre part, publiables sur une plate-forme d'édition. Certaines de ces actions sont structurelles et consistent à compléter un corpus où plusieurs pièces manquent, que ce soit en raison d'un déficit de transcription ou de l'absence des six dernières monographies dans le corpus d'\ia. Notre travail, en relevant les erreurs qui sont survenues lors de la phase d'\ocr{} et de l'activation du script \lse, devrait permettre d'accélérer l'acquisition de ces textes manquants --- à condition que les fichiers images soient trouvés.

Le niveau documentaire n'est pas finalisé. Les notes de bas de page, les renvois internes aux monographies et externes aux \textit{Ouvriers européens} n'ont aujourd'hui aucune réalité dans la TEI. Ils doivent être dument encodés de manière à permettre à l'utilisateur de les exploiter pleinement. Un système de type \textit{Distributed Text Services} (DTS) doit également venir renforcer la cohérence documentaire en permettant au texte d'être cité et à ses données d'être ré-utilisées selon un niveau de granularité qui doit encore être défini.

L'effort principal devra se concentrer sur l'encodage scientifique. L'index des familles enquêtées, le traitement des tableaux de budgets et la qualité des transcriptions suscitent le plus d'attentions et d'attentes de la part des chercheurs et des chercheuses. Nous avons tenter de donner plusieurs pistes pour les mettre en valeur au mieux. Certaines sont à porter de main, notamment pour l'index et la correction des transcriptions ; la solution pour les tableaux, qui consiste à les laisser à l'état d'images tout en les indexant, est beaucoup moins satisfaisante. Cela illustre les limites de la technologie pour la recherche et fait espérer un avenir de la recherche pour la technologie, en particulier celle de l'acquisition des tableaux complexes.