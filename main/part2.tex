\part{Une structuration à reprendre : des tâches manuelles et semi-automatiques}

\clearpage
\thispagestyle{empty}
\cleardoublepage

\chapter{Outils, méthodologie et gestion du projet}


\section{Outils de développement}

\subsection{\gitlab}

La gestion quotidienne du programme \timeus{} se fait grâce à un dépôt sur \gitlab{}, un logiciel libre permettant aux entités comme Inria de créer une plate-forme interne de développement informatique. Sur \gitlab{} se trouve le dépôt central, organisé en plusieurs dossiers, dont un est réservé aux \odm. La technologie \textit{git} permet d'administrer les différentes versions du projet, sur une échelle à la fois verticale (les anciennes versions, appelées \commits, restant accessibles à travers un historique) et horizontale (le travail peut s'effectuer sur des \textit{branches} divergentes de la branche principale dite \master{} sans affecter l'état de cette dernière). Un commentaire est ajouté par l'utilisateur à chacun de ses \commits{} pour résumer ses modifications.

Chaque participant peut rapatrier le dépôt \gitlab{} en local pour travailler dessus ; ce rapatriement est un \textit{pull}. Il peut ensuite effectuer l'opération inverse, un \textit{push}, consistant à mettre ses \commits{} en ligne. Son équipe peut ainsi prendre connaissance des dernières avancées de la branche sur laquelle il travaille.

Une fois le travail dans une branche achevé, celle-ci peut être fusionnée avec \master. \gitlab{} propose une interface pour effectuer des \mergerequests{}. Il s'agit d'une demande de fusion, où \gitlab{} affiche l'historique de la branche. Les utilisateurs peuvent ainsi contrôler l'ensemble des \commits{} de la branche locale et vérifier qu'ils ne vont pas corrompre \master{} en créant des conflits (\textit{merge conflicts}).

\gitlab{} permet enfin à ses utilisateurs d'exposer un élément posant problème, d'effectuer des suggestions ou encore de porter à l'attention de leur équipe une ressource utile à travers des \issues{} (ou \textit{tickets}). Les \issues{} et les \mergerequests{} sont numérotées (\texttt{\#\textbackslash d} et \texttt{!\textbackslash d}\footnote{Par exemple, \texttt{\#5} fait référence à la cinquième \issue{} et \texttt{!5} à la cinquième \mergerequest{}.}), écrire leurs numéros dans \gitlab{} ou dans le commentaire de modification d'un \commit{} permettant de faire automatiquement référence à elles à travers un lien interne.

L'équipe ALMAnaCH possède un espace sur la plate-forme \gitlab{} d'Inria, où un dossier (en accès restreint) est réservé au programme \timeus{}. C'est ici, dans un sous-ensemble, que se trouve le dépôt des \odm. Le script \lse{} est déposé dans un dossier externe.

La branche \master{} compte quatre sous-dossiers :

\begin{itemize}
    \item \texttt{source} contient les treize fichiers XML-TEI des volumes des \odm{} ;
    \item \texttt{script} contient les scripts développés afin d'automatiser le traitement des fichiers ;
    \item \texttt{files} contient les fichiers XML-TEI générés à partir des fichiers sources et modifiés automatiquement ou à la main en vue de leur publication (monographies et fichiers de paratexte) ;
    \item \texttt{metadata} contient des fichiers de métadonnées sur le projet.
\end{itemize}

Ce dossier était notre espace de travail principal, notre première action ayant consisté en son rapatriement au niveau local. Notre méthodologie de travail était la suivante :

\begin{enumerate}
    \item Une branche était créée pour chaque mission ;
    \item Des \commits{} étaient effectués en local sur cette branche ;
    \item Les \commits{} d'une journée étaient mise en ligne sur \gitlab{} par un \textit{push} ;
    \item Lorsque la mission était achevée, une \mergerequest{} était ouverte ;
    \item La \mergerequest{} était acceptée et la branche fusionnée avec \master.
\end{enumerate}

\begin{figure}
    \centering
    \includegraphics[width=15cm]{img/pylint_output.png}
    \caption{Exemple d'un contrôle de code par \textit{Pylint}, qui donne une note à chaque script (\textit{module}) et détaille ensuite les erreurs détectées.}
    \label{fig:pylint}
\end{figure}

Le \textit{linter} \textit{Pylint} était également implanté dans le dépôt ; il s'agit d'un système de vérification de code Python. Son rôle est de contrôler à chaque \textit{push} la qualité du code des scripts, notamment la longueur des lignes, les intitulés des variables ou encore la documentation des fonctions. Le résultat est affiché dans une console (\fig{} \ref{fig:pylint}).

\subsection{\pycharm{} et \oxygen}

Pour manipuler les fichiers XML, développer et activer les scripts Python, nous usions des logiciels \oxygen{} (éditeur de code XML sous licence propriétaire) et \pycharm{} (environnement de développement intégré pour la programmation en Python, une version est sous licence libre et une seconde sous licence propriétaire). 

Plusieurs fonctionnalités d'\oxygen{} ont facilité le traitement du corpus, notamment la possibilité de rassembler l'ensemble des fichiers dans un \og projet \fg. Par ce biais, des opérations --- par exemple effectuer une recherche ou remplacer une expression par une autre --- peuvent être menées sur le corpus sans requérir l'ouverture successive de chaque fichier. En outre, le logiciel dispose d'un système de vérification du code, qui là encore peut être appliqué à tout le corpus à travers l'outil \og projet \fg.

\pycharm{} est également équipé d'un outil de type \textit{linter} comparable à \textit{Pylint}, qui permet de s'assurer de la validité et de la lisibilité du code et facilite la programmation. \textit{Pylint} est cependant plus exigeant que le \textit{linter} natif de \pycharm, un double contrôle était donc nécessaire.

\section{Espaces de discussion}

La situation de confinement d'avril à mai et le maintien de la fermeture aux stagiaires des locaux d'Inria de juin à juillet a conduit à la mise en place d'outils de discussion.

\subsection{\Mattermost}

\Mattermost{} est un logiciel de discussion instantanée dont le code, écrit à l'origine sous un format propriétaire, a été publié en \opensource{}\footnote{Consultable sur \github{} (\url{https://github.com/mattermost/mattermost-server}, consulté le \today).} en 2015\footnote{Lindsay Brock, \textit{Open source Slack-alternative reaches 1.0: Self-host ready, Slack-compatible, MIT licensed}, 2 octobre 2015, \url{https://mattermost.com/blog/mattermost-3-4-16/} (consulté le \today).}. Auto-hébergé --- Inria stocke le code dans ses propres installations et n'a pas recours à un serveur distant ---, il s'agit de l'espace de discussion principal des agents d'Inria.

Composé de plusieurs \og chaînes \fg{} (\textit{public channels}) organisées de façon thématique, il leur permet d'échanger sur les différents projets et de suivre leur avancement, mais aussi d'exposer les difficultés techniques qu'ils rencontrent dans leur travail quotidien afin d'obtenir de l'aide. La chaîne \og \ocr \fg{} est ainsi fréquemment utilisée par les utilisateurs de \kraken{} (\fig{} \ref{fig:mattermost}).

\begin{figure}
    \centering
    \includegraphics[width=16cm]{img/mattermost.jpg}
    \caption{Exemple de messages sur la chaîne \og \ocr \fg{} du \Mattermost{} d'Inria : une utilisatrice demande de l'aide au sujet d'une erreur dans l'exécution de \kraken{}. Sur le volet gauche se trouve la liste des chaînes disponibles.}
    \label{fig:mattermost}
\end{figure}

\subsection{\textit{Issues} et \mergerequests{} sur \gitlab}

Le programme \timeus{} possède également une chaîne ; cependant, pour les échanges afférents à nos missions, nous usions des espaces de discussion de \gitlab.

Les \issues{} et les \mergerequests{} n'ont en effet pas pour seule utilité de permettre aux utilisateurs d'exposer des problèmes ou de demander des fusions de branches. Il s'agit d'espaces dynamiques qui participent pleinement à la gestion de projet en offrant aux participants la possibilité de donner leur avis ou d'apporter des solutions, par exemple pour résoudre un \textit{merge conflict}.

Cet aspect est facilité par l'usage du langage à balises \markdown\footnote{\gitlab{} use de sa propre version du \markdown, le \textit{GitLab Flavored Markdown} : \url{https://docs.gitlab.com/ee/user/markdown.html} (consulté le \today).}. Il permet une mise en forme légère (listes, liens hypertexte, diagrammes), ainsi qu'une intégration d'échantillons de code. Si ceux-ci ne sont pas fonctionnels, le \markdown{} permet de leur appliquer une coloration syntaxique, les rendant ainsi plus facile à lire (\fig{} \ref{fig:ex_gitlab}).

Les échanges sur ces espaces restent accessibles après la clôture des \issues{} ou des \mergerequests{}, constituant ainsi un historique de l'avancement du projet.

\begin{figure}
    \centering
    \includegraphics[width=16cm]{img/gitlab.png}
    \caption{Exemple de messages échangés avec une coloration syntaxique d'un code Python dans une \issue{} sur \gitlab.}
    \label{fig:ex_gitlab}
\end{figure}

\section{Feuille de route}

\subsection{Les missions du stage}

Au commencement de notre stage, huit \issues{} étaient ouvertes sur le \gitlab{} des \odm{}. Chacune correspondait à un problème ou à un point qui n'avait pas encore pu être développé. Elles constituaient donc notre \og feuille de route \fg, \cad{} l'exposé des missions que nous avions à réaliser (\ann{} \ref{ann:feuille_route}). Il nous a été demandé de les utiliser pour poser nos questions ou proposer nos solutions.

Les missions qui nous ont été confiées étaient de deux ordres.

Une première moitié consistait à contrôler les résultats du script \lse, tant au niveau du corpus qu'à celui de chaque fichier, et à effectuer des reprises si nécessaire. Tout d'abord, il s'agissait de détecter les erreurs de découpage des fichiers de volume et d'opérer les fusions ou les séparations nécessaires. Ceci avait pour but de donner à chaque fichier un identifiant unique après s'être assuré de l'unité de son contenu (\issue{}~1). Dans un second temps, nous devions nous intéresser à des éléments particuliers dans chaque fichier, à l'instar du contenu des balises \texttt{<facsimile>} (\issue{}~2), de la qualité des transcriptions (\issue{}~6) ou encore de la validité du schéma TEI (\issue{}~3); le contrôle de l'implémentation de la structure logique ayant constitué notre occupation principale.

L'autre moitié de nos missions avait pour but de valoriser les données des \odm{} en menant des actions ciblées. La principale consistait à identifier les individus enquêtés en liant les informations onomastiques du deuxième paragraphe (\textit{État civil de la famille}) à un tableau prosopographique établi par Stéphane Baciocchi, chercheur de l'EHESS (\issue{}~4). De plus, nous devions implémenter dans les fichiers un système permettant la citation de passages précis afin de faciliter les études des chercheurs du programme en leur offrant la possibilité d'établir un lien direct entre leur travail et les données contenues dans les fichiers XML (\issue{}~5).

Il nous a enfin été demandé de publier sur le blog Hypothèses de \timeus{} des billets rendant compte de notre travail\footnote{Consultable à cette adresse : \url{https://timeus.hypotheses.org/}.}.

\subsection{Une gestion de projet ?}

Nous n'avons pas utilisé d'outil véritablement dédié à la gestion de projet, détournant plutôt un outil courant de \gitlab, les \issues.

Des fonctionnalités de \gitlab{} sont pourtant dédiées à une gestion plus fine. La principale est l'outil \og tableau de bord \fg{} (\textit{boards}). Par défaut, les \issues{} sont affichées sous la forme d'une liste présentant toutes celles qui sont ouvertes, deux onglets permettant d'accéder à celles qui ont été résolues ou bien de les afficher toutes. Avec l'outil \og tableau de bord \fg{}, l'utilisateur a accès à un tableau de quatre colonnes, la première listant les issues ouvertes, la seconde affichant une liste de tâches (\textit{todo list}), la troisième présentant les tâches en cours (\textit{doing}) et la dernières les \issues{} terminées.

Au-delà de la présentation des tâches, le tableau de bord est également interactif. Ainsi, sélectionner une \issue{} affiche ses métadonnées (agent en charge de sa résolution, label, temps estimé, échéance). Un système de labels --- des étiquettes thématiques --- permet également de classer les \issues{} et les tâches de la \textit{todo list}. Le tableau de bord offre donc une vision globale du projet selon une organisation chronologique, tout en permettant des vues synthétiques ciblées.

Le fonctionnement des \textit{boards} de \gitlab{} est comparable à celui d'autres applications, comme par exemple \textit{Trello}. Il s'agit d'un organisateur de tâche, également sous forme de tableaux, qui est utilisé par ALMAnaCH et les Archives nationales pour coordonner le projet de lecture automatique des répertoires du Minutier central des notaires de Paris, Lectaurep.

Pour autant, ces outils ne sont utilisés ni dans le cadre général du programme \timeus, ni dans le cadre particulier des \odm. Plusieurs raisons peuvent l'expliquer. En premier lieu, \timeus{} s'appuie sur une documentation dont le caractère disparate --- tant géographique que chronologique et typologique --- se heurte à toute volonté de gestion centralisée, d'autant que chaque membre est chargé de la gestion de sa documentation locale.

Un tel outil doit également être constamment maintenu à jour afin de garantir un gain de productivité. À l'échelle des \odm{} et de notre stage, le nombre relativement faible de missions (8) et d'intervenants (2) ne justifiaient pas la mise en place du tableau de bord \gitlab{} ou la création d'un \textit{Trello}. L'affichage basique des \issues{} sous forme d'une liste se suffisait à lui-même.

\chapter{Contrôle du découpage des fichiers source}

\section{Les différents niveaux d'encodage}

L'encodage d'un texte brut s'effectue sur plusieurs niveaux, depuis une échelle documentaire surplombante jusqu'à celle plus fine de l'analyse scientifique. Ces niveaux sont décrits dans un document intitulé \textit{Best Practices for TEI in Libraries}, édité par le Consortium TEI et disponible en ligne\footcite[\textit{4.2. Encoding Levels}]{bestpratice}.

Le premier niveau est celui du découpage documentaire, soit la constitution d'un fichier qui reproduit le texte brut d'une unité codicologique et lui associe des métadonnées. Dans le cas des \odm, il s'agit du découpage des treize fichiers des volumes, qui ne s'est pas fait sans erreur, et de la constitution du \texttt{<teiHeaer>}.

Le second niveau est celui dit de l'encodage \og minimal \fg, dont le seul but est d'améliorer la navigation dans le document. Il s'agit d'identifier les paragraphes par des balises \texttt{<p>} et de lier celles-ci aux ensembles \texttt{<facsimile>} des images d'origine par un identifiant, tant en marquant les changements de page par des éléments \texttt{<pb>}.

Le troisième niveau s'intéresse au découpage éditorial, \cad{} à l'identification et à la reproduction de la structure hiérarchique du texte, en l'occurrence la structure initiée par Frédéric Le Play et reprise par ses continuateurs.

Le quatrième niveau est celui de l'encodage sémantique, destiné à mettre en valeur les éléments internes au texte afin d'en faire une production électronique autonome. Dans les fichiers des \odm, cela s'est traduit par l'élimination des éléments de mise en page des volumes tels que les en-têtes ou les numéros de page.

Le cinquième et dernier niveau est celui de l'annotation scientifique. Pour l'équipe ALMAnaCH, il s'agissait de repérer les éléments d'onomastique et d'implanter un système permettant la citation de passage précis selon une granularité qui restait à définir.

L'ensemble de ces niveaux a été contrôlé lors du stage. Lorsqu'une correction était nécessaire, nous devions favoriser son automatisation par le biais d'un script Python. Cependant, ceci n’a pas toujours été possible, nous conduisant à engager des actions manuelles plus d'une fois.

\section{Vérification de la cohérence documentaire}

Nous avons commencé par contrôler le découpage des fichiers source, dont les résultats étaient étonnants pour certain volume. Ainsi, plus de trente fichiers avaient résulté du troisième volume de la deuxième série (\ann{} \ref{mappings2t3}).

Le contrôle a été opéré manuellement, par une ouverture successive des fichiers. Les monographies devaient commencer par la reproduction de l'en-tête, point de repère du script \lse{} pour le découpage. Deux erreurs majeures ont été constatée.

\subsection{Fission horizontale lors de la segmentation}

La première consistait en une erreur de découpage dans les précis n° 48 bis\footcite{mono048b}, 66 bis\footcite{mono066b} et 66 ter\footcite{mono066c}. Le premier est scindé en dix fichiers contenant deux pages chacun (\texttt{s2t1\_chapt\_6.xml} à \texttt{s2t1\_chapt\_15.xml}), le second et le troisième en respectivement sept et quatorze fichiers contenant quatre (\texttt{s2t3\_chapt\_7.xml} et \texttt{s2t3\_chapt\_8.xml} ; \texttt{s2t3\_chapt\_14.xml}) et deux pages (\texttt{s2t3\_chapt\_9.xml} à \texttt{s2t3\_chapt\_13.xml} ; \texttt{s2t3\_\-chapt\_15.xml} à \texttt{s2t3\_chapt\_27.xml}). Lorsqu'il y a deux pages, il s'agit toujours d'un recto et d'un verso, et de deux rectos et deux versos lorsqu'il y en a quatre.

Mis à part les premiers fichiers, qui commencent par le titre du précis, les \texttt{<body>} de ces fichiers ont un point commun : ils débutent tous par les deux mêmes lignes, \texttt{<p>\textsc{précis de monographie}</p> \textbackslash n <p>[numéro de la page]</p>}. Or ces deux informations --- le rappel du titre du chapitre et le numéro de la page courante --- se trouvent sur une seule et même ligne dans les images. Une erreur de fission horizontale résultant d'une mauvaise segmentation s'est donc produite au moment de l'\ocr\footcite[p. 5-6]{karpinski}.

Dans son comportement normal, \lse{} est programmé pour détecter tout ce qui relève de l'en-tête ou du pied de page et le retirer afin de permettre une reconstitution optimale des paragraphes au moment de la transformation des fichiers XML pour une éventuelle édition. Dans le cas présent, le script a été dupé par l'erreur de segmentation : il n'a pas vu des rappels de titre et des numéros de page, mais des titres et des numéros de chapitre. En conséquence, il a opéré une coupure à cet endroit, \cad{} achevé le fichier en cours pour en amorcer un nouveau.

Une question demeure néanmoins : pourquoi trois des fichiers contiennent-ils quatre pages, à l'inverse des autres qui n'en contiennent que deux ? Le fait est que les en-têtes des troisième pages ont été détectés convenablement, et donc retirés du flux du texte. Aussi la séparation n'était-elle pas nécessaire.

L'erreur a été résolue par un transfert manuel du contenu de chaque fichier particulier dans un fichier global (\texttt{s2t1\_chapt\_6.xml}, \texttt{s2t3\_chapt\_7.xml} et \texttt{s2t3\_chapt\_14}). Le choix a été fait de ne pas modifier la numérotation des fichiers suivants, notamment afin de pouvoir effectuer un suivi sur la durée. C'est la raison pour laquelle, dans la deuxième série, le n°~6 est suivi du n°~16 dans le premier volume et, dans le troisième, le  n°~7 par le n°~14, lui-même précédant le n°~28 (\ann{} \ref{mappings2t1} et \ref{mappings2t3}, p. \pageref{mappings2t1}).

Cette opération a conduit à la suppression de vingt-huit fichiers. Ajoutons à cela le retrait de \texttt{s2t1\_chapt\_23.xml}, doublon de \texttt{s2t2\_chapt\_5.xml} (\cf{} \ref{part:I1_longue_pub}, p. \pageref{part:I1_longue_pub}) : un total de vingt-neuf fichiers ont été supprimés, le corpus passe donc de 223 à 194 fichiers.

\subsection{Défaut de transcription}

\begin{figure}[t]
    \centering
    \includegraphics[width=15cm]{img/deficit_transcrip.png}
    \caption{Dans six fichiers, le contenu de la section \textit{Observations préliminaires} n'a pas été considéré comme du texte mais comme des figures. À l'inverse, la section \textit{Notes} a bien été prise en compte comme du texte. Exemple du fichier \texttt{s1t4\_chapt\_8.xml}.}
    \label{fig:deficit}
\end{figure}

La seconde erreur majeure à laquelle nous avons été confronté est le défaut d'une partie de la transcription dans six monographies (\ann{} \ref{ann:deficit-transcr}). Le titre et l'ensemble de la partie \textit{Observations préliminaires} manquent (n° 30\footcite{mono030a}, 33\footcite{mono033a}, 37\footcite{mono037a}, n° 44\footcite{mono044a}, 45\footcite{mono045a} et 46\footcite{mono046a}). Il ne s'agit pas d'une erreur de découpage, puisque déjà présente dans les treize fichiers source. C'est néanmoins l'opération de contrôle du découpage qui a permis de s'en rendre compte.

Il ne s'agit pas d'un déficit total. L'analyse de la mise en page et la segmentation se sont effectuées de manière convenable, comme en atteste la présence d'ensemble \texttt{<facsimile>} entre le \texttt{<teiHeader>} et le \texttt{<text>}, repris des éléments \texttt{<TextBlock>} des fichiers ALTO d'origine. Néanmoins, les paragraphes §1 à §16, qui peuvent contenir des figures ou des tableaux mais sont principalement composés de textes, ont été considérés comme des éléments graphiques. En conséquence, chaque page est représentée par un élément \texttt{<figure>} contenant une balise \texttt{<graphic>} que l'attribut \texttt{@facs} relie à un \texttt{<facsimile>} ; \texttt{<pb>} venant signifier le changement de page (\fig{} \ref{fig:deficit}).

Que s'est-il passé ? Le problème provient d'une fonction de \lse, \texttt{where\_do\_\-budgets\_start}. Son  rôle est de déterminer l'index de la page où commencent les tableaux de budget et de placer cette donnée dans la variable \texttt{budget\_start}. Pour cela, il cherche la ligne \og \textsc{budget des recettes de l'année} \fg, qui se trouve entre l'en-tête de la page et la ligne supérieure délimitant le tableau de budget. Une fois cette information connue, le script sait que la page correspondante et l'ensemble des suivantes jusqu'au commencement de la section \textit{Notes} (index placé dans la variable \texttt{budget\_stop}) devront être considérées comme des objets graphiques.

Or les trois monographies concernées du quatrième volume (n°~30, 33 et 37) présentent un problème au niveau de la segmentation de la première page du budget. En effet, soit l'en-tête a été considéré comme du texte et le reste comme une illustration (\fig{}~\ref{fig:odm30tkb} et~\ref{fig:odm33tkb}), soit l'ensemble de la page a été considéré comme une illustration (\fig{}~\ref{fig:odm37tkb}). \kraken, à qui les zones de segmentation sont adressées, n'a donc rien à transcrire (à l'exception de la ligne d'en-tête dans deux cas), et \lse{} n'obtient pas de résultat quand il cherche la ligne \og \textsc{budget des recettes de l'année} \fg{}. La variable \texttt{budget\_start} est équivalente à \texttt{false} et \lse{} considère que les budgets débutent à la première page de la monographie. Il remplace alors toutes les transcriptions par des objets graphiques jusqu'aux \textit{Notes} qu'il traite normalement, et nous obtenons un fichier partiellement transcrit.

\begin{figure}[t]
    \centering
    \begin{subfigure}{0.3\textwidth}
     \includegraphics[width=1\linewidth]{img/transkribus_30.png}
     \caption{N° 30, p. 104.}
     \label{fig:odm30tkb}
    \end{subfigure}
    \hspace{5pt}
    \begin{subfigure}{0.3\textwidth}
     \includegraphics[width=1\linewidth]{img/transkribus_33.png}
     \caption{N° 33, p. 260.}
     \label{fig:odm33tkb}
    \end{subfigure}
    \hspace{5pt}
    \begin{subfigure}{0.3\textwidth}
     \includegraphics[width=1\linewidth]{img/transkribus_37.png}
     \caption{N° 37, p. 426.}
     \label{fig:odm37tkb}
    \end{subfigure}
    \caption{Segmentations dans \textit{Transkribus} des pages liminaires des budgets des monographies n° 30,  33 et 37. Dans les monographies 30 et 33, seul la ligne d'en-tête est considérée comme du texte.}
    \label{fig:odmtkbs1t4}
\end{figure}

Ces fichiers requièrent une nouvelle \ocr{} et une nouvelle structuration ; une correction par le biais d'une transcription manuelle ne fait en effet pas partie des possibilités acceptables pour l'équipe ALMAnaCH. Cette opération a été repoussée et nous n'avons pas eu à la mener. La reprise de ces six fichiers pourra cependant servir à valider les modifications apportées à \lse{} en tenant compte de l'ensemble des axes d'amélioration relevés au cours du stage et présentés dans ce mémoire. Notamment, l'éventualité de l'absence de valeur pour \texttt{budget\_start} pourrait 
être envisagée et traduite par un message d'erreur informant l'opérateur que le calibrage de la segmentation doit être changé.

\subsection{Identification, cartographie et inclusion}

Au-delà d'une vérification de la cohérence du découpage, cette étape de vérification avait pour but de reconnaître le contenu des fichiers, de leur associer un identifiant unique et de produire un fichier de cartographie ou \og mapping \fg{} du corpus. Une partie des identifiants provenaient d'une bibliographie réalisée au format BibTeX\footnote{BibTeX est le logiciel de gestion de bibliographie du langage \LaTeX.} par Stéphane Baciocchi à partir de son exemplaire personnel du corpus.

Une partie seulement, car le fichier bibliographique ne s'intéressait qu'aux monographies, là où ALMAnaCH avait la volonté de travailler sur l'ensemble du corpus, monographies et fichiers de paratexte compris. Il a donc été nécessaire d'établir de nouveaux identifiants sur le modèle de ceux des monographies. Ces derniers se composent d'une série de trois chiffres suivie d'une lettre, \texttt{a} pour une monographie, \texttt{b} pour le précis bis et \texttt{c} pour le ter (\texttt{\textbackslash d\textbackslash d\textbackslash d[a-c]}). Nous avons donc choisi de donner au premier fichier de paratexte le numéro \texttt{401a} et de continuer jusqu'au dernier.

Trois versions de la cartographie ont été réalisée : la première contient uniquement les monographies, la seconde le paratexte et la troisième l'ensemble des fichiers\footnote{La troisième cartographie a été scindée en fonction des volumes afin de constituer l'annexe \ref{mapping} du présent mémoire.}. Ces versions sont enregistrées au format CSV : en texte brut, elles se présentent sous la forme d'une succession de lignes contenant des données (identifiant, intitulé puis libellé du fichier), ces données étant séparées par des virgules. Ces fichiers CSV peuvent ensuite être interprétés comme des tableaux, chaque ligne correspondant à une ligne tabulaire et chaque virgule à une séparation entre deux colonnes. Ce format, outre son extrême simplicité, a l'avantage d'être libre et de pouvoir être interprété par le langage Python. Les données peuvent donc être manipulées par un script.

Une fois les CSV constitué, il a fallu implanté les identifiants dans les fichiers TEI : il s'agissait de la première opération que nous avons pu automatiser. Après avoir relevé les identifiants et les libellés de fichier dans le CSV général pour constituer un dictionnaire Python (\texttt{dict\_xml = \{'401a': 's1t1\_chapt\_1.xml', etc\}}), le script ouvrait chaque fichier un par un. À l'aide de la librairie \textit{BeautifulSoup}, il analysait ensuite l'arbre XML (\fig{} \ref{fig:xmltree1}, p. \pageref{fig:xmltree1}) et implantait l'identifiant comme valeur de l'attribut \texttt{@xml:id} de \texttt{<TEI>} et de l'attribut \texttt{@ana} de la première \texttt{<div>}.

La première exécution de ce script a permis de faire remonter une erreur dans la constitution des identifiants : la grammaire TEI n'autorise qu'un \texttt{@xml:id} unique et qui commence par une lettre dans chaque document\footnote{\og \textit{Values for the \texttt{@xml:id} attributes must be unique within a single document, and \texttt{@xml:id} values must begin with a letter}\fg{} : \textit{TEI Guidelines, 3.10.2 Creating New Reference Systems} (\url{https://www.tei-c.org/release/doc/tei-p5-doc/fr/html/CO.html\#CORS2}, consulté le \today).}. Or il n'était pas possible de revenir sur ces identifiants, utilisés par d'autres chercheurs. La seule solution trouvée a été d'ajouter le préfixe \texttt{ID-} devant chaque identifiant afin de satisfaire aux règles de la TEI.

\begin{figure}
    \centering
    \includegraphics[width=15cm]{img/xinclude.png}
    \caption{Premières lignes du fichier \texttt{master.xml}. Dans la balise \texttt{<master>}, l'attribut \texttt{@xmlns:xi} contient l'adresse de l'espace de nom \textit{XInclude}.}
    \label{fig:xinclude}
\end{figure}

ALMAnaCH souhaitait enfin pouvoir disposer d'un fichier XML regroupant l'ensemble des 194 fichiers. Pour cela, nous avons eu recours à un mécanisme d'inclusion grâce au langage XInclude.

Le principe est de constituer un fichier --- ici, \texttt{master.xml} --- dont le seul contenu sera une balise \texttt{<master>} englobant une succession d'éléments \texttt{<xi:include>}. Ceux-ci possèdent un attribut \texttt{@href} contenant l'URI d'un fichier XML de paratexte ou de monographie. Il s'agit de l'identifiant uniforme d'une ressource (\textit{Uniform Resource Identifier}), \cad{} une séquence de caractères qui localise et nomme une ressource de manière pérenne\footnote{\og \textit{A URI is an identifier consisting of a sequence of characters matching the syntax rule named \texttt{<URI>}. It enables uniform identification of resources via a separately defined extensible set of naming schemes} \fg{} : RFC 3986, \textit{Uniform Resource Identifier (URI): Generic Syntax}, IETF, janvier 2005 (\url{https://tools.ietf.org/html/rfc3986}, consulté le \today).}. Dans notre corpus, les URI des fichiers ne sont pas leurs identifiants mais leurs libellés : au moment d'analyser \texttt{master.xml}, le logiciel --- par exemple, \oxygen{} --- va lui adjoindre le contenu de la ressource localisée par la première URI, \cad{} le fichier \texttt{s1t1\_chapt\_1.xml}, qui contient la page de titre du premier volume, puis passer à l'URI suivante (le premier paratexte du premier volume), et ainsi de suite jusqu'au dernier fichier du troisième volume de la troisième série.

Ce fichier \texttt{master.xml} a été là encore constitué de manière automatique avec un script qui rassemble tous les libellés des fichiers dans une liste et, pour chacun d'eux, écrit la ligne \texttt{<xi:include href="[libellé]"/>} (\fig{} \ref{fig:xinclude}).

\section{Vérification de l'encodage minimal}

La mise en forme minimale des documents a été correctement exécutée par \lse, aucune partie de la transcription n'ayant été laissée sans encodage.

Les objets graphiques ont pu néanmoins poser problème au script. Nous avons vu dans la section précédente que dans six fichiers, toute une partie du texte avait été représentée par des éléments \texttt{<figure>}. L'inverse s'est également produit, \cad{} que des objets graphiques véritables --- à l'instar des tableaux de budget des paragraphes 14 à 16 --- ont souvent été considérés comme du texte. Ainsi, dans la monographie n° 56\footcite{mono056a}, le tiers du premier tableau du paragraphe 14 est encodé dans un élément \texttt{<figure>} (articles 1 à 2), le reste, à partir de l'article 3, étant transcrit et placé dans des \texttt{<p>} (\fig{} \ref{fig:tableodm56xml}).

Pour autant, ces faits sont moins gênants que les cas inverses, dans la mesure où il s'agit de surplus de transcription. Les \texttt{<div>} encadrant ces sections de budgets étant en place, il est envisageable d'effacer automatiquement les balises \texttt{<head>} et \texttt{<p>} qu'elles contiennent et de n'ajouter que des \texttt{<graphic>} avec l'adresse de l'image de la page.

\begin{figure}
    \centering
    \includegraphics[width=15cm]{img/table_s2t2_chapt_5.png}
    \caption{Encodage du début du §14 de la monographie 56.}
    \label{fig:tableodm56xml}
\end{figure}

Cependant, toute mesure d'ajout ou de suppression d'élément \texttt{<p>} ou \texttt{<head>} a pour conséquence de rompre la logique interne au document. En effet, ces balises possèdent un attribut \texttt{@facs} dont la valeur renvoie à l'identifiant (\texttt{@xml:id}) d'un élément \texttt{<zone>} dans les \texttt{<facsimile>}. Pousser la précision de ces derniers jusqu'au niveau des paragraphes devient inutile puisque la correspondance entre les \texttt{<zone>} et les \texttt{<p>} ou les \texttt{<head>} ne peut plus 
être garantie. De fait, le niveau de référence devient la page (\texttt{<surface>}) et non le paragraphe. Les nombreuses modifications que nous avons menées sur ces balises nous ont donc conduit à retirer l'ensemble des éléments \texttt{<zone>} et des attributs \texttt{@facs} des paragraphes et des titres par le biais d'un script. Dans l'arbre XML de départ (\fig{} \ref{fig:xmltree1} p. \pageref{fig:xmltree1}), le seul descendant de l'élément \texttt{<surface>}  est désormais \texttt{<graphic>}.

\chapter[Découpages éditorial et sémantique]{Contrôle des découpages éditorial et sémantique}

Le contrôle du découpage éditorial, \cad{} de l'implémentation de la structure logique, a donné lieu à la reprise la plus importante. Il s'agit d'un point crucial pour les fichiers des \odm, du fait de l'importance accordée par l'école leplaysienne à une structure logique qui seule permet de transformer les \og faits observés \fg{} sur le terrain en des \og faits décris \fg{} par la monographie\footcite[p. 87]{baciocchi2}.

\section{Découpage éditorial}

\subsection{Niveau des titres (\texttt{<head>})}

Pour commencer, nous avons cherché à évaluer l'étendue des corrections à effectuer à l'aide d'un script. Il comparait une structure idéale --- \cad{} l'idée que chaque fichier de monographie devait contenir dans des \texttt{<head>} deux titres de niveau section (B et C), quatre titres de sous-section (I à IV) et seize titres de paragraphes (§1 à §16, \cf{} \ann{} \ref{structure}) --- à la structure réelle contenue les fichiers. Les titres qui n'avaient pas été trouvés étaient listés en sortie.

Le script a fait remonter plus de quatre cent cinquante erreurs. Ce chiffre comptait de nombreux faux positifs en raison de la non-prise en compte de la qualité des transcriptions. Certains titres étaient en effet présents mais leurs transcriptions étaient fautives. Les erreurs principales pouvaient néanmoins être identifiées dès cette étape :

\begin{itemize}
    \item Des distances d'édition trop faible entre deux titres ont causé le remplacement de l'un par l'autre :
    \begin{itemize}
        \item (II.) \og Mode d'existence de la famille \fg{} par (III.) \og Moyens d'existence de la famille \fg.
        \item \og § 11. -- Récréations \fg{} par \og § 7. -- Subventions \fg.
    \end{itemize}
    \item La monographie n° 44\footcite{mono044a} est la dernière à posséder des titres de paragraphe sur une seule ligne, un retour à la ligne étant ensuite systématiquement effectué entre le numéro et le libellé du paragraphe. Cet élément n'a pas été compris par \lse, qui a pu répéter ou non le numéro du paragraphe sur la ligne du libellé.
    \begin{itemize}
        \item En cas de répétition, le numéro original est placé dans un \texttt{<p>} et le titre avec le numéro ajouté dans un \texttt{<head>} (\fig{} \ref{fig:ex_structure} p. \pageref{fig:ex_structure}).
        \item En cas de non répétition, le titre n'est pas toujours détecté et les deux lignes sont encodées dans des \texttt{<p>}.
    \end{itemize}
    \item Des titres transcrits avec une distance trop grande par rapport au modèle n'ont pas été détectés. Ce point concerne l'ensemble des titres, sans prévalence particulière.
    \item Les titres des tableaux de budgets sont transcrits plus d'une fois.
    \item Les cas particuliers n'ont pas été pris en compte :
    \begin{itemize}
        \item La monographie porte sur un ouvrier ou une ouvrière et non pas une famille, ce qui conduit à un changement dans la terminologie (par exemple, \og État civil de la famille \fg{} devient \og État civil de l'ouvrier \fg) : concerne notamment \texttt{s1t2\_chapt\_4.xml}, \texttt{s1t3\_chapt\_7.xml} et \texttt{10} ;
        \item Les titres du paragraphes n'étaient pas séparés du texte mais imprimé en italique au début de celui-ci (\ann{} \ref{ann:structure-allegee}) ;
        \item Le titre n'est pas imprimé (\ann{} \ref{ann:titre-non-imprimes}, \ref{ann:structure-allegee} et \ref{ann:no-notes-no-16}) ;
        \item La monographie possède une structure qui lui est propre (\ann{} \ref{ann:remarques-et-cas-particuliers}).
        \item La sous-section des budgets est titrée dans huit monographies.
    \end{itemize}
\end{itemize}

\subsection{Niveau des divisions (\texttt{<div>})}

L'implémentation de la structure logique ne s'est pas uniquement traduite par une mise en forme des titres : chaque \texttt{<head>} est en effet attaché à une \texttt{<div>} (\fig{} \ref{fig:deficit} p. \pageref{fig:deficit}). Tous deux possèdent un identifiant \texttt{@type} qui définit leur niveau (\texttt{section}, \texttt{sub\_section} ou \texttt{sub\_sub\_section}), la \texttt{<div>} étant en plus nantie d'un numéro (\texttt{@n}).

Trois erreurs ont été constatées à ce niveau :

\begin{itemize}
    \item Lorsqu'un titre n'était pas détecté, la \texttt{<div>} correspondante n'était pas créée et le paragraphe était intégré dans la division précédente, invalidant de fait l'ensemble de la numérotation ;
    \item Des \texttt{<div>} surnuméraires ont été implantées (\fig{} \ref{fig:ex_structure} p. \pageref{fig:ex_structure}) ;
    \item Au cour du stage, il a été décidé de placer le titre de la monographie dans une \texttt{<div>} de type \texttt{section}, ce qui impliquait une reprise dans l'ensemble des fichiers.
\end{itemize}

\subsection{Méthodologie : \textit{tabula rasa}}

Le nombre élevé de cas particuliers rendait impossible une automatisation complète de cette reprise. Nous allons d'abord exposer la méthode suivie pour le traitement de ces cas avant de rendre compte de celle utilisée pour le reste du corpus.

\subsubsection{Cas particuliers}

Les monographies d'ateliers (\texttt{s3t1\_chapt\_2.xml}\footcite{mono472a} et \texttt{s3t2\_chapt\_10.xml}\footcite{mono473a}) imposaient une reprise manuelle du fait de leur structure unique. De plus, ces fichiers, ainsi que certaines monographies de familles, avaient certains de leurs titres qui n'étaient pas détachés des paragraphes. Le recours à la balise \texttt{<head>} n'était pas possible en l'état, dans la mesure où elle constitue normalement un bloc séparé des \texttt{<p>}. Plusieurs solutions, toutes insatisfaisantes, ont été envisagées.

La plus légère consistait à signaler la mise en forme de ces titres (italique ou gras) par des balises de mise en valeur telles que \texttt{<emph style="italic">} ou \texttt{<hi rend="i">}. Nous n'avons pas opté pour ce choix en raison des règles de la TEI qui ne prévoient pas d'attributs pour caractériser ces titres (notamment, \texttt{@type}). Aussi ne serait-il pas possible de les différencier d'autres passages avec une mise en forme similaire dans le corps du texte lors de la transformation des fichiers XML en d’autres formats.

Une deuxième solution, beaucoup plus lourde pour le fichier, aurait consisté en l'imbrication de  \texttt{<div>}. Le \texttt{@type} assigné à la division aurait permis de savoir que le \texttt{<head>} et le \texttt{<p>}, eux-même placés dans une sous-division, ne constituaient qu'un seul paragraphe dans le volume original (\fig{} \ref{fig:solution2}). La contre-partie est que chaque paragraphe aurait dû être placé dans une sous-division.

\begin{figure}[h]
    \centering
    \includegraphics[width=15cm]{img/solution2.png}
    \caption{Représentation de la deuxième solution, dont le résultat est de produire une imbrication malvenue de \texttt{<div>}.}
    \label{fig:solution2}
\end{figure}

Au final, \timeus{} n'a opté pour aucune solution, considérant que le nombre de cas était suffisamment bas pour ne pas avoir à mettre en place une solution particulière. Pour autant, cette décision n'a pas eu pour effet d'arrêter le niveau de granularité des titres à la sous-sous-section.

En effet, les fichiers \texttt{s1t5\_chapt\_9.xml}\footcite{mono041a} et \texttt{10}\footcite{mono042a} comprennent des précis de monographie dotés de leur propre structure logique et intégrés dans le corps de leurs sections \textit{Notes}. Les titres étant détachés des paragraphes, il a été possible de mobiliser un niveau inférieur à la sous-sous-section, le \texttt{paragraph}, afin de rendre compte de cette structuration.

\subsubsection{Reconstruction de la structure}

Ces cas particuliers réglés, il restait à prendre en charge l'essentiel du corpus, \cad{} une soixantaine de monographies. Une reprise entièrement manuelle n'était pas envisageable en raison du temps trop important qu'elle aurait nécessité. À l'inverse, une automatisation totale aurait nécessité une longue étude et un relevé minutieux des aléas de la détection et de la transcription des titres.

Nous avons donc choisi d'implanter la structure logique par des expressions régulières appliquées sur un lot de fichiers à l'aide d'un script. Les expressions régulières sont des motifs qui permettent de décrire des chaînes de caractères dans un texte. Le module \texttt{re} de Python permet de mobiliser ces expressions dans un script, et notamment d'effectuer des substituions grâce à la fonction \texttt{re.sub()}.

Du fait de la désorganisation massive des \texttt{<div>}, la première expression régulière consistait à faire table rase de ces balises. Le script reconstruisait ensuite la structure en progressant selon ses niveaux. Il commençait par implanter la \texttt{<div>} de type \texttt{chapter} et celle de la première section (la page de titre). Venaient ensuite les \texttt{<div>} des deux sections restantes (\textit{Observations préliminaires} et \textit{Notes}), puis celles des sous-sections, etc.

Les expressions régulières recherchaient les libellés des titres lorsqu'il s'agissait de sections ou de sous-sections, car la plupart étaient convenablement transcrits. En revanche, il recherchait le numéro des sous-sous-sections, qui avaient tous pour point commun de commencer par le symbole typographique \texttt{§}. Les paragraphes \textit{Notes} des quatre-vingt-quatre premières monographies ayant pour particularité d'
être numérotés par une lettre entre parenthèses, le motif recherché des premiers lots était légèrement différent.

Une dernière commande, qui affichait la liste des balises \texttt{<head>} de chaque fichier, nous permettait de contrôler l'intégrité de la nouvelle structure. Si une erreur était constatée, nous intervenions manuellement afin de la résoudre. Pour chaque lot (qui correspondait \textit{grosso modo} à un volume), jusqu'à 50\% d'erreur était constaté. Cela était dû à des titres qui ne correspondaient pas aux motifs en raison de l'absence d'un symbole ou d'une transcription trop éloignée.

L'opération s'est conclue par la constitution automatique d'un tableau synoptique, le contenu de chaque balise \texttt{<head>} étant placé dans une cellule. Son étude a permis, d'une part, de corriger les dernières erreurs résiduelles, et d'autre part de rédiger une note exposant l'ensemble des défauts restant (\ann{} \ref{ann:releve_erreurs}).

\section{Découpage sémantique}

L'opération de reconstruction de la structure logique a mis en évidence une erreur qui n'avait pas été détectée jusque là : des en-têtes (rappel du type de chapitre et de son titre, numéro de page ou de la monographie) et des bas de page (numéro du cahier) étaient toujours présents dans le flux du texte. L'élément révélateur a été la mise en forme par \lse{} de plusieurs rappels de titre comme s'il s'agissait de titre à part entière. 

Pour les supprimer, nous avons là encore utilisé des expressions régulières. L'automatisation n'était pas possible pour les numéros de page, de cahier ou de monographie --- par exemple, \texttt{<p>3</p>} --- car certaines transcriptions de tableau avaient produits des données numériques semblables qu'il fallait conserver.

Pour résoudre ce problème, nous avons utilisé la fonction de recherche de l'outil \og projet \fg{} d'\oxygen, qui permet de rechercher une expression dans l'ensemble des fichiers du corpus. Les expressions étaient toutes construites par rapport à la balise \texttt{<pb/>} qui marque un changement de page, et toute occurrence d'une donnée numérique et d'un titre était systématiquement contrôlée (\fig{} \ref{fig:supprnumpage} et \ref{fig:suppnumcahier}).

Cette opération minutieuse a permis de garantir qu'au moment de la transformation des fichiers la reconstitution des paragraphes très peu perturbée par des éléments externes. Des contrôles visuels au cours d'autres opérations ont cependant permis de déceler des numéros subsistants, placés à la fin de certains paragraphes ou bien transcrits par des lettres (\fig{} \ref{fig:suppnumcahiervisu}).

\begin{figure}
    \centering
    \includegraphics[width=15cm]{img/suppr_num_page.png}
    \caption{Suppression du numéro de page 28 dans le fichier  \texttt{s2t1\_chapt\_20.xml} (image du \commit{} sur \gitlab{} --- monographie n° 53).}
    \label{fig:supprnumpage}
\end{figure}

\begin{figure}
    \centering
    \includegraphics[width=16cm]{img/suppr_num _cahier.png}
    \caption{Suppression du numéro de cahier 16 dans le fichier  \texttt{s2t2\_chapt\_10.xml} (image du \commit{} sur \gitlab{} --- monographie n° 59, page 221).}
    \label{fig:suppnumcahier}
\end{figure}

\begin{figure}
    \centering
    \includegraphics[width=16cm]{img/suppr_num _cahier_visuel.png}
    \caption{Suppression du numéro de cahier 36 dans le fichier  \texttt{s2t2\_chapt\_10.xml} après un contrôle visuel (image du \commit{} sur \gitlab{} --- table alphabétique et analytique du deuxième volume de la deuxième série, page 529).}
    \label{fig:suppnumcahiervisu}
\end{figure}


\section{Validation du schéma XML-TEI}

Les erreurs de structuration et de transcription retirées des quatre premiers niveaux d'encodage, nous pouvions envisager de formaliser le schéma des fichiers. Il s'agit d'un document qui définit quelles balises peuvent être utilisées dans les fichiers, l'ordre de leur imbrication, le nombre et la nature des attributs qui peuvent leur être attachés.

Nous avons choisi d'établir un fichier ODD, entièrement écrit dans une syntaxe XML, qui permet d'établir très précisément les éléments, les séquences d'éléments (nombre et ordre) et leurs attributs. \oxygen{} est capable d'en générer un après avoir analysés un ensemble de fichiers TEI, puis de le transformer en un schéma relaxNG. Ce document relaxNG contient le schéma servant à la validation du fichier TEI, l'ODD étant un fichier XML de documentation et de description de ce schéma. Le document relaxNG est finalement associé au document TEI (\fig{} \ref{fig:tei-odd-rng}), une des fonctionnalités d'\oxygen{} permettant de contrôler à tout moment la validité du code par rapport au schéma.

\begin{figure}[h]
    \centering
    \includegraphics[width=5cm]{img/tei-odd-rng.png}
    \caption{Schématisation du processus de validation à l'aide d'une ODD.}
    \label{fig:tei-odd-rng}
\end{figure}

Le document ODD est généré automatiquement par \oxygen{} grâce au scénario de transformation du Consortium TEI intitulé \textit{oddbyexample}. Il s'agit d'une feuille de style XSL qui, appliquée à un document TEI, analyse l'ensemble des balises et leurs attributs pour les comparer aux préconisations de la TEI\footnote{\textit{The XSLT stylesheet which traverses a nominated directory tree looking for *.xml files which have <TEI> or <teiCorpus> root elements. It analyzes the collection of elements and attributes in the resulting corpus, and compares that to the whole of TEI P5.} : \textit{oddbyexample} sur le \textit{TEI Wiki} (\url{https://wiki.tei-c.org/index.php/Oddbyexample}, consulté le \today).}. Puis il génère ensuite une ODD où les modules TEI sont chargés, les balises et les attributs non-utilisés étant supprimés d'office et les valeurs des attributs collectés et inscrites dans des listes.

\begin{figure}[h]
    \centering
    \includegraphics[width=9cm]{img/odd.png}
    \caption{Extrait de l'ODD des \odm{} concernant la balise \texttt{<div>}. Les attributs \texttt{@facs} et \texttt{@source} sont supprimés (\texttt{mode="delete"}), \texttt{@type} est requis (\texttt{usage="req"}) et sa liste de valeurs est close (\texttt{type="closed"}).}
    \label{fig:tei-odd-rng}
\end{figure}

L'utilisateur peut ensuite modifier le document obtenu en imposant des règles plus restrictives ou au contraire en en relâchant certaines ; il peut également ajouter des valeurs d'attributs, rendre des attributs obligatoires ou contraindre l'enchaînement de certaines balises. Par exemple, nous avons rendu l'attribut \texttt{@type} obligatoire pour les balises \texttt{<head>} et \texttt{<div>}, et clos la liste de ses valeurs possibles (\fig{} \ref{fig:tei-odd-rng}).

Une ODD peut également être documentée, \cad{} que son rédacteur peut justifier ses choix d'encodage. Disposer d'un schéma et de sa documentation est un élément extrêmement important pour un projet en humanités numériques : c'est en effet une assurance pour la pérennité des données qu'il a rassemblées et structurées. Celle-ci ne sont plus dépendant des ingénieurs ou des structures institutionnelles et peuvent être réutilisées par d'autres projets\footcite[p. 61]{jolivet}.

Il existe plusieurs possibilités pour permettre la vérification du document TEI. La première et la plus courante est la spécification d'une instruction de traitement au début du document sous la forme suivante : \texttt{<?xml-model href="\#" type="application/xml" schematypens="http://relaxng.org/ns/structure/1.0"?>}, l'attribut \texttt{@href} devant être complété par l'adresse du schéma. Cela suppose bien évidemment que ce schéma soit hébergé, par exemple sur \gitlab. La ligne peut ensuite être ajoutée automatiquement à l'ensemble des fichiers du corpus grâce à la fonction de recherche et de remplacement.

Une autre possibilité est une validation externe, \cad{} qu'un logiciel ou un script vérifie la compatibilité des fichiers par rapport à un schéma qui lui est fourni par l'utilisateur à l'instant \texttt{t} de la validation. C'est une des fonctionnalités du mode \og projet \fg{} d'\oxygen. \timeus{} a cependant souhaité se détacher de tout logiciel, et \textit{a fortiori} d'un logiciel propriétaire, afin de se réserver la possibilité de contrôler la validité de ses fichiers à tout moment.

Le module \texttt{etree} de la librairie Python \texttt{lxml}, qui permet d'analyser un arbre XML, possède un module répondant parfaitement à ce besoin. En effet, un schéma RNG peut être chargé avec la classe \texttt{etree.RelaxNG()}, et la fonction \texttt{validate()} contrôle ensuite la validité de tout fichier lui étant donné comme argument par rapport à ce schéma.

La spécificité de ce script est qu'il s'applique à un corpus entier, ce qui avait pour effet d'allonger démesurément son exécution sur notre ordinateur (environ une demi-heure pour traiter une vingtaine de fichiers). Nous avons donc charger l'ensemble des fichiers, le schéma et le script sur le \textit{cluster} \rioc{} d'Inria, ce qui a considérablement rétréci le temps d'exécution  (vingt-quatre minutes au total).

