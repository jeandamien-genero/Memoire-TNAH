\part{Une structuration à reprendre : des tâches manuelles et semi-automatiques}

\clearpage
\thispagestyle{empty}
\cleardoublepage

\chapter{Gestion de projet et méthodologie de travail}


\section{Outils de développement}

\subsection{\gitlab}

La gestion quotidienne du programme \timeus{} se fait grâce à un dépôt sur \gitlab{}, un logiciel libre permettant aux entités comme Inria de créer une plate-forme interne de développement informatique. Sur \gitlab{} se trouve le dépôt central, organisé en plusieurs dossiers, dont un est réservé aux \odm. La technologie \textit{git} permet d'administrer les différentes versions du projet, sur une échelle à la fois verticale (les anciennes versions, appelées \commits, restant accessibles à travers un historique) et horizontale (le travail peut s'effectuer sur des \textit{branches} divergentes de la branche principale dite \master{} sans affecter l'état de cette dernière). Un commentaire est ajouté par l'utilisateur à chacun de ses \commits{} pour résumer ses modifications.

Chaque participant peut rapatrier le dépôt \gitlab{} en local pour travailler dessus ; ce rapatriement est un \textit{pull}. Il peut ensuite effectuer l'opération inverse, un \textit{push}, consistant à mettre ses \commits{} en ligne. Son équipe peut ainsi prendre connaissance des dernières avancées de la branche sur laquelle il travaille.

Une fois le travail dans une branche achevé, celle-ci peut être fusionnée avec \master. \gitlab{} propose une interface pour effectuer des \mergerequests{}. Il s'agit d'une demande de fusion, où \gitlab{} affiche l'historique de la branche. Les utilisateurs peuvent ainsi contrôler l'ensemble des \commits{} de la branche locale et vérifier qu'ils ne vont pas corrompre \master{} en créant des conflits (\textit{merge conflicts}).

\gitlab{} permet enfin à ses utilisateurs d'exposer un élément posant problème, d'effectuer des suggestions ou encore de porter à l'attention de leur équipe une ressource utile à travers des \issues. Les \issues{} et les \mergerequests{} sont numérotées (\citecode{\#\textbackslash d} et \citecode{!\textbackslash d}\footnote{Par exemple \citecode{\#5} fait référence à la cinquième \issue{} et \citecode{!5} à la cinquième \mergerequest{}.}), écrire leurs numéros dans \gitlab{} ou dans le commentaire de modification d'un \commit{} permettant de faire automatiquement référence à elles à travers un lien interne.

\subsection{Le dépôt \gitlab{} des \odm}

L'équipe ALMAnaCH possède un espace sur la plate-forme \gitlab{} d'Inria, où un dossier (en accès restreint) est réservé au programme \timeus{}. C'est ici, dans un sous-ensemble, que se trouve le dépôt des \odm. Le script \lse{} est déposé dans un dossier externe.

La branche \master{} compte quatre sous-dossiers :

\begin{itemize}
    \item \citecode{source} contient les treize fichiers XML-TEI des volumes des \odm{} ;
    \item \citecode{script} contient les scripts développés afin d'automatiser le traitement des fichiers ;
    \item \citecode{files} contient les fichiers XML-TEI générés à partir des fichiers sources et modifiés automatiquement ou à la main en vue de leur publication (monographies et fichiers de paratexte) ;
    \item \citecode{metadata} contient des fichiers de métadonnées sur le projet.
\end{itemize}

Ce dossier était notre espace de travail principal, notre première action ayant consisté en son rapatriement au niveau local. Notre méthodologie de travail était la suivante :

\begin{enumerate}
    \item Une branche était créée pour chaque mission ;
    \item Des \commits{} étaient effectués en local sur cette branche ;
    \item Les \commits{} d'une journée étaient mise en ligne sur \gitlab{} par un \textit{push} ;
    \item Lorsque la mission était achevée, une \mergerequest{} était ouverte ;
    \item La \mergerequest{} était acceptée et la branche fusionnée avec \master.
\end{enumerate}

\section{Espaces de discussion}

La situation de confinement d'avril à mai et le maintien de la fermeture aux stagiaires des locaux d'Inria de juin à juillet a conduit à la mise en place d'outils de discussion, ou à l'exploitation intensive d'outils pré-existants.

\subsection{\Mattermost}

\Mattermost{} est un logiciel de discussion instantanée dont le code, écrit à l'origine sous un format propriétaire, a été publié en \opensource{}\footnote{Consultable sur \github{} (\url{https://github.com/mattermost/mattermost-server}, consulté le \today).} en 2015\footnote{Lindsay Brock, \textit{Open source Slack-alternative reaches 1.0: Self-host ready, Slack-compatible, MIT licensed}, 2 octobre 2015, \url{https://mattermost.com/blog/mattermost-3-4-16/} (consulté le \today).}. Auto-hébergé --- Inria stocke le code dans ses propres installations et n'a pas recours à un serveur distant ---, il s'agit de l'espace de discussion principal des agents d'Inria.

Composé de plusieurs \og chaînes \fg{} (\textit{public channels}) organisées de façon thématique, il leur permet d'échanger sur les différents projets et de suivre leur avancement, mais aussi d'exposer les difficultés techniques qu'ils rencontrent dans leur travail quotidien afin d'obtenir de l'aide. La chaîne \og \ocr \fg{} est ainsi fréquemment utilisé par les utilisateurs de \kraken{} (\fig{} \ref{fig:mattermost}). Le programme \timeus{} possède également une chaîne.

\begin{figure}
    \centering
    \includegraphics[width=16cm]{img/mattermost.jpg}
    \caption{Exemple de messages sur la chaîne \og \ocr \fg{} du \Mattermost{} d'Inria : une utilisatrice demande de l'aide au sujet d'une erreur dans l'exécution de \kraken{}. Sur le volet gauche se trouve la liste des chaînes disponibles.}
    \label{fig:mattermost}
\end{figure}

\subsection{\textit{Issues} et \mergerequests{} sur \gitlab}

\Mattermost{} permet également d'engager une conversation personnelle avec un utilisateur ; cependant, pour les échanges afférents à nos missions, nous usions des espaces de discussion de \gitlab.

Les \issues{} et les \mergerequests{} n'ont pas pour seule utilité de permettre aux utilisateurs d'exposer des problèmes ou de demander des fusions de branches. Il s'agit d'espaces dynamiques qui participent pleinement à la gestion de projet en offrant aux participants la possibilité de réagir au sujet ou d'apporter des solutions, par exemple pour résoudre un \textit{merge conflict}.

Cet aspect est facilité par l'usage du langage à balises \markdown\footnote{\gitlab{} use de sa propre version de \markdown, le \textit{GitLab Flavored Markdown} : \url{https://docs.gitlab.com/ee/user/markdown.html} (consulté le \today).}. Il permet une mise en forme légère (listes, liens hypertexte, diagramme), ainsi qu'une intégration d'échantillons de code. Si ceux-ci ne sont pas fonctionnels, le \markdown{} permet de leur appliquer une coloration syntaxique, les rendant ainsi plus facile à lire (\fig{} \ref{fig:ex_gitlab}).

\begin{figure}
    \centering
    \includegraphics[width=16cm]{img/gitlab.png}
    \caption{Exemple de messages échangés avec une coloration syntaxique d'un code Python dans une \issue{} sur \gitlab.}
    \label{fig:ex_gitlab}
\end{figure}

\section{Feuille de route}

\subsection{Une gestion de projet ?}

Au commencement de notre stage, huit \issues{} étaient ouvertes sur le \gitlab{} des \odm{}. Chacune correspondait à un problème ou à un point qui n'avait pas encore pu être développé. Elles constituaient donc notre \og feuille de route \fg, \cad{} l'exposé des missions que nous avions à réaliser (\ann{} \ref{ann:feuille_route}). Il nous a été demandé de les utiliser pour poser nos questions ou proposer nos solutions, elles ont ainsi constitué notre principal espace d'échange.

Ce faisant, les \issues{} ont constitué notre principal espace de gestion et de suivi du projet. Nous n'avons donc pas utilisé d'outil véritablement dédié à la gestion de projet, détournant plutôt un outil courant de \gitlab.

Des fonctionnalités de \gitlab{} sont pourtant dédiées à une gestion plus fine. La principale est l'outil \og tableau de bord \fg{} (\textit{boards}). Par défaut, les \issues{} sont affichées sous la forme d'une liste présentant toutes celles qui sont ouvertes, deux onglets permettant d'accéder à celles qui ont été résolues ou bien à la fois celles en cours et celles qui sont fermées. Avec l'outil \og tableau de bord \fg{}, l'utilisateur a accès à un tableau à quatre colonnes, la première listant les issues ouvertes, la seconde affichant une liste de tâches (\textit{todo list}), la troisième présentant les tâches en cours (\textit{doing}) et la dernières les \issues{} terminées.

Au-delà de la présentation des tâches, le tableau de bord est également interactif. Ainsi, sélectionner une \issue{} affiche ses métadonnées (agent en charge de sa résolution, label, temps estimé, échéance). Un système de labels --- des étiquettes thématiques --- permet également de classer les \issues{} et les tâches de la \textit{todo list}.

Le tableau de bord offre donc une vision globale du projet selon une organisation chronologique, tout en permettant des vues synthétiques ciblées. Le fonctionnement des \textit{boards} de \gitlab{} est comparable à celui d'autres applications, comme par exemple \textit{Trello}. Il s'agit d'un organisateur de tâche, également sous forme de tableaux, qui est utilisé par ALMAnaCH et les Archives nationales pour coordonner le projet de lecture automatique des répertoires du Minutier central des notaires de Paris, Lectaurep.

Pour autant, ces outils ne sont utilisés ni dans le cadre général du programme \timeus, ni dans celui des \odm. Plusieurs raisons peuvent l'expliquer. En premier lieu, \timeus{} s'appuie sur une documentation dont le caractère disparate --- tant géographique que chronologique et typologique --- se heurte à toute volonté de gestion centralisée, d'autant que chaque membre est chargé de la gestion de sa documentation locale.

Un tel outil doit être constamment maintenu à jour afin de garantir un gain de productivité. À l'échelle des \odm{} et de notre stage, le nombre relativement faible de missions (8) et d'intervenants (2) ne justifiaient pas la mise en place du tableau de bord \gitlab. L'affiche basique des \issues{} sous forme d'une liste se suffisait à lui-même.

\subsection{Les missions du stage}

Les missions qui nous ont été confiées étaient de deux ordres (\ann{} \ref{ann:feuille_route}).

Une moitié consistait à contrôler les résultats du script \lse, tant au niveau du corpus qu'à celui de chaque fichier. Dans un premier temps, il s'agissait de vérifier de détecter les erreurs de découpage des fichiers de volume et d'opérer les fusions ou séparations nécessaires. Ceci avait pour but de pouvoir identifier fichiers en leur donnant un numéro unique. Dans un second temps, nous nous sommes intéressés à des éléments particuliers dans chaque fichier, à l'instant de l'implémentation de la structure logique, des balises \citecode{<facsimile>} ou encore de la validité du schéma TEI.

L'autre moitié avait pour but de valoriser les données des \odm.

\chapter{Contrôle du découpage des fichiers source}

Début du chapitre 5.

\chapter[Correction des erreurs d'implémentation]{Correction des erreurs d'implémentation de la structure logique}

Début du chapitre 6.