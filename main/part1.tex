\part{Un corpus déjà structuré}

\clearpage
\thispagestyle{empty}
\cleardoublepage
\vspace*{\stretch{0.3}}
Le corpus sur lequel nous avons travaillé est un ensemble de fichiers XML-TEI. Ces derniers résultent d'une version numérisée des \odm{} réalisée et hébergée par le site \ia, elle-même issue du traitement des volumes physiques de l'Université de Toronto. Dans cette première partie, nous allons décrire ces différents états des \odm{} et commenter les opérations qui ont mené à leur structuration, tout en interrogeant les raisons ayant conduit le programme \timeus{} à les privilégier par rapport à d'autre version similaires.
\vspace*{\stretch{1.7}}

\chapter{Un corpus d'imprimés}

La publication des \odm{} s'échelonne sur soixante-treize années (1857-1930). Divisée en trois séries, son mode de publication évolue à partir de 1885  : là  où les monographies de la première série avaient été directement publiées en volume (1857-1885), celles des deuxième (1887-1899) et troisième série (1904-1928) sont publiées sous forme de fascicules trimestriels qui sont ensuite reliés en volumes\footnote{\cite[p. 124]{lorry}.}.

Cette évolution entraîne des conséquences au niveau codicologique. Les reliures des volumes des deux dernières séries peuvent en effet changer d'un corpus à l'autre. Notamment, les relieurs ont pu choisir de conserver ou de ne pas conserver les premières ou  quatrièmes de couverture des fascicules, ainsi que les pages consacrées au rappel des fascicules antérieurs ou à de la publicité. Il n'existe donc pas un modèle ayant autorité dans la reliure des \odm{}.

\chapter{Des numérisations multiples}

Début du deuxième chapitre

\chapter{Un encodage automatique}

Début du troisième chapitre