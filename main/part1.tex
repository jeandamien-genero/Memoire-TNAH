\part{Un corpus déjà structuré}

\clearpage
\thispagestyle{empty}
\cleardoublepage
\vspace*{\stretch{0.3}}
Le corpus sur lequel nous avons travaillé est un ensemble de fichiers XML-TEI. Ces derniers résultent d'une version numérisée des \odm{} réalisée et hébergée par le site \ia, elle-même issue du traitement des volumes physiques de l'Université de Toronto. Dans cette première partie, nous allons décrire ces différents états des \odm{} et commenter les opérations qui ont mener à leur structuration, tout en interrogeant les raisons ayant conduit le programme \timeus{} à les privilégier par rapport à d'autre version similaires.
\vspace*{\stretch{1.7}}

\chapter{Un corpus d'imprimés}

Début du premier chapitre

\chapter{Des numérisations multiples}

Début du deuxième chapitre

\chapter{Un encodage automatique}

Début du troisième chapitre